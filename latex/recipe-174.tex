\begin{recipe}{鱼羊肚烩}

\ingredients

\ingredient{鱼肚}{二两}
\ingredient{羊肚}{七两}
\ingredient{火腿}{一两}
\ingredient{去亮冬勞}{一两}
\ingredient{豌豆尖苞}{约十根}
\ingredient{特级奶汤}{一斤半}
\ingredient{化猪油}{一斤耗二两}
\ingredient{味精}{二分}
\ingredient{胡椒面}{二分}
\ingredient{盐}{三分}
\ingredient{料酒}{三钱}
\ingredient{鸡油}{一两}
\ingredient{清汤}{半斤}

\preparation

\step 将锅放在炉上,倒入化油,烧至四成火,投入洗净的鱼肚,微火慢慢地炸泡;泡后
捞起,放在温热水内泡胀,再切成一寸五的斧头块。又将羊肚洗净,切成一寸五长、五分
宽,分幵装入碗内;再将料酒、盐、姜、葱、清汤放入羊肚碗内上笼蒸粑。火腿、冬笋切
成一寸五的薄片,选好豌豆尖苞待用」

\step 将锅放在旺火上,倒入清汤、胡椒、盐、料酒、味精,烧至微开,倒下鱼肚,烧三
分钟,捞起鱼肚,倒去锅内的汤不用;再将锅放在旺火上,放入化油烧至五成火,投入姜
、葱、奶汤,再捞起姜、葱不用,将汤烧至微开投入火腿、冬笋片、盐、味精、料酒、胡
椒,再将鱼、羊肚一并投入锅内,约煮五分钟放入豌豆尖苞,淋入鸡油,起锅入大汤盘内
即成。

\features

味浓、汤白;鲜美可口,适宜冬季,特受老年人欢迎。

\end{recipe}

% vim: filetype=tex noautoindent nojoinspaces
% vim: fileencoding=utf-8 formatoptions+=m
% vim: textwidth=78 tabstop=4 shiftwidth=4 softtabstop=4
