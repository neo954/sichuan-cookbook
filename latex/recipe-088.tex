\begin{recipe}{羊耳鸡卷}

\ingredients

\ingredient{生鸡脯肉}{半斤}
\ingredient{慈蒜(去皮)}{二两}
\ingredient{鸡蛋}{三个}
\ingredient{干豆粉}{六钱}
\ingredient{料酒}{三钱}
\ingredient{味精}{二分}
\ingredient{胡椒}{一分}
\ingredient{盐}{五分}
\ingredient{酱油}{二钱}
\ingredient{生菜}{四两}
\ingredient{网油}{一斤}
\ingredient{白糖}{二钱}
\ingredient{醋}{二钱}
\ingredient{香油}{三钱}
\ingredient{菜油}{一斤}
\ingredient{花椒面}{三分}

\cooking

\step 鸡脯肉洗净,片成七分宽的薄片,放在碗内,用盐、. 料酒、酱油、味精、胡椒,拌和均匀待用。

之.蛋清和豆粉调成蛋清豆粉;用三分之一的蛋清豆粉将 1 鸡脯肉拌匀;把网油平铺于案上,去掉油硬,将余下的蛋清 豆粉抹在网油上,将拌好的鸡脯平铺于网油上,再铺上慈碌 片,面上再盖上一层鸡脯,然后将网油裹成八分宽的扁形, 照样裹完成鸡卷。

\step 锅在旺火上放下菜油一斤烧至六成热时,将裹好的鸡 卷一根一根地放入锅内炸至金黄色,捞起放在墩子上切成,斜 方块,放入条盘内的一头;再将生菜洗净,挤干水份,拌上 糖、醋、香油放在鸡卷的另一头;走菜时外配椒盐碟子一个 即成。

\notes

颜色金黄,外脆内嫩,适宜佐酒。

\end{recipe}

% vim: filetype=tex noautoindent
% vim: fileencoding=utf-8
% vim: textwidth=78 tabstop=4 shiftwidth=4 softtabstop=4
