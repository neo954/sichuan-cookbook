\begin{recipe}{凤尾鸽蛋}

\ingredients

\ingredient{鸽蛋}{二十个}
\ingredient{鸡蛋}{一个}
\ingredient{熟火腿}{二两}
\ingredient{丝瓜}{三两}
\ingredient{鲜笋}{二两}
\ingredient{猪瘦肉}{三两}
\ingredient{生鸡脯肉}{三两}
\ingredient{化猪油}{五钱}
\ingredient{盐}{四分}
\ingredient{清汤}{一斤半}
\ingredient{胡椒面}{一分}
\ingredient{料酒}{二钱}
\ingredient{味精}{三分}

\cooking

\step 9用一个鸡蛋摊成蛋皮。丝瓜刮后车下瓜皮,用沸水沮 熟捞起,切成二寸长的段,用清水漂起。将鲜笋煮熟,火腿 连同丝瓜皮等都切成二寸长的细丝,分开装入盘内待用。

\step 猪瘦肉、生鸡脯肉各捶成茸子,分开装入碗内。用二 十根瓷调羹,抹上一层化油,依次摆入圆盘内,将切好的蛋 皮、丝瓜、火腿、鲜笋丝各放三根在调羹把上头,二十根同样 放完,再打破鸽蛋,逐一放在调羹内,将丝的一头压着,上 笼蒸五分钟即熟。取出调羹内的鸽蛋,装入一个碗内用清汤 漂起。

炒锅放于旺火上,放入一斤半清汤、料酒、盐,再将 捶好的猪瘦肉茸子用清水解散,倒入锅内用汤瓢推动数转, 肉沫浮起打尽;再将鸡脯肉用清水解散倒入锅内,肉沫浮起 打尽杂质。装鸽蛋的碗泌去清汤,再用清汤过一次,翻入大 碗内,放入味精,将锅内的清汤倒入鸽蛋碗内即成。

\notes

汤清洁白,蛋嫩色美,鲜香可口,特别适宜老年人。

鸽蛋	十二个	土司	一碎
鸡晡肉	一'两五	肥膘	一两
鸡蛋	二个	火腿	一'两
丝瓜	一根	化猪油	二斤4毛二两
净生菜	二两	白糖	五钱
盐	三分	料酒	三钱
水豆粉	五钱

\cooking

\step 鸽蛋洗干净,用碗装起,盛入清水,淹过鸽蛋,上笼蒸 二十分钟取下,用清水冷却,轻轻将蛋壳剥去。细心地不要将 鸽蛋剥滥,保持完整。用刀逢中切开,两头大小相等。鸡蛋 二个摊成蛋皮,连同丝瓜皮、火腿各修切成小瓜子片、鱼眼 睛片,各七十二片待用。

土司切成一寸二宽、一寸八长、一分五厚的旗子块二 十四片。鸡脯、肥膘、蛋清等搅成“鸡糁”。每片土司上先 涂上一层鸡糁,约二分厚;再将半边鸽蛋嵌在中间,紧挨鸽 蛋的周围点缀小瓜子片和鱼眼睛片,增加美观,上笼气五分 钟成半制成品取下。

化猪油在旺火上烧至六成热时,即将蒸好的半制成品 取出,鸽蛋向下,土司向上放入油锅内炸;炸时用小铲细心 翻动,呈金黄色时即涝起在盘中摆好,两端镶上生菜即成。

\notes

形态美观,脆嫩可口。

\end{recipe}

% vim: filetype=tex noautoindent
% vim: fileencoding=utf-8
% vim: textwidth=78 tabstop=4 shiftwidth=4 softtabstop=4
