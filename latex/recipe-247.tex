\begin{recipe}{凤尾鸽蛋}

\ingredients

\ingredient{鸽蛋}{二十个}
\ingredient{鸡蛋}{一个}
\ingredient{熟火腿}{二两}
\ingredient{丝瓜}{三两}
\ingredient{鲜笋}{二两}
\ingredient{猪瘦肉}{三两}
\ingredient{生鸡脯肉}{三两}
\ingredient{化猪油}{五钱}
\ingredient{盐}{四分}
\ingredient{清汤}{一斤半}
\ingredient{胡椒面}{一分}
\ingredient{料酒}{二钱}
\ingredient{味精}{三分}

\cooking

\step 9用一个鸡蛋摊成蛋皮。丝瓜刮后车下瓜皮,用沸水沮熟捞起,切成二寸长的段,用清水漂起。将鲜笋煮熟,火腿连同丝瓜皮等都切成二寸长的细丝,分开装入盘内待用。

\step 猪瘦肉、生鸡脯肉各捶成茸子,分开装入碗内。用二十根瓷调羹,抹上一层化油,依次摆入圆盘内,将切好的蛋皮、丝瓜、火腿、鲜笋丝各放三根在调羹把上头,二十根同样放完,再打破鸽蛋,逐一放在调羹内,将丝的一头压着,上笼蒸五分钟即熟。取出调羹内的鸽蛋,装入一个碗内用清汤漂起。

炒锅放于旺火上,放入一斤半清汤、料酒、盐,再将 捶好的猪瘦肉茸子用清水解散,倒入锅内用汤瓢推动数转, 肉沫浮起打尽;再将鸡脯肉用清水解散倒入锅内,肉沫浮起 打尽杂质。装鸽蛋的碗泌去清汤,再用清汤过一次,翻入大 碗内,放入味精,将锅内的清汤倒入鸽蛋碗内即成。

\notes

汤清洁白,蛋嫩色美,鲜香可口,特别适宜老年人。

\end{recipe}

% vim: filetype=tex noautoindent
% vim: fileencoding=utf-8
% vim: textwidth=78 tabstop=4 shiftwidth=4 softtabstop=4
