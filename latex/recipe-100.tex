\begin{recipe}[叫化鸡]{泥糊鸡}

\ingredients

\ingredient{仔鸡一只}{约三斤}
\ingredient{猪肥瘦肉}{二两}
\ingredient{泡辣椒}{五钱}
\ingredient{芽菜}{一两五}
\ingredient{姜}{五钱}
\ingredient{葱}{五钱}
\ingredient{花椒}{约十粒}
\ingredient{酱油}{一两}
\ingredient{料酒}{五钱}
\ingredient{生菜}{三两}
\ingredient{白糖}{三两}
\ingredient{醋}{三钱.}
\ingredient{香油}{三钱}
\ingredient{菜油}{一两}
\ingredient{盐}{一两}
\ingredient{荷叶}{六张.}

\ingredient{土饼子}{三个}
\ingredient{细麻绳}{一1两}

\cooking

\step 仔鸡剖腹去脏,洗净滴干,砍去头足支翅,两腿剖开剔去棒子骨,用酱油、花椒、料酒、拍松的姜、葱和勻。将鸡周身内外抹匀,装入碗内渍起。

\step 芽菜洗净盐沙,泡辣椒去蒂去籽及猪肉分别剁细,先将猪肉下入菜油锅熵去血水,加酱油烹入料酒,再下芽菜、辣椒,造转扞起。

\step 鸡渍一小时后,提起抖掉葱、姜、花椒,将熵好的馅子装入腹内。

\step 选约一尺过心(宜大张新鲜无孔的荷叶(若是鲜荷叶先上笼汽一分钟)洗净抹干,将鸡放入包紧,共包六层,每层封口交叉,最后用麻绳浑身缠紧,糊上发好加盐的土饼泥(揉糙无硬颗),放在炭火炉上烤至泥巴大干。火力宜小,随烤随翻,约一小时后即剥开,剔下鸡肉宰成一字条,装在盘内横排一条线。

\step 生菜洗净滴干,加糖、醋、香油拌匀,连同鸡腹馅子各镶一边即成。

\notes

味鲜香嫩,别具风格。

\end{recipe}

% vim: filetype=tex noautoindent
% vim: fileencoding=utf-8 formatoptions+=m
% vim: textwidth=78 tabstop=4 shiftwidth=4 softtabstop=4
