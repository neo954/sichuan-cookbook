\begin{recipe}{松鼠鱼}

\ingredients

\ingredient{大輕鱼一尾}{二斤半}
\ingredient{菜油}{二斤耗三两}
\ingredient{酱油}{二两}
\ingredient{白糖}{二两}
\ingredient{醋}{五钱}
\ingredient{姜、葱、蒜}{各五分}
\ingredient{水豆粉}{二两}
\ingredient{香油}{五钱}
\ingredient{火腿}{五钱}
\ingredient{鲜笋}{一两}
\ingredient{鲜碗豆米}{一两}
\ingredient{料酒}{五钱}
\ingredient{味精}{二分}
\ingredient{胡椒}{二分}

\preparation

\step 鲤鱼剖腹、去鱗、去头,清洗干净,从背脊逢中对开成两个半边,但各边应保持一
半完整的鱼尾,去骨去刺;去尽后用刀在腹面划成大荔枝形,不要伤着鱼皮,用葱、姜、
料酒、豆油一半将鱼身抹遍,浸渍十分钟;火腿、鲜笋各切成小丁;其余葱、姜、蒜切成
姜蒜米、葱花。

\step 菜油在旺火上烧至七成火候,鱼先在条盘内用干水豆粉抹匀(皮向内,花子向外,
卷成圆筒形),下油锅炸至呈金黄色时起锅摆在盘内。

\step 锅内留原油二两,先将火腿丁、鲜笋丁、豌豆米(去

\features

形似松厚,外酥内嫩,甜酸味道。

\end{recipe}

% vim: filetype=tex noautoindent
% vim: fileencoding=utf-8 formatoptions+=m
% vim: textwidth=78 tabstop=4 shiftwidth=4 softtabstop=4
