\begin{recipe}{锅贴腰片}

\ingredients

\ingredient{猪腰}{四两}
\ingredient{猪肥膘肉}{一斤(连皮)}
\ingredient{熟火腿}{五钱}
\ingredient{鸡蛋清}{二个}
\ingredient{干豆粉}{八钱}
\ingredient{韭菜}{二两}
\ingredient{猪油}{三钱}
\ingredient{酱油}{一两}
\ingredient{醋}{二钱}
\ingredient{香油}{三钱}
\ingredient{盐}{二分}
\ingredient{料酒}{少许}
\ingredient{姜}{一钱}
\ingredient{葱}{一銬}

\preparation

\step 猪肥膘煮熟,晾冷去皮,修整齐;火腿切细末;韭菜只用白头子,切为磉磴节子;
调蛋清豆粉。
\step 修好的肥膘片成一寸二宽、一寸五长、一分厚的薄片;猪腰的片法稍小于肥膘,各
为二十四片。猪腰用白酱油、料酒、姜葱先调拌均匀入味。用净布把猪肥膘的油揩去,一
面涂抹蛋清豆粉后放火腿末少许,再把腰片揩干水份放上面与肥膘粘拢。
\step 中火。用猪油浪匀炙锅。逐一把二十四个粘好的腰择贴于锅内炕起,火不大不小,
慢慢使肥膘的汕浸出部分,肥膘逐渐炕黄,腰片随之至熟即起锅,镶生菜入席。

\features

香酥,脆嫩,鲜美可口。

\end{recipe}

% vim: filetype=tex noautoindent nojoinspaces
% vim: fileencoding=utf-8 formatoptions+=m
% vim: textwidth=78 tabstop=4 shiftwidth=4 softtabstop=4
