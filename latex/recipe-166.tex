\begin{recipe}{红烧鹿筋}

\ingredients

\ingredient{鹿筋一副}{二斤}
\ingredient{肥母鸡}{一只}
\ingredient{猪肉}{一斤}
\ingredient{乌鸡白菜}{三棵}
\ingredient{料酒}{半斤}
\ingredient{姜}{二两}
\ingredient{大葱}{四根}
\ingredient{胡椒面}{三分}
\ingredient{味精}{三分}
\ingredient{盐}{六分}
\ingredient{酱油}{一两}
\ingredient{化猪油}{二两}
\ingredient{二汤}{二十一斤}

\cooking

\step 鹿筋用清水泡一小时,撕去筋上的鹿肉、油皮及一切杂质,切成一寸五长的段。猪肉选连皮五花肉,刮洗干净,切成一'寸五长、二分宽、八分厚的条块。肥母鸡切成七分长、四分大的条方块。乌鸡白菜去外叶用嫩心,每棵得心二两左右。姜切成五块,每块重约四钱,拍松。

之.锅在旺火上烧红,放入猪油,先用姜四钱、葱一根下 锅稍炸,随即放入二汤、料酒,再将鹿筋段全部放入汤中, 煮十五分钟捞出;锅中各料倒去不用。按上述用料和作法再 继续煮三次,把鹿筋捞出待用。

\step 将猪油放入锅中,再放入猪肉、鸡块煸炒,然后加入姜、料酒、二汤和味精、酱油、盐、鹿筋段等,烧开后全部舀入砂锅中,用微火煨至鹿筋粑时为度。

\step 将另一口锅放在炉上,舀入四两烧鹿筋的汤,再将淘洗干净的乌鸡白菜心放入,煮粑后起锅,盛入大圆盘中。然后把砂锅中鸡块捞出,盖在白菜上;再把砂锅中鹿筋捞出,有次序地摆在鸡块上。最后将砂锅中的汤泌入炒锅中,用旺火熬酽,淋于盘中菜上入席。

\notes

鹿筋属山珍之一,富于营养,故较名贵。因鹿筋有膻: 味,以红烧为适宜。

\end{recipe}

% vim: filetype=tex noautoindent
% vim: fileencoding=utf-8
% vim: textwidth=78 tabstop=4 shiftwidth=4 softtabstop=4
