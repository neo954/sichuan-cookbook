\begin{recipe}{松花肉}

\ingredients

\ingredient{猪肥瘦肉}{二两五}
\ingredient{五香面}{一分}
\ingredient{水发口笨}{三钱}
\ingredient{鸡蛋}{六个}
\ingredient{味精'}{三分}
\ingredient{葱花}{三钱}
\ingredient{酱油}{三钱}
\ingredient{料酒}{二钱}
\ingredient{面粉}{一-两}
\ingredient{白糖}{一分}
\ingredient{化猪油}{一斤二两}
\ingredient{耗二两五}{盐二分}
\ingredient{豆尖}{数根}

\cooking

\step 蛋黄调散,蛋清快速调成蛋泡。面粉过箩筛,同味精、盐慢慢加入蛋泡内,用筷子轻轻调匀待用。

\step 猪肉、冬笋、口茉,分别用刀剁碎。锅内放猪油少许,将猪肉调匀,依次加入蛋黄、冬笋、口茉、料酒及葱花、酱油、白糖、五香面等,炒熟成焰子,起锅待用。

\step 炒锅置于文火上,将猪油烧至三成热,把调好的蛋泡倒入一半,煎成圆形、直径约六寸的蛋饼。将炒熟的馅子,倒于蛋饼中心,立即将另一半蛋泡盖于馅子上,并按上鲜豆尖。同时另用一锅将全部猪油烧沸,再慢慢舀淋入锅内的蛋泡上。油舀完约五分钟,泌去油,扞入盘内即成。

\notes

泡嫩,鲜香,美观适口

\end{recipe}

% vim: filetype=tex noautoindent
% vim: fileencoding=utf-8
% vim: textwidth=78 tabstop=4 shiftwidth=4 softtabstop=4
