\begin{recipe}{蜂窝豆腐}

\ingredients

\ingredient{鲜虾仁}{三两}
\ingredient{肥膘}{二两}
\ingredient{豆腐}{四个}
\ingredient{鸡蛋清}{五个}
\ingredient{膪肝}{二个}
\ingredient{熟猪肚}{五钱}
\ingredient{脑花}{一个}
\ingredient{熟鸡肉}{一两}
\ingredient{瑶蛀}{三钱}
\ingredient{熟蹄筋}{一两}
\ingredient{熟猪舌}{一两}
\ingredient{猪腰}{}
\ingredient{熟火腿}{三钱}
\ingredient{水发口笨}{五钱}
\ingredient{面粉}{-一钱五}
\ingredient{豆尖}{数根}
\ingredient{盐}{八分}
\ingredient{味精}{五分}
\ingredient{花椒}{数颗}
\ingredient{姜、葱}{各五钱}
\ingredient{好鸡汤}{一斤半}

\cooking

\step 以上主料除火腿、口茉外都洗净、微煮,除去腥味启,改成五分见方的方块,加味上笼蒸粑。口茉剪成小蜂窝形。火腿切成指甲壳待用。

\step 用虾仁一半与肥膘、蛋清二个搅成“虾糁”,豆腐去皮过萝,混同虾糁搅匀。用大圆盘一个,先涂上少许一层油,将搅合的虾糁按直径六寸面积用小刀刮平刮圆,上笼蒸五分钟

取下。鸡蛋清三个加面粉,快速调成蛋泡,再盖于糁上刮平 整,又上笼气二分钟取下,用筷子在上面插成均匀整齐的小 孔,如蜂窝状。留下的一半虾仁,在沸水内冒成红色,捞起,晾 干水气,即按每个蜂窝孔嵌入一半在内。走菜时,先在笼上 取下什景,垫入碗底,上面即是蜂窝豆腐糁,周围镶火腿、口 茉、豆尖,再灌入烧好的清汤即成。

\notes

菜美汤清,色鲜味佳,酒饭均宜。

\end{recipe}

% vim: filetype=tex noautoindent
% vim: fileencoding=utf-8
% vim: textwidth=78 tabstop=4 shiftwidth=4 softtabstop=4
