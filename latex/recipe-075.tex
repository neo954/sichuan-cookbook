\begin{recipe}[生烧大转弯]{生烧鸡翅}

\ingredients

\ingredient{公鸡翅八对}{约一半}
\ingredient{水发口蘑}{一两八}
\ingredient{冬笋剥皮去根}{二两五}
\ingredient{葱白}{六钱}
\ingredient{料酒}{一两五}
\ingredient{酱油}{三钱}
\ingredient{味精}{一分五}
\ingredient{化猪油}{一两}
\ingredient{火腿}{六钱}
\ingredient{盐}{一分五}
\ingredient{姜}{一钱五}
\ingredient{水豆粉}{二钱}
\ingredient{鸡汤}{一斤}
\ingredient{香油}{二钱}

\preparation

\step 将宰好去毛的生公鸡,用刀齐鸡身取下翅膀八对,洗净揩干,在酒精火上燎去细
毛,切去翅尖部分不用;每个翅膀从关节上一点切成两段(两段长短大致相等),再去
掉两头关节骨,放入锅内煮三分钟捞出,除去血腥水。火腿切成一寸长、两分宽、一分
厚的片。冬笋切成一寸长、宽厚各二分的片。口蘑大的可用刀对切成两瓣。

\step 猪油放入锅内置于旺火上,将拍松的姜及葱白、料酒、鸡翅倒入锅内稍煵(约三分
钟),再加入酱油、盐和鸡汤等,烧开后用汤瓢撇去浮沫,倒入另外一个锅内,用微火烧
约一点半钟。直至肉将离骨而未离骨时,去掉姜、葱,再倾入炒锅内,即放味精及水豆
粉、勾芡,起锅时淋下香油上盘。

\features

此菜选料为鸡全身的活动部分,再经生烧久焖,入口更觉稠烂香嫩,色鲜味浓,非常可口。

\end{recipe}

% vim: filetype=tex noautoindent nojoinspaces
% vim: fileencoding=utf-8 formatoptions+=m
% vim: textwidth=78 tabstop=4 shiftwidth=4 softtabstop=4
