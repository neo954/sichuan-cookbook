\begin{recipe}{豆渣烘猪头}

\ingredients

\ingredient{猪头肉}{一斤二两}
\ingredient{冰糖}{^一两}
\ingredient{花椒}{~}
\ingredient{\}{十五粒	大葱(切成小节)	六钱}

\ingredient{小葱(切葱花)}{四钱}
\ingredient{豆渣}{七两}
\ingredient{绍酒}{一^两}
\ingredient{老姜(拍破〉}{四钱}
\ingredient{猪油}{二两五}
\ingredient{食盐}{酌量}

\cooking

将猪头的毛去尽,刮洗干净,在锅内煮到七分火色,以 刀切成寸多长的方形块子,置鏆或锑锅中,加清水约二中碗, 将冰糖炒化和绍酒、花椒、老姜、大葱、食盐一齐放下,用 武火烧开后,以文火烘约二小时半,到软如豆腐状时为止。 将肉放在碗内,放时有皮的一面向碗底,用漏瓢去掉汤内的 花椒、老姜、大葱等,把原汁淋在肉上,并放在蒸笼里去蒸。

将豆渣用清水淘洗干净,以筲箕滤干,用猪油在锅里烧 辣放入豆渣,炒干水气,到豆渣已酥,吐油成黄色时,将猪 头肉取出,并将其原汁倒进豆渣里,同时放进葱花、味精少 许及适当数量的食盐,稍炒后,盛入猪头肉上,然后翻盛盘 中供食。切忌久炒。吃时可以荷叶饼佐食。

\notes

肉质柔糯,肥而不腻,豆渣酥香。

\end{recipe}

% vim: filetype=tex noautoindent
% vim: fileencoding=utf-8
% vim: textwidth=78 tabstop=4 shiftwidth=4 softtabstop=4
