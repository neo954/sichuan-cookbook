% BSD 3-Clause License
%
% Copyright (c) 2023 Quux System and Technology. All rights reserved.
%
% Redistribution and use in source and binary forms, with or without
% modification, are permitted provided that the following conditions are met:
%
% 1. Redistributions of source code must retain the above copyright notice, this
%    list of conditions and the following disclaimer.
%
% 2. Redistributions in binary form must reproduce the above copyright notice,
%    this list of conditions and the following disclaimer in the documentation
%    and/or other materials provided with the distribution.
%
% 3. Neither the name of the copyright holder nor the names of its
%    contributors may be used to endorse or promote products derived from
%    this software without specific prior written permission.
%
% THIS SOFTWARE IS PROVIDED BY THE COPYRIGHT HOLDERS AND CONTRIBUTORS "AS IS"
% AND ANY EXPRESS OR IMPLIED WARRANTIES, INCLUDING, BUT NOT LIMITED TO, THE
% IMPLIED WARRANTIES OF MERCHANTABILITY AND FITNESS FOR A PARTICULAR PURPOSE ARE
% DISCLAIMED. IN NO EVENT SHALL THE COPYRIGHT HOLDER OR CONTRIBUTORS BE LIABLE
% FOR ANY DIRECT, INDIRECT, INCIDENTAL, SPECIAL, EXEMPLARY, OR CONSEQUENTIAL
% DAMAGES (INCLUDING, BUT NOT LIMITED TO, PROCUREMENT OF SUBSTITUTE GOODS OR
% SERVICES; LOSS OF USE, DATA, OR PROFITS; OR BUSINESS INTERRUPTION) HOWEVER
% CAUSED AND ON ANY THEORY OF LIABILITY, WHETHER IN CONTRACT, STRICT LIABILITY,
% OR TORT (INCLUDING NEGLIGENCE OR OTHERWISE) ARISING IN ANY WAY OUT OF THE USE
% OF THIS SOFTWARE, EVEN IF ADVISED OF THE POSSIBILITY OF SUCH DAMAGE.
%
\begin{recipe}{豆渣烘猪头}

\ingredients

\ingredient{猪头肉}{一斤二两}
\ingredient{冰糖}{一两}
\ingredient{花椒}{十五粒}
\ingredient{大葱(切成小节)}{六钱}
\ingredient{小葱(切葱花)}{四钱}
\ingredient{豆渣}{七两}
\ingredient{绍酒}{一两}
\ingredient{老姜(拍破)}{四钱}
\ingredient{猪油}{二两五}
\ingredient{食盐}{酌量}

\preparation

将猪头的毛去尽,刮洗干净,在锅内煮到七分火色,以刀切成寸多长的方形块子,置罐或
铝锅中,加清水约二中碗,将冰糖炒化和绍酒、花椒、老姜、大葱、食盐一齐放下,用武
火烧开后,以文火烘约二小时半,到软如豆腐状时为止。将肉放在碗内,放时有皮的一面
向碗底,用漏瓢去掉汤内的花椒、老姜、大葱等,把原汁淋在肉上,并放在蒸笼里去蒸。

将豆渣用清水淘洗干净,以筲箕滤干,用猪油在锅里烧辣放入豆渣,炒干水气,到豆渣已
酥,吐油成黄色时,将猪头肉取出,并将其原汁倒进豆渣里,同时放进葱花、味精少许及
适当数量的食盐,稍炒后,盛入猪头肉上,然后翻盛盘中供食。切忌久炒。吃时可以荷叶
饼佐食。

\features

肉质柔糯,肥而不腻,豆渣酥香。

\end{recipe}

% vim: filetype=tex noautoindent nojoinspaces
% vim: fileencoding=utf-8 formatoptions+=m
% vim: textwidth=78 tabstop=4 shiftwidth=4 softtabstop=4
