\begin{recipe}{口袋豆腐}

\ingredients

\ingredient{嫩豆腐}{三块}
\ingredient{剔刺鱼肉}{一两八}


\ingredient{豆粉(碾细〉}{六钱}
\ingredient{小苏打粉(艰细〉}{少许}
\ingredient{鸡蛋(取白搅散)}{三个}
\ingredient{火腿(宰细〉}{少许}
\ingredient{浓母鸡汤汁}{三杓}
\ingredient{绍酒}{三钱}
\ingredient{食盐}{少许}
\ingredient{胡椒}{少许}

\cooking

豆腐削去周围的表皮用刀背捣烂,挤干水份。以刀背将 鱼肉捶绒,连同豆粉、苏打、食盐、火腿和蛋清等加入豆腐 内和拌均匀。用调羹连续盛入约七成滚的油内炸呈鸭黄色时 捞起,滤干,倾入烧沸的肉汤里川一道,以去其油味,捞起 再倾入浓鸡汤汁内,加入绍酒用武火(不旺不小的火)煮约 五分钟即成。起锅时才放进胡椒盛入碗内。

\notes

味浓厚鲜美,富营养,内含浆包细嫩化渣,状似米口袋, 故名“ 口袋豆腐”。

\end{recipe}

% vim: filetype=tex noautoindent
% vim: fileencoding=utf-8
% vim: textwidth=78 tabstop=4 shiftwidth=4 softtabstop=4
