% BSD 3-Clause License
%
% Copyright (c) 2023 Quux System and Technology. All rights reserved.
%
% Redistribution and use in source and binary forms, with or without
% modification, are permitted provided that the following conditions are met:
%
% 1. Redistributions of source code must retain the above copyright notice, this
%    list of conditions and the following disclaimer.
%
% 2. Redistributions in binary form must reproduce the above copyright notice,
%    this list of conditions and the following disclaimer in the documentation
%    and/or other materials provided with the distribution.
%
% 3. Neither the name of the copyright holder nor the names of its
%    contributors may be used to endorse or promote products derived from
%    this software without specific prior written permission.
%
% THIS SOFTWARE IS PROVIDED BY THE COPYRIGHT HOLDERS AND CONTRIBUTORS "AS IS"
% AND ANY EXPRESS OR IMPLIED WARRANTIES, INCLUDING, BUT NOT LIMITED TO, THE
% IMPLIED WARRANTIES OF MERCHANTABILITY AND FITNESS FOR A PARTICULAR PURPOSE ARE
% DISCLAIMED. IN NO EVENT SHALL THE COPYRIGHT HOLDER OR CONTRIBUTORS BE LIABLE
% FOR ANY DIRECT, INDIRECT, INCIDENTAL, SPECIAL, EXEMPLARY, OR CONSEQUENTIAL
% DAMAGES (INCLUDING, BUT NOT LIMITED TO, PROCUREMENT OF SUBSTITUTE GOODS OR
% SERVICES; LOSS OF USE, DATA, OR PROFITS; OR BUSINESS INTERRUPTION) HOWEVER
% CAUSED AND ON ANY THEORY OF LIABILITY, WHETHER IN CONTRACT, STRICT LIABILITY,
% OR TORT (INCLUDING NEGLIGENCE OR OTHERWISE) ARISING IN ANY WAY OUT OF THE USE
% OF THIS SOFTWARE, EVEN IF ADVISED OF THE POSSIBILITY OF SUCH DAMAGE.
%
\begin{recipe}{口袋豆腐}

\ingredients

\ingredient{嫩豆腐}{三块}
\ingredient{剔刺鱼肉}{一两八}
\ingredient{豆粉(碾细)}{六钱}
\ingredient{小苏打粉(碾细)}{少许}
\ingredient{鸡蛋(取白搅散)}{三个}
\ingredient{火腿(宰细)}{少许}
\ingredient{浓母鸡汤汁}{三勺}
\ingredient{绍酒}{三钱}
\ingredient{食盐}{少许}
\ingredient{胡椒}{少许}

\preparation

豆腐削去周围的表皮用刀背捣烂,挤干水份。以刀背将鱼肉捶绒,连同豆粉、苏打、食
盐、火腿和蛋清等加入豆腐内和拌均匀。用调羹连续盛入约七成滚的油内炸呈鸭黄色时捞
起,滤干,倾入烧沸的肉汤里汆一道,以去其油味,捞起再倾入浓鸡汤汁内,加入绍酒用
武火(不旺不小的火)煮约五分钟即成。起锅时才放进胡椒盛入碗内。

\features

味浓厚鲜美,富营养,内含浆包细嫩化渣,状似米口袋,故名“口袋豆腐”。

\end{recipe}

% vim: filetype=tex noautoindent nojoinspaces
% vim: fileencoding=utf-8 formatoptions+=m
% vim: textwidth=78 tabstop=4 shiftwidth=4 softtabstop=4
