\begin{recipe}{炸豆芽饼}

\ingredients

\ingredient{生黄豆芽瓣}{五两}
\ingredient{肥膘}{四两}
\ingredient{熟瘦火腿}{一'两五}
\ingredient{鸡蛋}{三个}
\ingredient{盐}{一钱}
\ingredient{料酒}{二钱}
\ingredient{胡椒面}{二分}
\ingredient{花椒面}{五分}

\ingredient{生菜}{四两}
\ingredient{白糖}{二钱.}
\ingredient{醋}{三钱}
\ingredient{香油}{五钱}
\ingredient{干豆粉}{一两}
\ingredient{味精}{三分}
\ingredient{化猪油}{一斤耗二两}

\cooking

\step 先将豆芽的渣皮选净,淘洗干净;将猪肥膘切成一分五见方的小颗,熟火腿也照样的切好,一并装入碗内。将鸡蛋清三个和干豆粉拌匀成蛋清豆粉,倒入装好豆芽瓣、火腿、肉颗的碗内,放下味精、盐、料酒、胡椒面拌匀。将选好的生菜洗净,切成丝或撕成块,用清水漂起待用。

\step 将炒锅放于旺火上,倒下化油烧至六成热时,将拌好的豆芽瓣,做成一寸过心、三分厚的小圆饼,共做二十四个,边做边投入油锅炸至稍带黄色捞起。再将锅内的油烧至七、八成热时,将炸好的豆芽饼一并投入油锅内炸成金黄色,泌去余油,淋入香油,起锅摆入盘的中央。

\step 将撕好或切好的生菜拌上糖、醋、香油,镶于长盘的两边,走菜时再兑一个椒盐碟子即成。

\notes

此菜颜色金黄,脆、嫩、香、酥,色鲜味美,适宜酒肴。

\end{recipe}

% vim: filetype=tex noautoindent
% vim: fileencoding=utf-8
% vim: textwidth=78 tabstop=4 shiftwidth=4 softtabstop=4
