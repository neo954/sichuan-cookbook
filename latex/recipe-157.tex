\begin{recipe}{芙蓉虾仁}

\ingredients

\ingredient{鲜虾}{一斤}
\ingredient{熟鸡肉}{五钱}
\ingredient{盐}{三分}
\ingredient{鸡蛋清}{五个}
\ingredient{猪瘦肉}{二两}
\ingredient{味精}{二分}
\ingredient{熟火腿}{五钱}
\ingredient{清汤}{一斤}
\ingredient{胡椒面}{二分}
\ingredient{鲜笋}{五钱}
\ingredient{千豆粉}{三钱}
\ingredient{料酒}{二钱}

\cooking

\step 鸡蛋清(四个)加冷汤及盐少许,用筷子搅匀,盛入蒸盘内,上笼蒸成白芙蓉。火
腿、鲜笋、鸡肉均切成豌豆米大的小丁。痩肉砸成茸子备用。

\step 鲜虾淘洗后用手挤仁去壳,漂于清水碗中将杂质清洗干净;将虾仁晾干水气,加盐
少许,用蛋清和干豆粉将虾仁拌匀。

\step 炒锅盛清水半锅烧沸,用手将虾仁分几次撒下,滑散;炉火过大可将锅提离火口一
半,待虾仁浮于水面即熟,立即 用漏瓢打起,漂于冷水碗中,若有粘连的即用手捏散;
再淘 洗一次,去渣,滤干水份,用清汤喂一次待用。

\step 芙蓉蛋由笼上取出拨于碗中,将喂好的虾仁舀于蛋面 上,再放上鸡肉、鲜笋、火
腿等丁,清汤用肉茸扫一次加1 少许及胡椒面、味精灌入碗内上席。

\notes

色美汤鲜,清淡爽口,四季适宜。

\end{recipe}

% vim: filetype=tex noautoindent
% vim: fileencoding=utf-8 formatoptions+=m
% vim: textwidth=78 tabstop=4 shiftwidth=4 softtabstop=4
