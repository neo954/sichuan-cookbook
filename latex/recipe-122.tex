\begin{recipe}{红烧舌掌}

\ingredients

\ingredient{鸭舌}{二十个}
\ingredient{火腿}{六钱}
\ingredient{盐}{三分}
\ingredient{鸭脚掌}{二十对}
\ingredient{青菜心}{一斤}
\ingredient{水发口茉}{六钱}
\ingredient{姜片}{二钱}
\ingredient{葱白}{二铁}
\ingredient{水豆粉}{二钱}
\ingredient{味精}{一分五}
\ingredient{胡椒面}{一分}
\ingredient{酱油}{三钱}
\ingredient{料酒}{六钱}
\ingredient{化猪油}{二两五}
\ingredient{清汤}{一斤四两}
\ingredient{鸡油}{三钱}

\preparation

\step 取肥痩净火腿切成一寸四长、四分宽、半分厚的片。鸭舌和鸭脚掌用清水洗净,去
掉表面粗皮,放入锅中,用沸水在旺火上汆十五分钟(如系老鸭的脚掌,汆的时间要长
些),用手指甲掐一下,如皮骨将离即捞起;晾至不烫手財,将鸭掌皮剥开,去净硬骨,
鸭舌抽去软骨,一同放入大蒸碗内;掺入清汤,加盐、姜片、料酒和葱白与火腿片;上笼
蒸四十分钟左右即蒸透。

\step 青菜心用清水洗净,削去筋皮,切成长一寸四,宽、厚各四分的长方条。将锅置于
旺火上,放入猪油一两,等油烧热时,将青菜心放入煸四、五分钟,再掺入清汤,汤要淹
过菜心,焖二十分钟即𤆵,端至微火上焙起。

\step 另将锅放在旺火上烧热,放入猪油、葱白,煸至油冒烟时,即将鸭舌、掌从笼内取
出,去姜葱不用,将滋汁泌入油锅中,然后将锅中葱白捞去,加入酱油、味精、胡椒面和
口茉等,用汤瓢搅匀,随即将舌掌等倒入锅中一边,再将青菜心从另锅中捞出(余汤不
用)放入锅中另一边,同烧约四分钟。等汁约余四两左右时,用漏瓢将菜心捞起放在盘中
心,再将舌、掌捞出放在菜心上面,然后将水豆粉入锅勾芡,淋入鸡油起锅,淋在舌、掌
上面即成。

\features

此菜颜色金黄,味浓而鲜(副料蔬菜冬季宜用青菜心,春季宜用春笋,夏季宜用黄秧白菜
或玉笋,秋季宜用板栗。)

\end{recipe}

% vim: filetype=tex noautoindent nojoinspaces
% vim: fileencoding=utf-8 formatoptions+=m
% vim: textwidth=78 tabstop=4 shiftwidth=4 softtabstop=4
