% BSD 3-Clause License
%
% Copyright (c) 2023 Quux System and Technology. All rights reserved.
%
% Redistribution and use in source and binary forms, with or without
% modification, are permitted provided that the following conditions are met:
%
% 1. Redistributions of source code must retain the above copyright notice, this
%    list of conditions and the following disclaimer.
%
% 2. Redistributions in binary form must reproduce the above copyright notice,
%    this list of conditions and the following disclaimer in the documentation
%    and/or other materials provided with the distribution.
%
% 3. Neither the name of the copyright holder nor the names of its
%    contributors may be used to endorse or promote products derived from
%    this software without specific prior written permission.
%
% THIS SOFTWARE IS PROVIDED BY THE COPYRIGHT HOLDERS AND CONTRIBUTORS "AS IS"
% AND ANY EXPRESS OR IMPLIED WARRANTIES, INCLUDING, BUT NOT LIMITED TO, THE
% IMPLIED WARRANTIES OF MERCHANTABILITY AND FITNESS FOR A PARTICULAR PURPOSE ARE
% DISCLAIMED. IN NO EVENT SHALL THE COPYRIGHT HOLDER OR CONTRIBUTORS BE LIABLE
% FOR ANY DIRECT, INDIRECT, INCIDENTAL, SPECIAL, EXEMPLARY, OR CONSEQUENTIAL
% DAMAGES (INCLUDING, BUT NOT LIMITED TO, PROCUREMENT OF SUBSTITUTE GOODS OR
% SERVICES; LOSS OF USE, DATA, OR PROFITS; OR BUSINESS INTERRUPTION) HOWEVER
% CAUSED AND ON ANY THEORY OF LIABILITY, WHETHER IN CONTRACT, STRICT LIABILITY,
% OR TORT (INCLUDING NEGLIGENCE OR OTHERWISE) ARISING IN ANY WAY OUT OF THE USE
% OF THIS SOFTWARE, EVEN IF ADVISED OF THE POSSIBILITY OF SUCH DAMAGE.
%
\begin{recipe}{干煸鱿鱼笋丝}

\ingredients

\ingredient{干鱿鱼}{三两}
\ingredient{化猪油}{二两五}
\ingredient{盐、味精}{各三分}
\ingredient{肥瘦猪肉}{二两五}
\ingredient{香油}{二钱}
\ingredient{酱油}{二钱}
\ingredient{料酒}{三钱}
\ingredient{冬笋}{三两五}
\ingredient{水豆粉}{一钱}
\ingredient{泡辣椒}{二根}

\preparation

\step 冬笋除去壳、笋衣和老根,取用鲜嫩部分横切成细丝;猪肉切二寸长的细丝;泡红
辣椒去籽,顺切成细丝;选用大张鱿鱼,去头尾,横切成细丝,用温水淘洗两次,去净泥
沙(淘洗时不宜在水中久泡),挤干水待用。

\step 将料酒、盐、味精、水豆粉等放入小碗中搅匀,成为滋汁。

\step 将猪油入锅烧红,放入鱿鱼丝煸炒后,烹入料酒,再煸一下,即将肉丝放入,与鱿
鱼丝一同炒熟;随后放入冬笋辣椒丝炒匀,烹入滋汁稍煸四五下,再将香油淋于菜上炒
匀,即起锅入盘上席。

\features

此菜干香味美,为佐酒佳肴。另一做法不加水豆粉不烹滋汁。

\end{recipe}

% vim: filetype=tex noautoindent nojoinspaces
% vim: fileencoding=utf-8 formatoptions+=m
% vim: textwidth=78 tabstop=4 shiftwidth=4 softtabstop=4
