\begin{recipe}{酸辣虾羹汤}

\ingredients

\ingredient{黄牛肉}{二两五}
\ingredient{菜油}{一两}
\ingredient{胡椒面}{三分}
\ingredient{水发海带}[\footnotemark]{一两五}
\ingredient{辣椒油}{二钱}
\ingredient{豆瓣}{二钱五}
\ingredient{酥肉丁}[\footnotemark]{二两五}
\ingredient{水发茗笋}[\footnotemark]{二两五}
\ingredient{水豆粉}{一两五}
\ingredient{料酒}{二钱}
\ingredient{水发响皮}[\footnotemark]{一两五}
\ingredient{醋}{四钱}
\ingredient{酱油}{二钱五}
\ingredient{虾汤}[\footnotemark]{一两二}
\ingredient{葱、姜}{少许}
\ingredient{味精}{二分五}
\ingredient{盐}{六分}

\preparation

\step 牛肉选用腿上净瘦肉,切成三分见方的薄片。水发茗笋每根用手掰成两半,横切为
薄片。水发响皮切成二分半方丁。水发海带切成六分长、一分宽的丝。

\step 菜油入锅中烧红,放入牛肉片炒散,再放入豆瓣与肉炒匀起锅。

\step 将虾汤倒入锅中,再放入水发茗笋片、水发海带丝、水发响皮丁、炒熟的牛肉片,
以及料酒、盐、酱油、胡椒面和味精等烧开,随后将水豆粉搅匀,徐徐倒入锅中,搅匀烧
开,去净汤内的泡沫,再加入醋,搅匀后即起锅,盛入碗中。最后淋上油辣椒,撒上酥肉
丁,即时上席。

\features

此菜味道鲜美,酥肉丁非常酥香,并能醒酒。

\footnotetext[1]{
水发海带:干海带用清水泡一小时后再用清水洗净,即可使用。
}

\footnotetext[2]{
\begin{subrecipe}{酥肉丁}

\ingredients

\ingredient{鸡蛋(大的)}{一个}
\ingredient{猪肉}{一两半}
\ingredient{干豆粉}{三钱}
\ingredient{菜油}{一斤(耗三钱)}

\preparation

猪肉片成三分厚的片。鸡蛋去壳与豆粉一起搅成蛋糊,抹于猪肉上。菜油入锅烧开,放入
摸好蛋糊的猪肉,炸成蛋黄色时捞出,切成三分方丁,再入油中炸酥捞起即成。

\end{subrecipe}
}

\footnotetext[3]{
水发茗笋:选用四川茗山干笋尖,用淘米水淘洗干净,再泡二十四小时,捞出洗净,用开
水汆一、二次,即可使用。
}

\footnotetext[4]{
\begin{subrecipe}{水发响皮}

\ingredients

\ingredient{猪肉皮}{三两}
\ingredient{菜油}{一斤(耗三钱)}

\preparation

选猪腿皮晾干。将菜油入锅上炉烧开,放入肉皮炸泡,捞出入清水中泡一小时便成。

\end{subrecipe}
}

\footnotetext[5]{
\begin{subrecipe}{虾汤}

\ingredients

\ingredient{鲜虾}{二两五}
\ingredient{清水}{一斤四两}

\preparation

锅中放入清水,再放入鲜虾,上炉用旺火煮五分钟。即制得虾汤一斤三两。

\end{subrecipe}
}

\end{recipe}

% vim: filetype=tex noautoindent nojoinspaces
% vim: fileencoding=utf-8 formatoptions+=m
% vim: textwidth=78 tabstop=4 shiftwidth=4 softtabstop=4
