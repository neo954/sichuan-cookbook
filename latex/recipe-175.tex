\begin{recipe}{鱿鱼腐皮}

\ingredients

\ingredient{豆腐皮}{三张}
\ingredient{水发鱿鱼}{六两}
\ingredient{熟鸡皮}{一两}
\ingredient{熟火腿}{一两}
\ingredient{豌豆尖}{约十朵}
\ingredient{化鸡油}{五钱}
\ingredient{草碱水}{少许}
\ingredient{水豆粉}{五钱}
\ingredient{盐}{四分}
\ingredient{味精}{二分}
\ingredient{胡椒面}{一分}
\ingredient{清汤}{八两}
\ingredient{化猪油}{一两}
\ingredient{姜、葱}{各三钱}

\preparation

\step 先将豆腐皮切成六分宽的旗子块,用清水漂起,并滴入碱水于清水内,再放入锅内
沮一下,到豆腐皮软和时即捞起,用清水漂起待用;选用大张明嫩的鱿鱼,同样改成六分
宽的旗子块,用开水洗去碱味,用清水漂起;鸡皮去尽细毛,同样切成六分宽的旗子块;
火腿切成七分宽、一寸二长的骨睥片子。

\step 将锅放在旺火上,倒入化猪油,烧至五成热时,放下姜、葱,爆出香味,再倒入清
汤烧开,打去姜、葱不用;再将豆腐皮泌去清水用二汤过一次(去掉碱味),再倒入锅内
;鱿鱼、鸡皮、火腿同盐、胡椒、味精,依次投入锅内,烧开时再勾入水豆粉成清二流芡
,放入豌豆尖I起锅时淋化鸡油盛入大窝盘内即成。

\features

此菜颜色美观,鲜嫩可口,适宜老年。

\end{recipe}

% vim: filetype=tex noautoindent nojoinspaces
% vim: fileencoding=utf-8 formatoptions+=m
% vim: textwidth=78 tabstop=4 shiftwidth=4 softtabstop=4
