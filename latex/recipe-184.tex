\begin{recipe}{清汤鱼肚卷}

\ingredients

\ingredient{黄鱼肚}{二两}
\ingredient{鱼肉}{三两}
\ingredient{肥膘}{二两}
\ingredient{鸡蛋}{四个}
\ingredient{干豆粉}{六钱}
\ingredient{猪瘦肉}{三两}
\ingredient{胡椒面}{二分}
\ingredient{盐}{八分}
\ingredient{料酒}{五钱}
\ingredient{味精}{三分}
\ingredient{特级清汤}{一斤半}
\ingredient{菠菜杆}{二两}
\ingredient{火腿}{一两}


\cooking

\step 先将炸过的鱼肚用温水泡胀,除尽油腻;改成二寸长、一寸宽、一分五厚的片子(剩余的边角另作他用),放在盘内,用二汤喂起;再将菠菜杆沮后晾凉(大杆撕破),切成一寸二长的二粗丝;火腿也照样切好;又将肥膘、鱼肉及蛋清豆粉打成“糁”;将猪痩肉捶茸5用少量清水解散,分幵放入碗内待用。

\step 其余的蛋清及豆粉拌成蛋清豆粉,将喂好的鱼肚用手挤干水份平铺于案上,抹上蛋清豆粉;将打好的糁,用筷子或竹片均匀地刮在二十四片鱼肚上;再将切好的菠菜杆、火腿丝各一丝横放于糁上;然后把鱼肚折转裹成圆筒,交口处抹上蛋清豆粉。二十四个照样裹完,放入抹上一层油的盘内(交口处压扁),上笼用大火蒸五分钟即熟,成鱼肚卷后取出。

\step 锅放在旺火上,倒入特级清汤,加料酒、胡椒、盐,烧至汤开,倒入茸子;肉末浮起,用瓢子打尽浮沫及杂质;把鱼

肚卷用清汤过一次泌去,滑入碗内;再将锅内的汤放下味 精,倒入碗内即成。

\notes

汤清菜美,鲜嫩可口,四季适宜。

\end{recipe}

% vim: filetype=tex noautoindent
% vim: fileencoding=utf-8
% vim: textwidth=78 tabstop=4 shiftwidth=4 softtabstop=4
