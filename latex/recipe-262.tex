\begin{recipe}{生片菊花锅}

\ingredients

\ingredient{鄉鱼}{八两}
\ingredient{胡椒}{三分}
\ingredient{粉条}{二钱}
\ingredient{油炸花生米}{一两五}
\ingredient{生鸡脯肉}{三两}
\ingredient{盐}{三分}
\ingredient{豌豆尖}{三两}
\ingredient{油条}{三根}
\ingredient{奶汤}{二斤半}
\ingredient{香菜}{一两}

\ingredient{干粉}{五钱}
\ingredient{菜油}{一'斤耗三两}
\ingredient{葱}{二钱}
\ingredient{白菜心}{三两}
\ingredient{鸡捃肝}{三个}
\ingredient{味精}{三分}
\ingredient{菠菜}{五两}
\ingredient{猪腰}{四两}
\ingredient{化猪油}{一两}
\ingredient{姜}{二钱}
\ingredient{粗撒子}{二把}
\ingredient{白菊花}{一朵}

\cooking

\step 鲫鱼洗净,去鱗,去肠杂,去刺(刺留用),用刀切成 二寸长、一分厚、宽同鱼身的薄片。鸡腊肝洗净,去表皮, 切成薄片。生鸡脯肉片为一寸长、八分宽的薄片。猪腰去筋 和腰臊,切成鸡脯大小的薄片(以上薄片,愈薄愈好)。这 四种片称为“四生片”,分别摆于七寸盘中成风车形。

\step 菜油一斤入锅烧红,投入粉条,炸成黄色;花生米去

皮;油条切成一寸长段;撒子捏散(临上席时再用油炸一次 以保持酥脆);分别放入七寸盘内,称为“四油酥”。

白菜心洗净,用手撕去筋;豌豆尖淘洗,只留嫩苞; 菠菜去老叶和茎根部分;香菜洗净;分别放于七寸盘中,称 为“四鲜菜”。

\step 姜一钱去皮,切成细末;葱切成葱花;与胡椒、味精 共摆一碟,每样占碟的四分之一。另以小碟盛盐备用。

\step 鲜菊花一大朵(最好为自色),用刀切去花蒂,抽出 花蕊,仍以菊花原形放入盘中。

\step 锅置火上,放入猪油,加入姜(拍松)、葱切成五分 长的短节,随即放入奶汤和鱼骨刺,烧沸,约五分钟捞去鱼 骨刺和姜、葱,把汤舀入铜锅内为鱼羹汤。

\step 桌上放一粗磁深盘,盘内盛酒精,盘上加铜架,架上 放铜锅,吃时将酒精燃烧,锅内汤沸时先将菊花放下,次将 各类生肉、生菜分别于汤内烫熟吃(除生片、生菜外其余油 蘇忌下锅内)。

\notes

此菜宜于秋季吃,生片、鲜菜汤味清鲜,佐酒下饭均 宜。菜内生片、鲜菜可按季节变动,但只能用四样。冬季改 用梅花,即名“梅花锅”。

\end{recipe}

% vim: filetype=tex noautoindent
% vim: fileencoding=utf-8
% vim: textwidth=78 tabstop=4 shiftwidth=4 softtabstop=4
