\begin{recipe}[麒麟鱼]{清汤五彩鱼}

\ingredients

\ingredient{效鱼一尾}{一斤半}
\ingredient{鸡蛋}{一个}
\ingredient{木耳}{五钱}
\ingredient{熟瘦火腿}{一两五}
\ingredient{慈菇}{三两}
\ingredient{丝瓜皮}{二两}
\ingredient{特级清泳}{一斤半}
\ingredient{料酒}{三钱}
\ingredient{胡椒面}{二分}
\ingredient{味精}{三分}
\ingredient{鸡脯肉}{四两}
\ingredient{姜}{一两}
\ingredient{醋}{一两}
\ingredient{香油}{二分}
\ingredient{葱}{一两}
\ingredient{花椒}{约十粒}
\ingredient{盐}{五分}

\preparation

\step 先将鲤鱼去鳞挖腮,剖腹去内脏,清洗干净,放在墩子上,在鱼两面用剽刀共剽十
刀,距离相等,用盐、料酒混抹,并使之浸入鱼的刀口处,放入盘中待用。

\step 将鸡蛋摊成蛋皮;木耳、火腿、慈菇、丝瓜、蛋皮分别切成八分长的半圆形(为五
色、再用干净布将抹鱼的汁沾干,然后将切好的五色片插于刀口处,在每一个开刀处插入
各色一片,共五片;插好后将鱼腹向下放于大盘内,再放入姜片、葱节、花椒、盐、胡椒
、料酒,入笼内大火蒸十五分钟馏起。

将鸡脯肉洗净,放在墩子上棰茸;再将锅放在旺火上,倒入特级清汤,用少许清水将鸡茸
解散,投入汤内,微转数下,汤至微开,肉沫浮起,打尽杂质及肉沬;再取出笼内的鱼
,去掉姜、葱、花椒,将锅内的清汤灌入盛鱼的盘中,走菜时随上一个姜醋碟子即成。

\features

色分五彩,汤味清香,肉嫩可口,秋夏适宜。

\end{recipe}

% vim: filetype=tex noautoindent nojoinspaces
% vim: fileencoding=utf-8 formatoptions+=m
% vim: textwidth=78 tabstop=4 shiftwidth=4 softtabstop=4
