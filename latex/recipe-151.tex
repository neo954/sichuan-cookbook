\begin{recipe}[麒麟鱼]{清汤五彩鱼}

\ingredients

\ingredient{效鱼一尾}{一斤半}
\ingredient{鸡蛋}{一个}
\ingredient{木耳}{五钱}
\ingredient{熟瘦火腿}{一两五}
\ingredient{慈菇}{三两}
\ingredient{丝瓜皮}{二两}
\ingredient{特级清泳}{—斤半}
\ingredient{料酒}{三钱}
\ingredient{胡椒面}{二分}
\ingredient{味精}{三分}
\ingredient{鸡脯肉}{四两}
\ingredient{姜}{一两}
\ingredient{醋}{一-两}
\ingredient{香油}{二分}
\ingredient{葱}{一两}
\ingredient{花概}{约十粒}
\ingredient{盐}{五分}

\cooking

\step 先将鲤鱼去鳞挖腮,剖腹去内脏,清洗干净,放在墩 子上,在鱼两面用剽刀共剽十刀,距离相等,用盐、料酒混 抹,并使之浸入鱼的刀口处,放入盘中待用。

\step 将鸡蛋摊成蛋皮;木耳、火腿、慈菇、丝瓜、蛋皮分别切 成八分长的半圆形(为五色、再用干净布将抹鱼的汁沾干, 然后将切好的五色片插于刀口处,在每一个开刀处插入各色 一片,共五片;插好后将鱼腹向下放于大盘内,再放入姜片、 葱节、花椒、盐、胡椒、料酒,入笼内大火蒸十五分钟馏起。

将鸡脯肉洗净,放在墩子上棰茸;再将锅放在旺火上, 倒入特级清汤,用少许清水将鸡茸解散,投入汤内,微转数 下,汤至微开,肉沫浮起,打尽杂质及肉沬;再取出笼内的 鱼,去掉姜、葱、花椒,将锅内的清汤灌入盛鱼的盘中,走 菜时随上一个姜醋碟子即成。

\notes

色分五彩,汤味清香,肉嫩可口,秋夏适宜。

\end{recipe}

% vim: filetype=tex noautoindent
% vim: fileencoding=utf-8
% vim: textwidth=78 tabstop=4 shiftwidth=4 softtabstop=4
