\begin{recipe}{家常海参}

\ingredients

\ingredient{肥瘦肉}{二两五}
\ingredient{水发海参}{一斤}
\ingredient{豆辨}{一两}
\ingredient{咸红酱油}{三两}
\ingredient{盐}{五分}
\ingredient{鉗酱}{一分}
\ingredient{味精}{三分}
\ingredient{建兰菜}{半斤}
\ingredient{清汤}{一斤}
\ingredient{料酒}{五钱}
\ingredient{水豆粉}{三钱}
\ingredient{杯田}{一两}
\ingredient{香油}{三钱}
\ingredient{化猪油}{三两}

\preparation

\step 先将海参淘洗干净,片成斧轮片,放入盘内待用;再将肥瘦肉洗净,放在墩子上剁
细;豆瓣剁细;蒜苗切成大粗花形;建兰菜选心,淘洗干净待用。

\step 将锅放在旺火上,放下清汤、盐、料酒及海参,喂一道;烧开时,捞起海参,锅内
的汤不用、再放入清汤,放下捞起的海参,再喂一道捞起放在盘内。

\step 锅内的汤倒去,放在炉上,放下化猪油,烧至四成火倒入烂肉,放下料酒、盐,熵
散后,扞入碗内,再放下化猪油和建兰菜,炒熟,加少许盐、料酒,起锅装于走菜的盘内
。

放入余下的化油入锅内,放入豆瓣熵出红色,放下甜 酱、清汤,打尽豆瓣渣,将海参及
烂肉一并倒入锅内,加料 酒、酱油、味精、蒜苗花,烧至亮油,加水豆粉,起锅时淋 入
香油,扦入建兰菜的面上即成。

\features

颜色鲜红,味浓可口。

\end{recipe}

% vim: filetype=tex noautoindent nojoinspaces
% vim: fileencoding=utf-8 formatoptions+=m
% vim: textwidth=78 tabstop=4 shiftwidth=4 softtabstop=4
