\begin{recipe}{荷叶蒸肉}

\ingredients

\ingredient{猪肥肉}{五两}
\ingredient{猪瘦肉}{五两}
\ingredient{鲜荷叶}{四张}
\ingredient{青黄豆}{二两五}
\ingredient{酱油}{三钱}
\ingredient{咸酱油}{二钱五}
\ingredient{甜酱}{三钱}
\ingredient{耢糟浮子}{三钱}
\ingredient{豆腐乳水}{一钱五}
\ingredient{红糖}{三钱}
\ingredient{葱姜末}{一钱五}
\ingredient{大米}{二两五}
\ingredient{花椒}{约十粒}
\ingredient{大料面}{一钱}
\ingredient{料酒}{六钱}

\cooking

\step 	将大米、花椒、大料面等一并放入锅中,在微火上炒 成黄色,再用小石磨磨成米粉。

\step 	取连皮的猪肥肉和净瘦肉各半斤,刮洗干净,拈净残 毛。肥肉切成一寸半长、一寸宽、一分二厚的片;痩肉切成 厚一分,长、宽与肥肉相等的片,各切二十片。另将红、白 酱油、料酒、甜酱、耢糟、红糖、豆腐乳水、葱、姜末等一并 放入盆内拌匀,将肥瘦肉片放入,调拌后醃十余分钟,使之入 味;再将米粉放入拌匀〈如太干可加少许清水〉,使米粉沾 在肉片上。青豆(即嫩黄豆,嫩时为青色〉用清水洗净。

\step 	将鲜荷叶用清水洗净,入开水里烫一下,使它柔软不 易折断;然后切成四寸长的等边三角形共二十张,取肥肉片 一片放在叶内,再放入青豆五、六粒,上面摆上瘦肉一片包 好,一共包成二十个,摆入盘内上笼蒸三小时即熟,取出即 可食用。

\notes

此菜颜色金黄,味咸甜,宜于热吃,鲜美清香,别具风 味。用鲜荷叶包裹,是取其清香味,故宜于荷花盛开时节烹 制。吃时由食用者自己剥去荷叶。

\end{recipe}

% vim: filetype=tex noautoindent
% vim: fileencoding=utf-8
% vim: textwidth=78 tabstop=4 shiftwidth=4 softtabstop=4
