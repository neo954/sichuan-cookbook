\begin{recipe}[八宝糯米鸡]{糯米鸡}

\ingredients

\ingredient{仔母鸡一只}{约三斤}
\ingredient{鲜琬豆仁}{二两五}
\ingredient{糯米}{二两}
\ingredient{熟火腿}{六钱}
\ingredient{金钩}{三钱}
\ingredient{水发香菌}{六钱}
\ingredient{苡仁}{二钱}
\ingredient{莲米}{三钱}
\ingredient{鸡蛋清}{一个}
\ingredient{香油}{三铁}
\ingredient{盐}{四分}
\ingredient{酱油}{六钱}
\ingredient{椒盐面}[\footnotemark]{二钱}
\ingredient{干豆粉}{五钱}
\ingredient{芡实}{二钱}
\ingredient{菜油}{二斤约耗二两}
\ingredient{胡椒}{一分}

\preparation

\step 仔母鸡一只宰杀后放净血,去掉毛、足,从鸡背颈部上面,顺割约三寸长的口,用
手挖去鸡縢,仰放在菜墩上,用刀从杀口处将鸡颈骨割断,再从鸡背颈刀口处将颈骨拉出
。然后将鸡立放在菜墩上,用手将鸡皮肉略为用力向夕卜翻,并向下褪,鸡骨架就逐渐露
出。鸡翅膀与胸腔骨相连的筋用刀割断,随顺着鸡的肉骨周围相连地方慢慢往下剐,边褪
边剐,直剐到鸡的尾部,把鸡的骨架内脏取出。再用刀尖拨开鸡的两腿棒子骨,使棒子骨
露出一点,两手用力将鸡腿肉往下按,棒子骨就全部露出,用刀剔下。此时鸡除头部、两
翅外,骨头都已剔下(剔剐过程中,注意保持肉皮完整无破损)。再将鸡皮面翻出来,在
清水中洗干净,滤干水份。

\step 鲜豌豆仁用沸水煮熟去壳,漂在清水中,保持绿色。糯米淘洗后在沸水锅内煮透,
以尚未成饭、米无硬心为度。莲米(退衣去心)、苡仁和芡实淘洗后用开水泡胀,装在硫
内,加清水泡浸,上笼蒸𤆵。金钩用沸汤泡胀,与香菌、火腿切成如豌豆大小的丁,然后
将鲜豌豆仁、糯米、莲米、苡仁、芡实、金钩、香菌和火腿加盐拌匀,从鸡颈开口处装入
腹内,装好后用削细的竹签(约五寸长)象缝衣似的将鸡颈开口处锁住(从外面看如同一
只有骨的鸡),放入沸汤锅中烫一下捞出,将鸡翅翻在背上盘起,鸡头仰翻在翅的下面压
住,再装入大蒸碗内,鸡背向上,不加水上笼干蒸;蒸约

两小时,用竹筷以能将鸡翅戳破为适合,然后把它翻入盘内,用干布帕将鸡身的水气沾
干,全身抹上酱油;再将鸡蛋加干豆粉搅打成蛋清豆粉抹上。

\step 大炒锅内放入菜油,在旺火上烧至八成热时,即将鸡顺着锅边放入,腹部向上(向
下容易炸糊)炸五分钟;炸时用汤瓢将鸡不断拨动,并圈沸油往鸡腹上不断浇淋;待鸡身
炸成金黄色时泌去炸油,淋入香油起锅。随后用刀将腹部轻轻划成象眼块,以将皮划断为
度,不可深及肉内。上盘时抽、掉颈部竹签,随同椒盐二碟上席。

\features

此鸡颜色金黄光亮,大方美观,皮酥肉嫩,酿馅鲜香,风味特佳。

\footnotetext{
椒盐是盐和花椒面拌合而成,配比是一比二。
}

\end{recipe}

% vim: filetype=tex noautoindent nojoinspaces
% vim: fileencoding=utf-8 formatoptions+=m
% vim: textwidth=78 tabstop=4 shiftwidth=4 softtabstop=4
