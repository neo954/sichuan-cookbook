\begin{recipe}{杏元土司}

\ingredients

\ingredient{土司}{一玲}
\ingredient{鸡蛋}{四个}
\ingredient{净鲜鱼肉}{四两}
\ingredient{猪肥瞧肉}{二两}
\ingredient{干豆粉}{五钱}
\ingredient{化猪油}{二斤耗二两}
\ingredient{生菜}{三两}
\ingredient{香油}{五钱}
\ingredient{白糖}{三钱}
\ingredient{酷}{三钱}
\ingredient{盐}{四分}
\ingredient{味精}{三分}
\ingredient{料酒}{二钱}

\cooking

\step 土司切成一分半厚的大片,再将大片改成八分见方的 小片,共切三十片,修去四角成八分过心的圆形,放入盘内 待用。

\step 将鲜鱼肉、猪肥膘分别棰茸,加鸡蛋清、盐、水豆粉 打成“鱼糁”放在碗内待用。

\step 每一片土司上均匀地放上二分厚的鱼糁,用手刮边沿 成中心高边沿低的形式,一直做完三十片。

\step 将锅置于旺火,倒入化猪油,烧至四成火,再一个个 地放下土司〔糁向上,土司向下),由低温到高温,土司的 颜色接近黄色(形色同杏元相似泌去锅内炸油,淋入香 油.起锅入条盘的一端。再将生菜淘洗干净3滤干水份,拌 入糖、醋、香油、味精,放在杏元土司的另一端即成。

\notes

酥、香、脆、嫩,在席桌中宜于行菜。

\end{recipe}

% vim: filetype=tex noautoindent
% vim: fileencoding=utf-8
% vim: textwidth=78 tabstop=4 shiftwidth=4 softtabstop=4
