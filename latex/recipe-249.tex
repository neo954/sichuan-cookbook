\begin{recipe}{酿鸽蛋}

\ingredients

\ingredient{鸽蛋}{二十个}
\ingredient{青豆}{数颗}
\ingredient{鸡脯肉}{一两}
\ingredient{肥瞟}{一两}
\ingredient{瘦火腿}{三钱}
\ingredient{胡椒}{一分}
\ingredient{鸡貪有}{料酒}
\ingredient{三钱}{}
\ingredient{金钩}{二钱}
\ingredient{干豆粉(碾细)}{一两三}
\ingredient{盐}{一分}
\ingredient{味精}{三分}
\ingredient{化猪油}{一斤耗一两五}
\ingredient{香油}{一钱}
\ingredient{椒盐碟}{适量}

\cooking

\step 鸡脯、肥膘肉打成“鸡糁”;金钩、火腿、青豆分别用刀铡成细末,加入味精、料酒
、胡椒、盐等作料拌匀成糁馅子待用。

用白酒杯二十个,洗净,先将内面抹少许油,鸽蛋二 十个逐一打破,每杯一个,上笼气
一下,待鸽蛋白刚迸皮 (不使蛋黄气热)立即取出。将糁馅子分成二十份,用竹筷 将每
份馅子酿入酒杯的鸽蛋内,蛋黄溢出自然封口。再上笼 气二分钟,至鸽蛋全熟取出,裹
上碾细的干豆粉。

\step 猪油在旺火上烧至五成火后,即将裹好豆粉的鸽蛋下入,边炸边翻动;炸至金黄色
,泌去油,淋入香油,簸匀起锅,上盘时镶入生菜、椒盐即成。

\features

颜色金黄,外酥内嫩,吃时用椒盐,鲜香可口。

\end{recipe}

% vim: filetype=tex noautoindent
% vim: fileencoding=utf-8 formatoptions+=m
% vim: textwidth=78 tabstop=4 shiftwidth=4 softtabstop=4
