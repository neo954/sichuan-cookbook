\begin{recipe}[如意蛋饺]{白汁蛋饺}

\ingredients

\ingredient{鸡蛋}{四个}
\ingredient{猪肥瘦肉}{六两}
\ingredient{大慈菇}{十个}
\ingredient{金钩}{五钱}
\ingredient{白菜秧心子}{六两}
\ingredient{味精}{二分}
\ingredient{胡椒面}{一令}
\ingredient{生鸡油}{一两}
\ingredient{盐}{三分}
\ingredient{水豆粉}{三钱}
\ingredient{料酒}{五钱}
\ingredient{猪油}{一1两}
\ingredient{清汤}{四两}
\ingredient{猪肥瞟}{一两}

\ingredient{猪网油(见方一尺)}{三两}

\cooking

\step 猪肥瘦肉洗净,宰成肉茸;慈菇洗净,削皮,在沸水内冒一下,用刀切成碎米颗;金钩用温热水淘洗干净,在沸:水内泡半小时,捞起切成碎米颗;鸡蛋打破在碗内调散,以一半在锅内摊成一张大蛋皮,其余一半留碗内待用。

\step 肉茸合慈菇、金钩颗装在大碗内,加剩余调散的鸡蛋三分之一,同时加进豆粉(二钱)、料酒(二钱)、盐(一

分:)、味精少许、胡椒面少许拌匀;把蛋皮铺在墩子上,将 拌好的肉茸扞一半均匀地抹在蛋皮上,两手从左右两边向中 心卷成“如意”形;上笼蒸约两分钟,取出晾冷后,切成二 分厚的片子待用。

\step 取汤瓢一个在火上烤热,用肥膘炙过,拿小调羹一个,舀起一调羹调散的蛋,倒入汤瓢内,均匀地摊成小蛋皮。另用小调羹一个,将按蛋皮大小固入拌好味的肉茸,放在蛋皮一边,再将另一边的蛋皮翻过来粘上成饺子形;如是作法将所有鸡蛋和肉茸做成十二个蛋饺,上笼蒸一分多钟取出待用04,白菜秧心子淘洗干净,用化猪油(五钱〕煸炒扞在盘内晾起;同时将生鸡油盛入小碗上笼蒸化。用蒸碗一个铺上网油,再将切好的“如意卷”及蛋饺对镶一层,剩下材料全部放在碗底,清汤倒入,再拈入白菜秧,淋鸡油五钱,上笼蒸一分钟取出,翻入白瓷条盘,再将清汤(二两)合其余味精、料酒(三钱)、盐(二分)、水豆粉(一钱),勾二流芡,下鸡油入滋汁淋上即成。

\notes

色泽美观,味鲜可口。

\end{recipe}

% vim: filetype=tex noautoindent
% vim: fileencoding=utf-8
% vim: textwidth=78 tabstop=4 shiftwidth=4 softtabstop=4
