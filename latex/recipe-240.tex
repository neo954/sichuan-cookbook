% BSD 3-Clause License
%
% Copyright (c) 2023 Quux System and Technology. All rights reserved.
%
% Redistribution and use in source and binary forms, with or without
% modification, are permitted provided that the following conditions are met:
%
% 1. Redistributions of source code must retain the above copyright notice, this
%    list of conditions and the following disclaimer.
%
% 2. Redistributions in binary form must reproduce the above copyright notice,
%    this list of conditions and the following disclaimer in the documentation
%    and/or other materials provided with the distribution.
%
% 3. Neither the name of the copyright holder nor the names of its
%    contributors may be used to endorse or promote products derived from
%    this software without specific prior written permission.
%
% THIS SOFTWARE IS PROVIDED BY THE COPYRIGHT HOLDERS AND CONTRIBUTORS "AS IS"
% AND ANY EXPRESS OR IMPLIED WARRANTIES, INCLUDING, BUT NOT LIMITED TO, THE
% IMPLIED WARRANTIES OF MERCHANTABILITY AND FITNESS FOR A PARTICULAR PURPOSE ARE
% DISCLAIMED. IN NO EVENT SHALL THE COPYRIGHT HOLDER OR CONTRIBUTORS BE LIABLE
% FOR ANY DIRECT, INDIRECT, INCIDENTAL, SPECIAL, EXEMPLARY, OR CONSEQUENTIAL
% DAMAGES (INCLUDING, BUT NOT LIMITED TO, PROCUREMENT OF SUBSTITUTE GOODS OR
% SERVICES; LOSS OF USE, DATA, OR PROFITS; OR BUSINESS INTERRUPTION) HOWEVER
% CAUSED AND ON ANY THEORY OF LIABILITY, WHETHER IN CONTRACT, STRICT LIABILITY,
% OR TORT (INCLUDING NEGLIGENCE OR OTHERWISE) ARISING IN ANY WAY OUT OF THE USE
% OF THIS SOFTWARE, EVEN IF ADVISED OF THE POSSIBILITY OF SUCH DAMAGE.
%
\begin{recipe}[如意蛋饺]{白汁蛋饺}

\ingredients

\ingredient{鸡蛋}{四个}
\ingredient{猪肥瘦肉}{六两}
\ingredient{大慈姑}{十个}
\ingredient{金钩}{五钱}
\ingredient{白菜秧心子}{六两}
\ingredient{味精}{二分}
\ingredient{胡椒面}{一分}
\ingredient{生鸡油}{一两}
\ingredient{盐}{三分}
\ingredient{水豆粉}{三钱}
\ingredient{料酒}{五钱}
\ingredient{猪油}{一两}
\ingredient{清汤}{四两}
\ingredient{猪肥膘}{一两}
\ingredient{猪网油(见方一尺)}{三两}

\preparation

\step 猪肥瘦肉洗净,宰成肉茸;慈姑洗净,削皮,在沸水内冒一下,用刀切成碎米颗;
金钩用温热水淘洗干净,在沸水内泡半小时,捞起切成碎米颗;鸡蛋打破在碗内调散,以
一半在锅内摊成一张大蛋皮,其余一半留碗内待用。

\step 肉茸合慈姑、金钩颗装在大碗内,加剩余调散的鸡蛋三分之一,同时加进豆粉(二
钱)、料酒(二钱)、盐(一分)、味精少许、胡椒面少许拌匀;把蛋皮铺在墩子上,将
拌好的肉茸搛一半均匀地抹在蛋皮上,两手从左右两边向中心卷成“如意”形;上笼蒸约
两分钟,取出晾冷后,切成二分厚的片子待用。

\step 取汤瓢一个在火上烤热,用肥膘炙过,拿小调羹一个,舀起一调羹调散的蛋,倒入
汤瓢内,均匀地摊成小蛋皮。另用小调羹一个,将按蛋皮大小舀入拌好味的肉茸,放在蛋
皮一边,再将另一边的蛋皮翻过来粘上成饺子形;如是作法将所有鸡蛋和肉茸做成十二个
蛋饺,上笼蒸一分多钟取出待用。

\step 白菜秧心子淘洗干净,用化猪油(五钱)煸炒搛在盘内晾起;同时将生鸡油盛入小
碗上笼蒸化。用蒸碗一个铺上网油,再将切好的“如意卷”及蛋饺对镶一层,剩下材料全部
放在碗底,清汤倒入,再拈入白菜秧,淋鸡油五钱,上笼蒸一分钟取出,翻入白瓷条盘,
再将清汤(二两)合其余味精、料酒(三钱)、盐(二分)、水豆粉(一钱),勾二流
芡,下鸡油入滋汁淋上即成。

\features

色泽美观,味鲜可口。

\end{recipe}

% vim: filetype=tex noautoindent nojoinspaces
% vim: fileencoding=utf-8 formatoptions+=m
% vim: textwidth=78 tabstop=4 shiftwidth=4 softtabstop=4
