\begin{recipe}{子母烩}

\ingredients

\ingredient{鸽子三只}{约一斤半}
\ingredient{鸽蛋}{十二个}
\ingredient{姜}{二钱}
\ingredient{葱白}{三钱}
\ingredient{味精}{二分}
\ingredient{盐}{一钱五}
\ingredient{花椒}{约十粒}
\ingredient{酱油}{三钱}
\ingredient{胡椒面}{一分五}
\ingredient{料酒}{三钱}
\ingredient{清汤}{二两五}
\ingredient{千豆粉}{六钱}
\ingredient{水豆粉}{三钱}
\ingredient{化猪油}{一斤约耗二两}

\cooking

\step 鸽子三只,宰杀后去血,退毛,从背脊上顺着割开,挖去五脏,去掉足爪;清洗干净后,用手将盐、酱油、料酒等在鸽的全身内外抹匀,然后将鸽的两翅翻向鸽背盘起。

在炒锅内放入猪油置于旺火上烧至七、八成热,将鸽子 炸五分钟捞起,用蒸碗装好,加入姜、葱白、花椒、清汤等 佐料;用一种有拉力的皮筋纸蒙紧碗口,在笼上蒸约两小时 半(根据鸽的老嫩增减蒸的时间,以骨松翅裂为度)。

\step 将鸽蛋连壳在笼上蒸熟,约一刻钟取出,用清水稍浸,即逐个剥去外壳,放在干豆粉上滚动,使鸽蛋上都裹上一层豆粉。再将炸鸽子的原油倒入炒锅中,放在旺火上烧至五成热,把鸽蛋顺锅边投入,用汤瓢慢慢拨动,炸约五分钟,至鸽蛋觅黄色即捞起待用。

\step 鸽子蒸好后去掉碗口的皮筋纸及姜、葱、花椒不用,取出鸽子摆在盘内,下面两只,上面一只,再将炸好的鸽蛋镶在鸽子的周围。随后把蒸鸽子的原汁汤泌在炒锅里,加入胡椒面、味精,开锅后用水豆粉勾芡,淋于鸽子及鸽蛋上”

成。

\notes

鸽肉鲜香.鸽蛋软嫩,清淡而富滋补。最宜秋冬季节食 用。因鸽子与鸽蛋同配一菜,故名子母会。(烩)

\end{recipe}

% vim: filetype=tex noautoindent
% vim: fileencoding=utf-8 formatoptions+=m
% vim: textwidth=78 tabstop=4 shiftwidth=4 softtabstop=4
