\begin{recipe}{鸡豆花①}

\ingredients

\ingredient{老白鸡脯肉}{二两五}
\ingredient{干豆粉}{二钱五}
\ingredient{特级清汤}{二斤}
\ingredient{鸡蛋清}{四个}
\ingredient{味精}{三分}
\ingredient{鲜菜心}{一两}
\ingredient{盐}{六分}
\ingredient{火腿}{一钱}
\ingredient{胡椒}{一分}

\cooking

\step 鸡脯肉要选用老白鸡的(此菜所用的鸡脯肉,必须选;用老鸡的,经过搅打,煮熟
后才能凝成如豆花形状。用嫩鸡肉不易凝聚,煮熟时成为“鸡淖”。其次要选用白皮鸡,才
能保证颜色雪白。雄鸡脯肉质粗不细,须捶成茸,但凝聚时仍有微细颗粒〕,去筋后用刀
背捶成茸,再用刀口剁数遍,剁后再捶,盛入碗内。鸡蛋清与豆粉混合后调匀。碗内的鸡
茸先用清水以竹筷搅散,再逐次加入蛋清豆粉、盐(用刀压为细末〕、味精、冷特级清汤
,每加一种佐料搅匀一次,分次加入,分次搅匀,最后搅为鸡茸糊。

\step 锅内揩净,加特级清汤。烧开时放味精、盐。随后将碗内鸡肉糊以竹筷搅匀入锅,
烧至微沸时把锅移于微火上烧十分钟,鸡茸糊凝聚在一块时即成豆花状。将鲜菜心入锅汆
过,用清水漂透心,用刀修齐两端,放于碗底,将鸡豆花舀于上面,再把火腿切成细末撒
在豆花上即成。

\features

此菜形如豆花,鲜嫩、清爽,宜于夏季佐餐。

①在四川所谓豆花就是豆腐脑,再经火煮比豆腐脑要老一些,用筷子就

可以挑起来.

\end{recipe}

% vim: filetype=tex noautoindent
% vim: fileencoding=utf-8 formatoptions+=m
% vim: textwidth=78 tabstop=4 shiftwidth=4 softtabstop=4
