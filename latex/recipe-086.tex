% BSD 3-Clause License
%
% Copyright (c) 2023 Quux System and Technology. All rights reserved.
%
% Redistribution and use in source and binary forms, with or without
% modification, are permitted provided that the following conditions are met:
%
% 1. Redistributions of source code must retain the above copyright notice, this
%    list of conditions and the following disclaimer.
%
% 2. Redistributions in binary form must reproduce the above copyright notice,
%    this list of conditions and the following disclaimer in the documentation
%    and/or other materials provided with the distribution.
%
% 3. Neither the name of the copyright holder nor the names of its
%    contributors may be used to endorse or promote products derived from
%    this software without specific prior written permission.
%
% THIS SOFTWARE IS PROVIDED BY THE COPYRIGHT HOLDERS AND CONTRIBUTORS "AS IS"
% AND ANY EXPRESS OR IMPLIED WARRANTIES, INCLUDING, BUT NOT LIMITED TO, THE
% IMPLIED WARRANTIES OF MERCHANTABILITY AND FITNESS FOR A PARTICULAR PURPOSE ARE
% DISCLAIMED. IN NO EVENT SHALL THE COPYRIGHT HOLDER OR CONTRIBUTORS BE LIABLE
% FOR ANY DIRECT, INDIRECT, INCIDENTAL, SPECIAL, EXEMPLARY, OR CONSEQUENTIAL
% DAMAGES (INCLUDING, BUT NOT LIMITED TO, PROCUREMENT OF SUBSTITUTE GOODS OR
% SERVICES; LOSS OF USE, DATA, OR PROFITS; OR BUSINESS INTERRUPTION) HOWEVER
% CAUSED AND ON ANY THEORY OF LIABILITY, WHETHER IN CONTRACT, STRICT LIABILITY,
% OR TORT (INCLUDING NEGLIGENCE OR OTHERWISE) ARISING IN ANY WAY OUT OF THE USE
% OF THIS SOFTWARE, EVEN IF ADVISED OF THE POSSIBILITY OF SUCH DAMAGE.
%
\begin{recipe}{鸡豆花}[\footnotemark]

\ingredients

\ingredient{老白鸡脯肉}{二两五}
\ingredient{干豆粉}{二钱五}
\ingredient{特级清汤}{二斤}
\ingredient{鸡蛋清}{四个}
\ingredient{味精}{三分}
\ingredient{鲜菜心}{一两}
\ingredient{盐}{六分}
\ingredient{火腿}{一钱}
\ingredient{胡椒}{一分}

\preparation

\step 鸡脯肉要选用老白鸡的(此菜所用的鸡脯肉,必须选用老鸡的,经过搅打,煮熟后
才能凝成如豆花形状。用嫩鸡肉不易凝聚,煮熟时成为“鸡淖”。其次要选用白皮鸡,才能
保证颜色雪白。雄鸡脯肉质粗不细,须捶成茸,但凝聚时仍有微细颗粒),去筋后用刀背
捶成茸,再用刀口剁数遍,剁后再捶,盛入碗内。鸡蛋清与豆粉混合后调匀。碗内的鸡茸
先用清水以竹筷搅散,再逐次加入蛋清豆粉、盐(用刀压为细末)、味精、冷特级清汤,
每加一种佐料搅匀一次,分次加入,分次搅匀,最后搅为鸡茸糊。

\step 锅内揩净,加特级清汤。烧开时放味精、盐。随后将碗内鸡肉糊以竹筷搅匀入锅,
烧至微沸时把锅移于微火上烧十分钟,鸡茸糊凝聚在一块时即成豆花状。将鲜菜心入锅汆
过,用清水漂透心,用刀修齐两端,放于碗底,将鸡豆花舀于上面,再把火腿切成细末撒
在豆花上即成。

\features

此菜形如豆花,鲜嫩、清爽,宜于夏季佐餐。

\footnotetext{
在四川所谓豆花就是豆腐脑,再经火煮比豆腐脑要老一些,用筷子就可以挑起来。
}

\end{recipe}

% vim: filetype=tex noautoindent nojoinspaces
% vim: fileencoding=utf-8 formatoptions+=m
% vim: textwidth=78 tabstop=4 shiftwidth=4 softtabstop=4
