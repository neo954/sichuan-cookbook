\begin{recipe}{包烧鸡①}

\ingredients

\ingredient{嫩仔鸡}{一只约二斤半}
\ingredient{网油}{一斤}
\ingredient{鸡蛋}{三个}
\ingredient{肥瘦肉}{二两}
\ingredient{芽菜}{二两}
\ingredient{泡辣椒}{四根}
\ingredient{千豆粉}{一两}
\ingredient{葱白}{一两}

\ingredient{姜}{三钱}
\ingredient{净生菜}{四两}
\ingredient{酱油}{七钱}
\ingredient{料酒}{二钱}
\ingredient{香油}{三钱}

\cooking

\step 鸡宰杀后去毛,去脏腹,洗净血腥,去头,去颈颡骨(颈皮留下),去鸡翘、鸡翅和四大骨。用酱油、料酒、葱、姜将鸡身内外抹匀,入味。

\step 肥瘦肉切成细丝,芽菜淘净泥沙切成短节,泡辣椒切成丝;用猪油少许先将肉丝熵散籽,再将芽菜、泡辣椒共同炒匀,起锅填入鸡腹内。鸡蛋、干豆粉调成蛋芡。

\step 网油洗净晾干铺于案上,去掉厚油边梗,抹干水份,涂上蛋芡,将鸡摆在网油上包好(共包三层,第一层不涂芡、交口处涂重点粘稳。用小铁叉一把,将鸡扁起叉上。上好后再涂一层蛋芡。

\step 用条砖砌成长约二尺、宽一尺五、高九寸的烤池,用细杠炭烧红拈入池内周围。把叉好的鸡上烤池翻烤成金黄色,至鸡熟为止。抹上香油取下叉子。网油划下切成大一字条摆于条盘的一端;烧熟的鸡去骨切成小一字条,摆入另一端;中间镶拌好的生菜入席。

\notes

网油酥香,鸡肉鲜嫩,色味倶佳。

①又名“叉烧鸡”、“罐耳鸡”

\end{recipe}

% vim: filetype=tex noautoindent
% vim: fileencoding=utf-8
% vim: textwidth=78 tabstop=4 shiftwidth=4 softtabstop=4
