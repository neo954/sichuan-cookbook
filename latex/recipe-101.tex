% BSD 3-Clause License
%
% Copyright (c) 2023 Quux System and Technology. All rights reserved.
%
% Redistribution and use in source and binary forms, with or without
% modification, are permitted provided that the following conditions are met:
%
% 1. Redistributions of source code must retain the above copyright notice, this
%    list of conditions and the following disclaimer.
%
% 2. Redistributions in binary form must reproduce the above copyright notice,
%    this list of conditions and the following disclaimer in the documentation
%    and/or other materials provided with the distribution.
%
% 3. Neither the name of the copyright holder nor the names of its
%    contributors may be used to endorse or promote products derived from
%    this software without specific prior written permission.
%
% THIS SOFTWARE IS PROVIDED BY THE COPYRIGHT HOLDERS AND CONTRIBUTORS "AS IS"
% AND ANY EXPRESS OR IMPLIED WARRANTIES, INCLUDING, BUT NOT LIMITED TO, THE
% IMPLIED WARRANTIES OF MERCHANTABILITY AND FITNESS FOR A PARTICULAR PURPOSE ARE
% DISCLAIMED. IN NO EVENT SHALL THE COPYRIGHT HOLDER OR CONTRIBUTORS BE LIABLE
% FOR ANY DIRECT, INDIRECT, INCIDENTAL, SPECIAL, EXEMPLARY, OR CONSEQUENTIAL
% DAMAGES (INCLUDING, BUT NOT LIMITED TO, PROCUREMENT OF SUBSTITUTE GOODS OR
% SERVICES; LOSS OF USE, DATA, OR PROFITS; OR BUSINESS INTERRUPTION) HOWEVER
% CAUSED AND ON ANY THEORY OF LIABILITY, WHETHER IN CONTRACT, STRICT LIABILITY,
% OR TORT (INCLUDING NEGLIGENCE OR OTHERWISE) ARISING IN ANY WAY OUT OF THE USE
% OF THIS SOFTWARE, EVEN IF ADVISED OF THE POSSIBILITY OF SUCH DAMAGE.
%
\begin{recipe}{包烧鸡}[\footnotemark]

\ingredients

\ingredient{嫩仔鸡}{一只约二斤半}
\ingredient{网油}{一斤}
\ingredient{鸡蛋}{三个}
\ingredient{肥瘦肉}{二两}
\ingredient{芽菜}{二两}
\ingredient{泡辣椒}{四根}
\ingredient{干豆粉}{一两}
\ingredient{葱白}{一两}
\ingredient{姜}{三钱}
\ingredient{净生菜}{四两}
\ingredient{酱油}{七钱}
\ingredient{料酒}{二钱}
\ingredient{香油}{三钱}

\preparation

\step 鸡宰杀后去毛,去脏腹,洗净血腥,去头,去颈颡骨(颈皮留下),去鸡翘、鸡翅
和四大骨。用酱油、料酒、葱、姜将鸡身内外抹匀,入味。

\step 肥瘦肉切成细丝,芽菜淘净泥沙切成短节,泡辣椒切成丝;用猪油少许先将肉丝煵
散籽,再将芽菜、泡辣椒共同炒匀,起锅填入鸡腹内。鸡蛋、干豆粉调成蛋芡。

\step 网油洗净晾干铺于案上,去掉厚油边梗,抹干水份,涂上蛋芡,将鸡摆在网油上包
好(共包三层,第一层不涂芡),交口处涂重点粘稳。用小铁叉一把,将鸡扁起叉上。上
好后再涂一层蛋芡。

\step 用条砖砌成长约二尺、宽一尺五、高九寸的烤池,用细杠炭烧红拈入池内周围。把
叉好的鸡上烤池翻烤成金黄色,至鸡熟为止。抹上香油取下叉子。网油划下切成大一字条
摆于条盘的一端;烧熟的鸡去骨切成小一字条,摆入另一端;中间镶拌好的生菜入席。

\features

网油酥香,鸡肉鲜嫩,色味俱佳。

\footnotetext{
又名“叉烧鸡”、“罐耳鸡”。
}

\end{recipe}

% vim: filetype=tex noautoindent nojoinspaces
% vim: fileencoding=utf-8 formatoptions+=m
% vim: textwidth=78 tabstop=4 shiftwidth=4 softtabstop=4
