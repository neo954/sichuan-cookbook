\begin{recipe}{鸡蒙葵菜}

\ingredients

\ingredient{鸡脯肉}{二两五}
\ingredient{蒸菜}{一斤半}
\ingredient{猪肥膘}{一两三}
\ingredient{水豆粉}{二钱}
\ingredient{鸡蛋清}{二个}
\ingredient{盐}{二分五}
\ingredient{猪瘦肉}{二两五}
\ingredient{味精}{二分}
\ingredient{特级清汤}{二斤}

\preparation

\step 鸡脯肉上的一层白膜用刀剽去,用刀背捶成茸,边捶边用刀尖剔尽肉内细筋,再用
刀剁两次(使鸡脯剁成极细的茸)。猪肥膘用刀剁成细泥如油脂状。鸡蛋清及鸡茸倒在碗
内,用竹筷使力向着一定方向不间歇地搅打十分钟,要搅散搅匀,直到把它滴一点在水面
上不沉底为止。这时分五次加入清水一两,边加边搅。搅匀后再放入盐、水豆粉,再同样
搅匀;而后又加入肥膘泥,继续搅匀成稀糊状。把瘦肉捶成茸,盛入碗内,用清水调散待
用。

\step 葵莱(冬苋菜)淘洗干净,把菜心(苞)连嫩茎掐下四十朵,再淘洗后用清水泡约
五分钟,捞出再换水洗一次(主要洗净菜的泥沙杂物)。洗净后把菜心倒入开水锅内,待
水重开,菜刚熟即用漏瓢捞入清水内冷透。随即捞出挤干水,抖散,用刀把每根葵菜修成
一寸长,使之整齐,摆在盘内。

炒锅紋入清水,烧至七成开时(水冒起鱼眼睛泡时),即将盘内葵菜逐一用手在稀糊状的
鸡茸内裹一层鸡茸,一朵一朵放入锅内,葵菜放完水已全开,鸡茸则全蒙在菜上,随即把
锅端离火口。

\step 将另一炒锅放入清汤烧开,加入盐,把调散的肉茸水一半倒入,用汤瓢在炒锅搅一
转,汤面则浮一层肉沫,即将锅端离火口一半,使锅内汤成半开状,浮沬则逐渐集中,随
用漏瓢撇去浮沫,苒把锅放回原处,倒入剩下的一半肉茸,开锅后同样撇去浮沫。此时汤
已变得清彻如白水,即将蒙好的葵菜用漏瓢捞入汤内煮一下,加味精后倒入碗内即成。

\features

此菜汤色清亮,菜嫩汤鲜,清爽可口,特别适宜于老年人食用。

\end{recipe}

% vim: filetype=tex noautoindent nojoinspaces
% vim: fileencoding=utf-8 formatoptions+=m
% vim: textwidth=78 tabstop=4 shiftwidth=4 softtabstop=4
