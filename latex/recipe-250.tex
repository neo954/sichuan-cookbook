\begin{recipe}[筋鞭鸽蛋]{筋尾鸽蛋}

\ingredients

\ingredient{鸽蛋}{二十个}
\ingredient{仔蹄筋}{三十根}
\ingredient{猪尾}{一斤半}
\ingredient{青笋}{二斤}
\ingredient{冰糖汁}{五钱}
\ingredient{酱油}{一两}
\ingredient{盐}{五分}
\ingredient{味精}{三分}
\ingredient{鸡油}{一两}
\ingredient{胡椒}{五分}
\ingredient{料酒}{二两}
\ingredient{化猪油}{一斤耗二两}
\ingredient{姜}{三钱}
\ingredient{葱}{五钱}
\ingredient{花椒}{约十粒}
\ingredient{清汤}{二斤}

\preparation

I猪尾拈净残毛,刮洗干净,去掉过细过粗的两头,只 留中间粗细均匀的一段,微煮后凉
冷,宰成一寸五长的节 子。将蹄筋洗干净后用纱布包起,放入汤锅内煮至七成火(倒 炤
不钯的程度)捞起。青笋削皮,削成六分大的慈菇形,再 倒入猪油锅内跑一次,将水气
炸干,捞起入碗待用。

冰糖炒成糖汁,姜、葱、花椒用白油炒出香味后捞起,包 入一个纱布内,再放入罐内。
将微煮后的鸡骨垫底,再放入蹄 筋、猪尾、清汤,用旺火烧开,打去泡沫,再依次放入
料酒、 胡椒、葱、姜(布包)、酱油、糖汁、盐,后改用小火慢慢烧 炤,分别将蹄筋、
猪尾捞起。

鸽蛋用清水煮好,去壳,在油锅内跑一次。再将猪尾、 蹄筋捞起定碗,定碗的形式是先
用鸽蛋三个在碗底中心,摆 成三方形,每一个空间相一节猪尾,周围再围一转蹄筋,外
面 围猪尾,猪尾外又是鸽蛋,其余的作为底子,底内再放入慈菇

形的青笋,青笋上面放一两生鸡油,再掺入二两包罐内的原 汁上笼蒸,蒸至青笋已炤即
取出,翻入大圆盘内,再将罐内 的原汁滤过收成浓汁淋于盘内即成。

\features

美观大方,味浓而鲜,富于营养,适合老年。

\end{recipe}

% vim: filetype=tex noautoindent nojoinspaces
% vim: fileencoding=utf-8 formatoptions+=m
% vim: textwidth=78 tabstop=4 shiftwidth=4 softtabstop=4
