\begin{recipe}[筋鞭鸽蛋]{筋尾鸽蛋}

\ingredients

\ingredient{鸽蛋}{二十个}
\ingredient{仔蹄筋}{三十根}
\ingredient{猪尾}{一斤半}
\ingredient{青笋}{二斤}
\ingredient{冰糖汁}{五钱}
\ingredient{酱油}{一两}
\ingredient{盐}{五分}
\ingredient{味精}{三分}
\ingredient{鸡油}{一两}
\ingredient{胡椒}{五分}
\ingredient{料酒}{二两}
\ingredient{化猪油}{一斤耗二两}
\ingredient{姜}{三钱}
\ingredient{葱}{五钱}
\ingredient{花椒}{约十粒}
\ingredient{清汤}{二斤}

\preparation

\step 猪尾拈净残毛,刮洗干净,去掉过细过粗的两头,只留中间粗细均匀的一段,微煮
后凉冷,宰成一寸五长的节子。将蹄筋洗干净后用纱布包起,放入汤锅内煮至七成火(倒
𤆵不𤆵的程度)捞起。青笋削皮,削成六分大的慈菇形,再倒入猪油锅内跑一次,将水气
炸干,捞起入碗待用。

\step 冰糖炒成糖汁,姜、葱、花椒用白油炒出香味后捞起,包入一个纱布内,再放入罐
内。将微煮后的鸡骨垫底,再放入蹄筋、猪尾、清汤,用旺火烧开,打去泡沫,再依次放
入料酒、胡椒、葱、姜(布包)、酱油、糖汁、盐,后改用小火慢慢烧𤆵,分别将蹄筋、
猪尾捞起。

\step 鸽蛋用清水煮好,去壳,在油锅内跑一次。再将猪尾、蹄筋捞起定碗,定碗的形式
是先用鸽蛋三个在碗底中心,摆成三方形,每一个空间相一节猪尾,周围再围一转蹄筋,
外面围猪尾,猪尾外又是鸽蛋,其余的作为底子,底内再放入慈菇形的青笋,青笋上面放
一两生鸡油,再掺入二两包罐内的原汁上笼蒸,蒸至青笋已𤆵即取出,翻入大圆盘内,再
将罐内的原汁滤过收成浓汁淋于盘内即成。

\features

美观大方,味浓而鲜,富于营养,适合老年。

\end{recipe}

% vim: filetype=tex noautoindent nojoinspaces
% vim: fileencoding=utf-8 formatoptions+=m
% vim: textwidth=78 tabstop=4 shiftwidth=4 softtabstop=4
