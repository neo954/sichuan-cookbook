\begin{recipe}{绣球鱼翅}

\ingredients

\ingredient{小毛鱼翅}{四两}
\ingredient{鸡腩肉}{二两}
\ingredient{鸡蛋清}{三个}
\ingredient{肥膘}{一两五}
\ingredient{火腿}{一两}
\ingredient{丝瓜}{一根}

\ingredient{水发木耳}{三朵}
\ingredient{水豆粉}{五钱}
\ingredient{盐}{四分}
\ingredient{料酒}{三钱}
\ingredient{味精}{二分}
\ingredient{胡椒}{二分}
\ingredient{特级清、,务}{二斤}
\ingredient{蛋皮}{一小张}

\cooking

\step 鱼翅先在沸水内煮数分钟,使之干净柔软;盛入蒸碗 内,掺沸水,上笼蒸半小时取出;去尽杂质,再放入料酒、盐、 清汤汆一次,捞起晾干水气;火腿、丝瓜皮、木耳、蛋皮各 用刀切成一寸长的细丝,连同鱼翅造勻待用。

鸡脯、肥膘、鸡蛋清等打成“鸡糁”。用大磁盘一个,先 在盘内将造匀的鱼翅等铺幵三分之一,把鸡糁萏成二十八个 糁元子放在丝上面;再将其余的三分之二丝子均匀地盖在面 上;然后再一个一个地拈入手中,将各丝连同糁元子团成绣球 形;在垫好纱布的笼内蒸五分钟成半成品;再拣入鱼碗内, 掺好汤、胡椒、料酒少许。走菜前气热,临走时泌去蒸汤不 用,将元子盛入碗内,再将烧好的特级清汤注入即可。

\notes

质优色美,味鲜清淡,适用夏天头菜。

\end{recipe}

% vim: filetype=tex noautoindent
% vim: fileencoding=utf-8
% vim: textwidth=78 tabstop=4 shiftwidth=4 softtabstop=4
