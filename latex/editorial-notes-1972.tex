% BSD 3-Clause License
%
% Copyright (c) 2023, 2024 Quux System and Technology. All rights reserved.
%
% Redistribution and use in source and binary forms, with or without
% modification, are permitted provided that the following conditions are met:
%
% 1. Redistributions of source code must retain the above copyright notice, this
%    list of conditions and the following disclaimer.
%
% 2. Redistributions in binary form must reproduce the above copyright notice,
%    this list of conditions and the following disclaimer in the documentation
%    and/or other materials provided with the distribution.
%
% 3. Neither the name of the copyright holder nor the names of its
%    contributors may be used to endorse or promote products derived from
%    this software without specific prior written permission.
%
% THIS SOFTWARE IS PROVIDED BY THE COPYRIGHT HOLDERS AND CONTRIBUTORS "AS IS"
% AND ANY EXPRESS OR IMPLIED WARRANTIES, INCLUDING, BUT NOT LIMITED TO, THE
% IMPLIED WARRANTIES OF MERCHANTABILITY AND FITNESS FOR A PARTICULAR PURPOSE ARE
% DISCLAIMED. IN NO EVENT SHALL THE COPYRIGHT HOLDER OR CONTRIBUTORS BE LIABLE
% FOR ANY DIRECT, INDIRECT, INCIDENTAL, SPECIAL, EXEMPLARY, OR CONSEQUENTIAL
% DAMAGES (INCLUDING, BUT NOT LIMITED TO, PROCUREMENT OF SUBSTITUTE GOODS OR
% SERVICES; LOSS OF USE, DATA, OR PROFITS; OR BUSINESS INTERRUPTION) HOWEVER
% CAUSED AND ON ANY THEORY OF LIABILITY, WHETHER IN CONTRACT, STRICT LIABILITY,
% OR TORT (INCLUDING NEGLIGENCE OR OTHERWISE) ARISING IN ANY WAY OUT OF THE USE
% OF THIS SOFTWARE, EVEN IF ADVISED OF THE POSSIBILITY OF SUCH DAMAGE.
%
\begin{center}%
\Large\bfseries%
{\hbadness=10000\makebox[5.5em][s]{编印说明}}%
\end{center}%

在伟大领袖毛主席{\sffamily“发展经济,保障供给”}的财政经济工作总方针的指引下,
商业战线同各条战线一样,形势大好,越来越好。我们饮食服务工作,为了适应革命、生
产大好形势的不断发展,进一步继承和发扬祖国的烹调遗产,提高饮食部门的口味质量,
增加花色品种,使饮食烹调技术更好地为广大工农兵服务,我们在有关部门的帮助和支持
下,编印了这本《四川菜谱》。

在编写《四川菜谱》的过程中,我们参照了过去积累的资料,同时也结合我们在实际操作
中初步摸索到的一些经验,特别是经过无产阶级文化大革命运动,遵照毛主席关于%
{\sffamily“我们必须继承一切优秀的文学艺术遗产,批判地吸收其中一切有益的东西”}%
的教导,对菜谱的品种、名称、内容等各个方面作了必要的修改整理并有所取舍。这次主
要参照了过去我们编写的《中国名菜谱第七辑》和搜集《重庆名菜谱》部分菜肴,内容分
肉食、鸡鸭、鱼虾、山珍海味、甜食、蔬菜、其它七个大类,共三百一十二个品种,基本
上按由浅入深,由简到繁的次序编排。

《四川菜谱》作为内部资料编印,主要是作为我们培训技术的教材用的。由于我们的理论
水平和技术水平都比较低,又缺少经验,这次编写《四川菜谱》一定会有不少缺点,希望
广大饮食服务战线的同志们,特别是具有丰富实践经验的老师傅,给我们提出宝贵的意
见,以便及时改进。为使饮食烹调技术更好地为广大工农兵服务而共同努力。

\vspace{-.335823\baselineskip}
\begin{flushright}
\singlespacing{
	成都市饮食公司革命委员会%
		\mbox{\hspace{2em}}\\
	技术培训班教研组集体整编%
		\mbox{\hspace{2em}}\\
	一九七二年七月%
		\mbox{\hspace{4em}}%
}
\end{flushright}

% vim: filetype=tex noautoindent nojoinspaces
% vim: fileencoding=utf-8 formatoptions+=m
% vim: textwidth=78 tabstop=4 shiftwidth=4 softtabstop=4
