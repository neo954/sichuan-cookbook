\begin{recipe}{红烧足鱼}

\ingredients

\ingredient{足鱼}{约二斤}
\ingredient{化猪油}{一1两五}
\ingredient{味精}{三分}
\ingredient{猪肉}{半斤}
\ingredient{姜}{四钱}
\ingredient{胡椒面}{一分}
\ingredient{鸡翅八个}{约五两}
\ingredient{葱白}{四钱}
\ingredient{清汤}{一斤三两}
\ingredient{独蒜}{二两}
\ingredient{盐}{六分}
\ingredient{鸡沄}{二斤}

\ingredient{料酒}{六钱}
\ingredient{酱油}{七钱}
\ingredient{三钱}{咸红酱油}
\ingredient{三钱}{}

\ingredient{二、}{制.方法:}

\ingredient{1.}{猪肉选用五花的,洗净切成块;鸡翅膀切去翅尖#砍}
\ingredient{为两段;猪肉与鸡翅一同用沸水在锅内汆五分钟,除尽浮沫-后捞出。将锅置于旺火上,放入猪油和一块拍松的姜、葱白}{段,:再放入猪肉、鸡翅,一并煸四、五分钟,加红、白酱油 和盐,再烹入料酒,加入鸡汤烧沸,倒入砂锅盖好,放在偏 火眼上,锆着即为老汤。净大蒜装盘上笼蒸粑。}

\ingredient{2,}{将足鱼平放在菜墩上,腹部向上,头即伸茁,等头伸}
\ingredient{出后一刀斩断,放净血;锅内水烧沸,将足鱼放入汆约十分}{钟穴用手指甲能刮掉裙边上的粗皮为度,过久或不够火候都 不易刮掉,汆的时间可视足鱼的老嫩而增减。)捞起,用小 刀将足鱼壳周围的裙边、腹部软皮与四肢粗皮刮净,再入沸 水中汆约十五分钟,去掉脊上甲壳和内脏,用清水洗净,切 去足爪,横砍成二寸长的块。用沸水汆约五分钟后,在清水 中漂约十五分钟,然后再用清汤加姜和葱白段,用微火汆约 八分钟,去其泥腥味。捞出后(汆汤不用)放入煨有猪肉和 鸡翅的“老汤”锅内煩十分钟,至足鱼肉把透,汤汁约余四 两左右时,即倒入锅中,将猪肉块取出另作他用,鸡翅放在 盘中心。%后将锅放在旺火上,将大蒜从笼内取出放入,用 汤瓢轻轻翻4搅,等滋汁浓缩约二两多时,加入胡椒面、味 精,淋入鸡油造勻,倒在盘中鸡翅上,再用竹筷将裙边拈在 面上即成。}

\notes

此菜色泽金黄,味浓而鲜美,肉粑可口,富于营养。但 须注意足鱼肉不可与苋菜同食。

\end{recipe}

% vim: filetype=tex noautoindent
% vim: fileencoding=utf-8
% vim: textwidth=78 tabstop=4 shiftwidth=4 softtabstop=4
