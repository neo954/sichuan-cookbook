\begin{recipe}{抄手鸭子}

\ingredients

\ingredient{软烧仔鸭}{一只}
\ingredient{抄手}{二十个}
\ingredient{清汤}{二两}
\ingredient{姜}{三钱}
\ingredient{葱}{三钱}
\ingredient{大蒜}{一个}
\ingredient{菜油}{二斤耗三两}
\ingredient{盐}{少许}

\cooking

\step 姜、蒜切花形片,葱切马耳朵待用。

\step 菜油在锅内烧至七成热,将抄手倒下,炸熟捞起;烧鸭撇去卤汁,用“须子”提着不断浇以沸油烫过,放在墩上将骨去净,宰成一字条,摆于盘内成“三叠水”;上面放以切好的姜、蒜片和葱,周围镶上炸过的抄手即成。

\notes

软烧仔鸭本为成都人所喜爱,再经过油烫更觉香脆,配 以油炸抄手上席,是古老川菜中的别具一格。

\end{recipe}

% vim: filetype=tex noautoindent
% vim: fileencoding=utf-8
% vim: textwidth=78 tabstop=4 shiftwidth=4 softtabstop=4
