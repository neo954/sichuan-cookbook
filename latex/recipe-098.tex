\begin{recipe}{桃酥鸡糕}

\ingredients

\ingredient{鸡脯肉}{四两}
\ingredient{鸡蛋}{四个}
\ingredient{猪肥膘}{二两}
\ingredient{桃仁}{二两}
\ingredient{干豆粉}{二两}
\ingredient{盐}{四分}
\ingredient{味精}{三分}
\ingredient{料酒}{三钱}
\ingredient{白糖}{三钱}
\ingredient{醋}{三钱}
\ingredient{香油}{五钱}
\ingredient{生菜}{三两}
\ingredient{菜油}{一斤耗二两}

\preparation

\step 先将桃仁用温热水泡十分钟,撕去粗皮(油烂的不要)。撕完后将锅放至旺火上,
倒入菜油半斤,烧至五成火,将桃仁滤干水份,倒下炸酥,打起放在墩子上,再用刀将桃
仁剁细(绿豆大)放入碗内。

\step 将鸡脯肉去下皮朦,放在墩子上用刀背捶茸,抽尽细筋,再捶数十下,肉内无籽,
用刀口剁三、四次,将细纤维陷在墩子上,用刀轻轻的将肉铲起放在一边。将肥膘肉剁细
如油脂,放在一个碗内装好。鸡茸放入木瓢加入蛋清,用手在瓢内搅十几分钟,加入八钱
清水及料酒,边搅边加,作四、五次加完,再放入调匀后的干豆粉及剁好的猪肥膘、桃
仁、味精,搅十几分钟,再加盐搅十几分钟,装入九寸盘内(盘心抹上一层油),刮平成
方形,高四分,放入笼内蒸十分钟。

\step 将生菜用清水淘洗干净,捞起将水份滤干,装入碗内;再取出笼内的鸡糕,用刀起
开,切成一寸二分长、四分宽的条块;剩余的干豆粉用筛子筛细,将切好的鸡糕拌匀豆粉
装入盘内。

\step 将锅放在旺火上倒入菜油,烧至八成火后,将鸡糕全部倒下,炸至浅黄色,皮已
酥,泌去炸油,将香油淋于鸡糕,起锅入条盘的一端;再将生菜用糖、醋、香油拌好,放
入条盘的另一端即成。

\features

酥、香、脆嫩,特别适宜酒肴。

\end{recipe}

% vim: filetype=tex noautoindent nojoinspaces
% vim: fileencoding=utf-8 formatoptions+=m
% vim: textwidth=78 tabstop=4 shiftwidth=4 softtabstop=4
