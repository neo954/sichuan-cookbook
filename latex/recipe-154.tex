% BSD 3-Clause License
%
% Copyright (c) 2023 Quux System and Technology. All rights reserved.
%
% Redistribution and use in source and binary forms, with or without
% modification, are permitted provided that the following conditions are met:
%
% 1. Redistributions of source code must retain the above copyright notice, this
%    list of conditions and the following disclaimer.
%
% 2. Redistributions in binary form must reproduce the above copyright notice,
%    this list of conditions and the following disclaimer in the documentation
%    and/or other materials provided with the distribution.
%
% 3. Neither the name of the copyright holder nor the names of its
%    contributors may be used to endorse or promote products derived from
%    this software without specific prior written permission.
%
% THIS SOFTWARE IS PROVIDED BY THE COPYRIGHT HOLDERS AND CONTRIBUTORS "AS IS"
% AND ANY EXPRESS OR IMPLIED WARRANTIES, INCLUDING, BUT NOT LIMITED TO, THE
% IMPLIED WARRANTIES OF MERCHANTABILITY AND FITNESS FOR A PARTICULAR PURPOSE ARE
% DISCLAIMED. IN NO EVENT SHALL THE COPYRIGHT HOLDER OR CONTRIBUTORS BE LIABLE
% FOR ANY DIRECT, INDIRECT, INCIDENTAL, SPECIAL, EXEMPLARY, OR CONSEQUENTIAL
% DAMAGES (INCLUDING, BUT NOT LIMITED TO, PROCUREMENT OF SUBSTITUTE GOODS OR
% SERVICES; LOSS OF USE, DATA, OR PROFITS; OR BUSINESS INTERRUPTION) HOWEVER
% CAUSED AND ON ANY THEORY OF LIABILITY, WHETHER IN CONTRACT, STRICT LIABILITY,
% OR TORT (INCLUDING NEGLIGENCE OR OTHERWISE) ARISING IN ANY WAY OUT OF THE USE
% OF THIS SOFTWARE, EVEN IF ADVISED OF THE POSSIBILITY OF SUCH DAMAGE.
%
\begin{recipe}{生爆虾仁}

\ingredients

\ingredient{鲜虾}{二斤}
\ingredient{熟火腿}{一两}
\ingredient{味精}{二分}
\ingredient{鸡蛋清}{二个}
\ingredient{料酒}{二钱}
\ingredient{化猪油}{三两耗二两}
\ingredient{干豆粉}{五钱}
\ingredient{胡椒面}{一分}
\ingredient{慈菇}{二两}
\ingredient{盐}{五分}

\preparation

\step 慈菇削皮,淘洗干净,与火腿分别切成碎丁。料酒、胡椒面、味精、盐、水豆粉少
许加鸡汤兑成滋汁。

\step 鲜虾用手挤仁去壳,漂于清水中,淘去杂质;然后将水气滤干,加上蛋清豆粉、盐
少许拌匀。

\step 炒锅在旺火上烧红,先舀油浪匀滗去,另舀新油,烧至五成火候,即倒下虾仁,先
用筷子滑散,多余的油滗去,即下慈菇、火腿丁炒转,再下滋汁,炒几转即起锅。

\features

鲜嫩可口,富于营养。按以上作法,将慈菇、火腿配料改成鲜豌豆米即为“翡翠虾仁”。

\end{recipe}

% vim: filetype=tex noautoindent nojoinspaces
% vim: fileencoding=utf-8 formatoptions+=m
% vim: textwidth=78 tabstop=4 shiftwidth=4 softtabstop=4
