% BSD 3-Clause License
%
% Copyright (c) 2023 Quux System and Technology. All rights reserved.
%
% Redistribution and use in source and binary forms, with or without
% modification, are permitted provided that the following conditions are met:
%
% 1. Redistributions of source code must retain the above copyright notice, this
%    list of conditions and the following disclaimer.
%
% 2. Redistributions in binary form must reproduce the above copyright notice,
%    this list of conditions and the following disclaimer in the documentation
%    and/or other materials provided with the distribution.
%
% 3. Neither the name of the copyright holder nor the names of its
%    contributors may be used to endorse or promote products derived from
%    this software without specific prior written permission.
%
% THIS SOFTWARE IS PROVIDED BY THE COPYRIGHT HOLDERS AND CONTRIBUTORS "AS IS"
% AND ANY EXPRESS OR IMPLIED WARRANTIES, INCLUDING, BUT NOT LIMITED TO, THE
% IMPLIED WARRANTIES OF MERCHANTABILITY AND FITNESS FOR A PARTICULAR PURPOSE ARE
% DISCLAIMED. IN NO EVENT SHALL THE COPYRIGHT HOLDER OR CONTRIBUTORS BE LIABLE
% FOR ANY DIRECT, INDIRECT, INCIDENTAL, SPECIAL, EXEMPLARY, OR CONSEQUENTIAL
% DAMAGES (INCLUDING, BUT NOT LIMITED TO, PROCUREMENT OF SUBSTITUTE GOODS OR
% SERVICES; LOSS OF USE, DATA, OR PROFITS; OR BUSINESS INTERRUPTION) HOWEVER
% CAUSED AND ON ANY THEORY OF LIABILITY, WHETHER IN CONTRACT, STRICT LIABILITY,
% OR TORT (INCLUDING NEGLIGENCE OR OTHERWISE) ARISING IN ANY WAY OUT OF THE USE
% OF THIS SOFTWARE, EVEN IF ADVISED OF THE POSSIBILITY OF SUCH DAMAGE.
%
\begin{recipe}{格花豆腐皮}

\ingredients

\ingredient{豆腐皮}{八张}
\ingredient{鸡蛋}{五个}
\ingredient{干豆粉}{一两}
\ingredient{绿色菜叶}{五两}
\ingredient{面粉}{一两}
\ingredient{化鸡油}{一两}
\ingredient{瘦火腿}{二两}
\ingredient{盐}{五分}
\ingredient{味精}{二分}
\ingredient{胡椒}{一分}
\ingredient{料酒}{五钱}
\ingredient{建兰菜}{五两}
\ingredient{水豆粉}{五钱}
\ingredient{清汤}{五两}

\preparation

\step 鸡蛋清、豆粉、面粉,调成不干不清的蛋清浆。火腿铡成细末;绿色菜叶泹后挤
干,铡成细末;蛋黄调散。火腿末、菜叶末、蛋黄分开放入碗内,再将蛋清浆分别放入这
三种色的碗内调匀待用。

\step 豆腐皮上笼蒸至五分钟,取出沾干气水,每张趁热抹上一层一个颜色,抹完为止。
再一张一张地重起,成为三种颜色。四张豆腐皮一叠,如果再重四张,共为两叠。上笼蒸
五分钟取出,晾冷用刀切成一寸八长、二分五宽的条子定碗,定时有颜色的一面挨碗底,
以二十四条摆成万字形;其余的用碗装满,加入鸡油、料酒及清汤少许,再上笼蒸五分
钟。走菜时用白油将建兰菜炒好垫入盘内,取出笼内的豆腐皮翻入菜上,锅内加入清汤,
见开泌下碗内的原汁,加入味精,起锅时加入鸡油,倒入盘内即成。

\features

色彩鲜艳,味道鲜美。

\end{recipe}

% vim: filetype=tex noautoindent nojoinspaces
% vim: fileencoding=utf-8 formatoptions+=m
% vim: textwidth=78 tabstop=4 shiftwidth=4 softtabstop=4
