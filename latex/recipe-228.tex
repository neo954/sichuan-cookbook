\begin{recipe}{格花豆腐皮}

\ingredients

\ingredient{豆腐皮}{八张}
\ingredient{鸡蛋}{五个}
\ingredient{干豆粉}{一两}
\ingredient{绿色菜叶}{五两}
\ingredient{面粉}{一两}
\ingredient{化鸡油}{一两}
\ingredient{瘦火腿}{二两}
\ingredient{盐}{五分}
\ingredient{味精}{二分}
\ingredient{胡椒}{一分}
\ingredient{料酒}{五钱}
\ingredient{建兰菜}{五两}
\ingredient{水豆粉}{五钱}
\ingredient{清汤}{五两}

\cooking

\step 鸡蛋清、豆粉、面粉,调成不干不清的蛋清浆。火腿1铡成细末;绿色菜叶沮后挤
干,铡成细末;蛋黄调散。火腿末、菜叶末、蛋黄分开放入碗内,再将蛋清浆分别放入这
三种色的碗内调匀待用。

\step 豆腐皮上笼蒸至五分钟,取出沾干气水,每张趁热抹上一层一个颜色,抹完为止。
再一张一张地重起,成为二种颜,色。四张豆腐皮一迭,如果再重四张,共为两迭。上笼
蒸.五分钟取出,晾冷用刀切成一寸八长、二分五宽的条子定碗,定时有颜色的一面挨碗
底,以二十四条摆成万字形;其余的用碗装满,加入鸡油、料酒及清汤少许,再上笼蒸五
分钟。走菜时用白油将建兰菜炒好垫入盘内,取出笼内的豆腐皮翻入菜上,锅内加入清汤
,见开泌下碗内的原汁,加入味精, 起锅时加入鸡油,倒入盘内即成。

\notes

色彩鲜艳,味道鲜美。

\end{recipe}

% vim: filetype=tex noautoindent
% vim: fileencoding=utf-8 formatoptions+=m
% vim: textwidth=78 tabstop=4 shiftwidth=4 softtabstop=4
