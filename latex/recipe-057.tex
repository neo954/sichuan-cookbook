\begin{recipe}{叉烧奶猪}

\ingredients

\ingredient{乳猪一只}{约十三斤}
\ingredient{咸红酱油}{二两}
\ingredient{香油}{二两}
\ingredient{社炭}{八斤}

\preparation

\step 饲养与宰杀:将乳猪买回,用稀饭喂养三、四天,以去其腹内污秽和长膘。杀时用
尖刀从咽喉处杀死,倒提后足流去血浆。再用干细炭灰匀撒猪身以手抹匀,以六至七成开
的热水烫洗去毛,从头至尾用镊子(头部凹缝处可用烧红的铁钎烙)清洗干净。然后从喉
头起将猪逢中对半剖开,去肚腹内脏,留腰子不取。

\step 盘足:将两前足屈起,从挨近颈下裙边处(用尖刀在两边各挖一个孔)塞入腹腔。
另用筷子两支,削成竹钉,分别穿过猪蹄前端,以防烤时滑出腹外;后足亦如前法盘入腹
腔,用筷子穿好。

\step 出坯:在热水中将猪皮烫伸,取出用干净布抹干水气,再以红酱油将皮抹成黄红
色。此时从杀口处取出颈颡骨一节(约一寸长),再由腹腔内将脊椎与肋骨接榫处(俗呼
龙骨)从中切剖约二寸长,以便上叉时猪身平伏不致突起。

\step 上叉:将叉子(铁制、两股)从猪后腿近肘处刺入猪身,穿过腹腔,在一、二匹肋
骨缝中刺进猪肉层,再从猪头两耳根下穿出(此时叉尖约与猪眼平行);将两耳边用刀修
去一部分,使之直立略呈尖形;将尾巴用竹签穿过,扭成乙形,尾尖向上略微倾向猪头。
用尖头竹签,从肚腹两边裙边处刺进猪肉,以不伤皮为度。在肋缝处可以斜刺,大约每匹
肋骨刺三下即可,作用是避免烧烤时猪身皮面起泡。注意切勿将猪皮刺伤。

\step 烧烤:先砌砖池一个,长二尺五、宽二尺、高六寸五,池底垫细炭灰一层约五分厚。

\step 烤头:将杠炭烧红,置于砖池中央,将猪头向下,腹向池外,用火烤至头顶,烤到
脸泡成淡黄色,耳朵烤干为止。

\step 吊膛:将杠炭火勾平散开,把叉子头、尖,平放在砖池上。此时猪腹向下,烤至水
气吊干,以手触之无水份,不粘手为度。此时再将杠炭勾至砖池周围,中间不留炭火,杠
炭分布在叉柄叉尖两端占三分之二(即放猪头、尾处),两侧三分之一。烤时猪背向下,
腹向上。在往两侧翻动时,起初角度应稍大,至翻到裙边肉处。到裙边微带黄色时,角度
渐小,直到只翻烤背部。烤至全身火色均匀,均呈黄色时,再周身滚转烤,至烤成深琥珀
色为止,用筷子敲一下清脆作卜卜声。(翻烤时手要稳,火要匀,并密切注意如发生起泡
现象,必须立即将叉离开砖池,用竹签由腹腔内距起泡不远处刺进以泄其气。如任其爆烈
则吃时顶牙,影响质量。)
烤好后,将杠炭散开铺满砖池,将猪身抹上香油,在火上翻滚烤约四五遍离池。将叉尖用
布搽净,用手扶住猪身后肘将叉取出。
将烤好的猪放在干净案板上,用尖刀将猪皮划成长方形骨牌片。划时先从头至尾划直线,
后划横线。再用小刀启松,使皮、肉相离,但仍保持原形。随配点心二盘(荷叶饼、烧饼
均可),连同葱酱碟上席。
此菜一般食皮弃肉,亦有将腰子、耳朵、尾巴、脑花、水肉(即皮下之肉)另切一盘上席
者谓之“小件头”,吃法与上同。

\features

此菜系传统烧烤大菜,用于上等筵席。

\end{recipe}

% vim: filetype=tex noautoindent nojoinspaces
% vim: fileencoding=utf-8 formatoptions+=m
% vim: textwidth=78 tabstop=4 shiftwidth=4 softtabstop=4
