\begin{recipe}{清炖牛尾汤}

\ingredients

\ingredient{黄牛尾}{五斤}
\ingredient{老姜(拍破)}{一'两五}

\ingredient{绍酒}{二两五}
\ingredient{味精}{一钱}
\ingredient{上等花椒}{六分}
\ingredient{猪油}{一两六}
\ingredient{净鲜肥母鸡}{三斤}
\ingredient{食盐(炒后磨细)}{七分半}

\cooking

将牛尾修去残余皮子,清洗干净。在每一骨节缝处切进三分之二深,不切断,入清水内漂二十分钟。

用大锑锅盛开水六斤,放入牛尾,置旺火上烧开,打去泡沫,继下老姜、花椒、绍酒、母鸡(去头及爪),再行烧开后,移微火上炖,每隔一点多钟翻动一下,以免粘锅。炖至七成熟时,以干净稀白布滤去汤中的花椒、老姜,重行放在旺火上烧开后,又置微火上烛炤为止。前后共需炖七小时半。母鸡只取其汁,烛好后,鸡取出作别用。吃时加味精食盐和猪油。

\notes

汤浓鲜、香美,富营养。

\end{recipe}

% vim: filetype=tex noautoindent
% vim: fileencoding=utf-8
% vim: textwidth=78 tabstop=4 shiftwidth=4 softtabstop=4
