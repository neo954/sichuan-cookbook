\begin{recipe}{豆渣猪头}

\ingredients

\ingredient{整猪头一个}{九斤}
\ingredient{化猪油}{一斤}
\ingredient{料酒}{半斤}
\ingredient{冰糖汁}{一-两}
\ingredient{八角}{二钱}
\ingredient{姜}{七钱}
\ingredient{花椒}{约二十颗}
\ingredient{味精}{三分}
\ingredient{生细豆淹}{一斤半}
\ingredient{清汤}{三斤}
\ingredient{捞糟}{二两五}
\ingredient{盐}{六分}
\ingredient{酱油}{一'两}
\ingredient{草果}{二钱}
\ingredient{大葱}{二两}
\ingredient{胡椒}{十余颗}

\cooking

\step 猪头一个(不要猪耳),先用夹子夹去猪毛,凹缝中的毛可用铁钎烧红烙去(注意
所有毛必须去净)。随后再用小刀刮洗,剔去骨头,去净肉中一切骨渣。再用清水刮洗,
直到洗净。然后将锅放在炉上,倒入清水十斤,将猪头肉和猪头骨一起放入,用旺火煮五
分钟后捞出,再用清水全部刮冼干净待用。
\step 豆渣放蒸碗中,上蒸笼蒸二十分钟,出笼晾冷,用净布包着挤干水。锅放炉上烧
红,倒入猪油烧开,再放入豆渣,用微火炒五分钟。炒时注意常用汤瓢拨刮锅心,以免豆
渣巴锅。炒至油和豆渣合为一体时,再加猪油四两,继续用微火炒五分钟,然后加猪油二
两,继续用微火炒,要炒酥炒香,炒至豆渣不吐油,再吐油时泌去余油,起锅待用。
\step 姜、葱洗净,用刀把姜拍松,用稀眼净布把姜、大葱、花椒、胡椒、八角、草果包
好待用。
\step 用大面砂锅一个,将清汤、料酒、耢糟、冰糖汁、盐、酱油、姜、葱布包一起放
入,再放入猪头骨,随后将猪头肉放在头骨之上,用旺火烧开,然后将锅口用草纸趔后继
续用旺火烧四小时左右,扯去草纸,捞出猪头肉盛入大圆盘中,再将砂锅中的原汁倒入锅
中熬酽,然后放入炒好的豆渣与味精拌匀,淋于猪头肉之上入席。

\notes

此菜为成都名菜,呈深鸭黄色,肉粑烂而脂肪多,豆渣 酥香,下酒佐饭均宜。

\end{recipe}

% vim: filetype=tex noautoindent
% vim: fileencoding=utf-8 formatoptions+=m
% vim: textwidth=78 tabstop=4 shiftwidth=4 softtabstop=4
