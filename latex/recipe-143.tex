\begin{recipe}{黄焖大鲢鱼头}

\ingredients

\ingredient{鸣足鸭掌}{四对}
\ingredient{化猪油}{一斤耗二两五}

\preparation

\step 大鲢鱼头一个,挖去腮,去掉牙板骨,在水中清洗干净I对准鱼口从脑顶骨中砍开
,不要砍断。鸡足鸭掌洗干-净,在开水锅内紧一下,去掉粗皮。白菌菰用清水泡半小、
时,待菌已开始胀,用手刮洗去泥沙,再在沸水锅内煮约十分钟,使菌发至全透;大的用
刀对剖成两瓣。猪肥膘肉切成:一寸长、三分宽厚的条。

\step 猪油放入大炒锅内置于旺火上烧至七成热,即将鱼头-放入稍炸,并烹入料酒,炸
至略呈微黄色时捞出;泌去猪油,留一、二两在锅内,将猪肥膘肉及鸡鸭足、料酒、.葱
姜放入痫一下,即加入鸡汤,开锅后撇去浮沫。

\step 用头号铝锅一个,锅底垫上竹篾篦,先将熵过的鸡鸭足在锅中捞出放于篾篦上垫好
,再将锅内鱼头及猪肥膘肉、鸡汤等全部倒入锅内,即加入酱油、红酱油、金钩、白菌菰
等。然后将锅移于微火上慢慢将鱼头烧熟(约三十分钟),即用筷子将鱼肉轻轻试一下,
至鸡汤已经烧到减少三分之二时去掉姜葱、猪肥膘肉、鸡鸭足等不用,再加入冰糖和蒸好
的大蒜;烧至冰糖溶化、汤已浓缩成汁,即盛入大盘内。

\features

鲢鱼肉以细嫩著称,尤以头部为最好。采用此种久烧慢 焙的家常做法,不仅质稠色亮,味更鲜美。

⑨鲢鱼:这里实际上是指鲇鱼四川人习惯上称鲇鱼为鲢鱼。这种鱼无 鱗,多涎液,身团口大,而头平扁,刺少肉嫩,颇为鲜美.

\end{recipe}

% vim: filetype=tex noautoindent
% vim: fileencoding=utf-8 formatoptions+=m
% vim: textwidth=78 tabstop=4 shiftwidth=4 softtabstop=4
