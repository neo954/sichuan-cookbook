\begin{recipe}{清汤竹参肝膏}

\ingredients

\ingredient{竹参}{二钱}
\ingredient{瘦猪肉}{二两五}

\ingredient{鸡脯肉}{四两}
\ingredient{母鸡汤}{六杓}



\ingredient{鸡蛋(取白)}{二个}
\ingredient{绍酒}{六钱}
\ingredient{食盐}{少许}
\ingredient{胡椒}{少许}

\cooking

竹参以淘米水发胀后,轻轻搓几下漂入清水内。

鸡肝洗净以刀背捶绒,将蛋白搅散掺入鸡肝内,并加入 胡椒、食盐和鸡汤一杓,合拌调匀。以丝罗筛滤渣取汁,连 续滤三次,然后将肝汁盛在碗内,蒸约十分钟即凝结成膏。 蒸时火力要适度,火大则会起蜂窝眼,火小则引起沉淀,不 能凝结。

鸡汤五杓装在小锑锅内,在微火上烧沸,将浮油打起不 要。将瘦猪肉用刀背捶绒,以冷汤调散倾入鸡汤内,并加进 绍酒用杓搅转,俟汤再沸时用漏杓将沉渣捞起不要。再将鸡 脯肉照样进行,并随时打净浮油,即成清汤〔清汤时火力不 宜大:)。

临吃前将清汤轻轻转入肝膏碗里,并将发胀漂净的竹参 切成寸节放入,蒸熟即成。

\notes

汤清彻如镜,味鲜美可口。竹参乃本省特产。

\end{recipe}

% vim: filetype=tex noautoindent
% vim: fileencoding=utf-8
% vim: textwidth=78 tabstop=4 shiftwidth=4 softtabstop=4
