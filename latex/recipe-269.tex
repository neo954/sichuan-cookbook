\begin{recipe}{清汤竹荪肝膏}

\ingredients

\ingredient{竹荪}{二钱}
\ingredient{瘦猪肉}{二两五}
\ingredient{鸡脯肉}{四两}
\ingredient{母鸡汤}{六杓}
\ingredient{鸡蛋(取白)}{二个}
\ingredient{绍酒}{六钱}
\ingredient{食盐}{少许}
\ingredient{胡椒}{少许}

\preparation

竹荪以淘米水发胀后,轻轻搓几下漂入清水内。

鸡肝洗净以刀背捶绒,将蛋白搅散掺入鸡肝内,并加入胡椒、食盐和鸡汤一杓,合拌调匀。
以丝罗筛滤渣取汁,连续滤三次,然后将肝汁盛在碗内,蒸约十分钟即凝结成膏。蒸时火
力要适度,火大则会起蜂窝眼,火小则引起沉淀,不能凝结。

鸡汤五杓装在小锑锅内,在微火上烧沸,将浮油打起不要。将瘦猪肉用刀背捶绒,以冷汤
调散倾入鸡汤内,并加进绍酒用杓搅转,俟汤再沸时用漏杓将沉渣捞起不要。再将鸡脯肉
照样进行,并随时打净浮油,即成清汤(清汤时火力不宜大)。

临吃前将清汤轻轻转入肝膏碗里,并将发胀漂净的竹荪切成寸节放入,蒸熟即成。

\features

汤清彻如镜,味鲜美可口。竹荪乃本省特产。

\end{recipe}

% vim: filetype=tex noautoindent nojoinspaces
% vim: fileencoding=utf-8 formatoptions+=m
% vim: textwidth=78 tabstop=4 shiftwidth=4 softtabstop=4
