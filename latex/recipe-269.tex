% BSD 3-Clause License
%
% Copyright (c) 2023 Quux System and Technology. All rights reserved.
%
% Redistribution and use in source and binary forms, with or without
% modification, are permitted provided that the following conditions are met:
%
% 1. Redistributions of source code must retain the above copyright notice, this
%    list of conditions and the following disclaimer.
%
% 2. Redistributions in binary form must reproduce the above copyright notice,
%    this list of conditions and the following disclaimer in the documentation
%    and/or other materials provided with the distribution.
%
% 3. Neither the name of the copyright holder nor the names of its
%    contributors may be used to endorse or promote products derived from
%    this software without specific prior written permission.
%
% THIS SOFTWARE IS PROVIDED BY THE COPYRIGHT HOLDERS AND CONTRIBUTORS "AS IS"
% AND ANY EXPRESS OR IMPLIED WARRANTIES, INCLUDING, BUT NOT LIMITED TO, THE
% IMPLIED WARRANTIES OF MERCHANTABILITY AND FITNESS FOR A PARTICULAR PURPOSE ARE
% DISCLAIMED. IN NO EVENT SHALL THE COPYRIGHT HOLDER OR CONTRIBUTORS BE LIABLE
% FOR ANY DIRECT, INDIRECT, INCIDENTAL, SPECIAL, EXEMPLARY, OR CONSEQUENTIAL
% DAMAGES (INCLUDING, BUT NOT LIMITED TO, PROCUREMENT OF SUBSTITUTE GOODS OR
% SERVICES; LOSS OF USE, DATA, OR PROFITS; OR BUSINESS INTERRUPTION) HOWEVER
% CAUSED AND ON ANY THEORY OF LIABILITY, WHETHER IN CONTRACT, STRICT LIABILITY,
% OR TORT (INCLUDING NEGLIGENCE OR OTHERWISE) ARISING IN ANY WAY OUT OF THE USE
% OF THIS SOFTWARE, EVEN IF ADVISED OF THE POSSIBILITY OF SUCH DAMAGE.
%
\begin{recipe}{清汤竹荪肝膏}

\ingredients

\ingredient{竹荪}{二钱}
\ingredient{瘦猪肉}{二两五}
\ingredient{鸡肝(以母鸡黄沙肝最好)}{四两}
\ingredient{鸡脯肉}{四两}
\ingredient{母鸡汤}{六杓}
\ingredient{鸡蛋(取白)}{二个}
\ingredient{绍酒}{六钱}
\ingredient{食盐}{少许}
\ingredient{胡椒}{少许}

\preparation

竹荪以淘米水发胀后,轻轻搓几下漂入清水内。

鸡肝洗净以刀背捶绒,将蛋白搅散掺入鸡肝内,并加入胡椒、食盐和鸡汤一杓,合拌调
匀。以丝箩筛滤渣取汁,连续滤三次,然后将肝汁盛在碗内,蒸约十分钟即凝结成膏。蒸
时火力要适度,火大则会起蜂窝眼,火小则引起沉淀,不能凝结。

鸡汤五杓装在小铝锅内,在微火上烧沸,将浮油打起不要。将瘦猪肉用刀背捶绒,以冷汤
调散倾入鸡汤内,并加进绍酒用杓搅转,俟汤再沸时用漏杓将沉渣捞起不要。再将鸡脯肉
照样进行,并随时打净浮油,即成清汤(清汤时火力不宜大)。

临吃前将清汤轻轻转入肝膏碗里,并将发胀漂净的竹荪切成寸节放入,蒸熟即成。

\features

汤清彻如镜,味鲜美可口。竹荪乃本省特产。

\end{recipe}

% vim: filetype=tex noautoindent nojoinspaces
% vim: fileencoding=utf-8 formatoptions+=m
% vim: textwidth=78 tabstop=4 shiftwidth=4 softtabstop=4
