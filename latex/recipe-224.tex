\begin{recipe}{双色豆腐淖}

(原鸳鸯豆腐淖)

\ingredients

\ingredient{豆腐}{六个}
\ingredient{化猪油}{四两}
\ingredient{水豆粉}{一两}
\ingredient{清汤}{四两}
\ingredient{鸡蛋}{四个}
\ingredient{胡椒面}{二分}
\ingredient{熟瘦火腿蒙}{五钱}
\ingredient{料酒}{五钱}
\ingredient{盐}{五分}
\ingredient{味精}{三分}
\ingredient{小白菜}{五两}

\cooking

\step 	豆腐下锅,加盐少许,微煮后捞起滴干,用丝箩过一 道,加捣细的盐和胡椒、味精、料酒、豆粉搅匀,平均分成两 个碗装起。取蛋清及蛋黄,分别放入两个碗内,再调勻。白 菜秧沮后浸冷,挤干宰细,熵后扞起待用。

\step 	猪油(一两五)在炒锅内烧至九成热,即倒入白色的 豆腐淖,炒散,扞入盘内的一边,撒上火腿蒙。同法再以其 余的猪油炒好黄色豆腐淖,扞入盘内的另一边,撒上白菜秧

末(青色〉即成。

\notes

色鲜可口,酒饭均宜。

\end{recipe}

% vim: filetype=tex noautoindent
% vim: fileencoding=utf-8
% vim: textwidth=78 tabstop=4 shiftwidth=4 softtabstop=4
