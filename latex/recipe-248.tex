\begin{recipe}{凤眼鸽蛋}

\ingredients

\ingredient{鸽蛋}{十二个}
\ingredient{土司}{一碎}
\ingredient{鸡晡肉}{一两五}
\ingredient{肥膘}{一两}
\ingredient{鸡蛋}{二个}
\ingredient{火腿}{一两}
\ingredient{丝瓜}{一根}
\ingredient{化猪油}{二斤4毛二两}
\ingredient{净生菜}{二两}
\ingredient{白糖}{五钱}
\ingredient{盐}{三分}
\ingredient{料酒}{三钱}
\ingredient{水豆粉}{五钱}

\preparation

\step 鸽蛋洗干净,用碗装起,盛入清水,淹过鸽蛋,上笼蒸二十分钟取下,用清水冷却
,轻轻将蛋壳剥去。细心地不要将鸽蛋剥滥,保持完整。用刀逢中切开,两头大小相等。
鸡蛋二个摊成蛋皮,连同丝瓜皮、火腿各修切成小瓜子片、鱼眼睛片,各七十二片待用。

\step 土司切成一寸二宽、一寸八长、一分五厚的旗子块二十四片。鸡脯、肥膘、蛋清等
搅成“鸡糁”。每片土司上先涂上一层鸡糁,约二分厚;再将半边鸽蛋嵌在中间,紧挨鸽蛋
的周围点缀小瓜子片和鱼眼睛片,增加美观,上笼气五分钟成半制成品取下。

\step 化猪油在旺火上烧至六成热时,即将蒸好的半制成品取出,鸽蛋向下,土司向上放
入油锅内炸;炸时用小铲细心翻动,呈金黄色时即涝起在盘中摆好,两端镶上生菜即成。

\features

形态美观,脆嫩可口。

\end{recipe}

% vim: filetype=tex noautoindent nojoinspaces
% vim: fileencoding=utf-8 formatoptions+=m
% vim: textwidth=78 tabstop=4 shiftwidth=4 softtabstop=4
