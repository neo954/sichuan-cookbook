\begin{recipe}{晾干肉}

\ingredients

\ingredient{猪瘦肉}{一斤}
\ingredient{化猪油}{一两五}
\ingredient{大头菜}{一两}
\ingredient{酱油}{五钱}
\ingredient{姜}{二钱}
\ingredient{料酒}{五钱}
\ingredient{葱}{三钱}
\ingredient{胡椒面}{二分}

\ingredient{大蒜}{一个一}
\ingredient{味精,}{二分}
\ingredient{菜油}{半斤耗二两}
\ingredient{清汤}{三两}
\ingredient{香油}{二钱}
\ingredient{水豆粉}{三钱}
\ingredient{醋}{三钱}
\ingredient{白糖}{三钱}

\cooking

\step 大头菜削皮,横切成极薄的半圆形片葱、蒜去皮, 蒜切片,葱切马耳朵待用。

选猪后腿净瘦肉一斤,不拘规格,用平刀片成极薄的 肉片,贴于筲箕背,将肉铺开晾起,肉面洒上酱油、胡椒面、 料酒等,待两三小时后水气已晾干。

\step 将菜油烧开,倒入干肉片炸黄,铲于墩子上,切成一 寸大小的姜糖块(即旗子块)。

\step 化猪油二两入锅烧热,即投入葱、姜、蒜、醋、白糖、 清汤、香油,勾二流芡成汁,最后将炕酥的肉片、大头菜倒 入,一烹即成。

\notes

此菜在六十年前常用于碟子,为群众所喜爱,味甜酸, 肉香酥,颇有风味。

\end{recipe}

% vim: filetype=tex noautoindent
% vim: fileencoding=utf-8
% vim: textwidth=78 tabstop=4 shiftwidth=4 softtabstop=4
