\begin{center}
\Large
编印说明
\end{center}

在伟大领袖毛主席“发展绢济 保障供给”的财政经 工作总方针的指引, 商业战线同 各条战线一,形势大, 越来越好 我们饮食服务工作 为了适应革命 生产大好形 势的不断发展 进一步继承和发扬祖国的焘调遗产 提高 食部门 的口咖质量 增加花色品种 使食焘调技术更好地 为广大工农兵服务 我们在有关部门的帮助和支持下 编印 了这 《四川菜谱》。 

在编 《四川菜谱》 的过程中 我们参照了过去累 资,同时也结合我们在实际操作中初步摸索到的一些经, 特别是经过无产阶级文化大革命运动 遵照毛主席关 “我 们必须承一切优秀的文学艺术遗 产 批判地吸收其中 ; 有益的东西 的教导 对菜谱的品种 名称 内等各个 面作必要的修改整理并有所取舍 这次主要参照 了过去 们编 写 《中 国 名菜谱第七韫 和搜 《重庆名菜谱 部分 菜脱 内容分肉食 鸡鸭 鱼虾 山珍海味 甜食 英、 其它七个大类 共三百一十二个品种 基上拳由浅入深, 由筒到繁的次序编排。 ˇ 

《川菜谱》 作为内 部资料编印 主要是作为我们培训 技术的教材用。 由 于我们的理论水平和技术水平都比较低 , 又缺少经验 这次编写 《四川菜谱》 一定会有不少鲜, 希望广大饮食服务战线的同志们 特别是有丰富实践经 的老师傅 给我们提出宝贵的意见 以便及时改进 为使饮 食焕技术更好为广大工农兵服务而共同助. 

\begin{flushright}
成都市饮食公司革命委员\\
技术培训班教研组集体整

一九七二年七月
\end{flushright}
