\begin{recipe}{炸斑指}

\ingredients

\ingredient{猪肥肠头}{三段(约二斤)}
\ingredient{葱白}{一两}
\ingredient{鸡蛋黄}{一个}
\ingredient{盐}{七钱}
\ingredient{生菜}{二两}
\ingredient{酱油}{三钱}
\ingredient{白糖}{四钱}
\ingredient{清汤}{二两}
\ingredient{甜酱}{三钱}
\ingredient{酷}{四钱}
\ingredient{白石凡(研细)}{二钱}
\ingredient{菜油}{一斤半耗一两}
\ingredient{姜(去皮)}{三钱五}
\ingredient{花椒}{十余粒}
\ingredient{料酒}{七钱}
\ingredient{水豆粉}{二钱五}
\ingredient{咖喱}{一分}
\ingredient{蒜}{四钱}
\ingredient{香油}{四钱}

\cooking

\step 选用体厚质佳的肥肠头三段,靠近肛门部份切去一圈不用,在清水中加入白矾末,
把肥肠放入,用力揉搓清洗一次,除去粘液涎水;另换清水加盐,同样揉搓清洗两次。即
把肠的里面(有油的一面)翻出来,用刀撕去油上粘的杂质脏物(注意不要撕掉油再继续
用水清洗。这样两面反复多洗几次,直到肥肠白净涩手,无臭味为止。最后仍把有油的一
面翻到内面去,洗净后放在开水锅内煮一刻钟捞出。再把肠两端用刀修去一分左右,使之
整齐。每段再分别切成两段(共六段)。用大蒸碗盛起,放入清水、盐、葱白三段、料
酒、花椒、姜等佐料,使水淹没过肥肠,上笼用旺火蒸三小时,至肥肠起着绉折为适合。
\step 葱白用刀切成一寸长段,两头切成细花翻起。甜酱用香油、白糖拌合调匀。大蒜去
皮,切成四分见方、半分厚的片。生菜洗净,用香油、醋少许、蛋黄、白糖、咖喱拌合
(炸斑指时才拌)。葱、姜、大蒜分别用刀切成细末,和香油、料酒、清汤、水豆粉、白
糖、醋、酱油等佐料,用碗盛好调匀,成为糖醋汁。
\step 菜油倒入炒锅内烧沸,将蒸好的肥肠取出(要炸时才取,葱、姜等不要),放在盘
内将水泌干,顺着锅边放入锅内,并立即用汤瓢将肠拨横过来,不使两头肠口向着人(为
肠内有水,刚一下锅要炸几下,溅起来的油容易烫伤人)。炸时要用汤瓢拨动肥肠,以便
炸匀。约五分钟,肠呈金红色,即把油泌去,楙上香油,颠两下即捞出。然后在案板上用
刀把肥肠切成四分厚的圆圈,堆在盘的中间,盘的两端分别放上拌好的生菜和葱酱蒜。再
在炸斑指的原锅内将糖醋汁煎热,盛在两个汤杯内,与斑指一同上桌。

\features

此菜色泽金红,皮酥里嫩,细软而香,吃时可佐以葱酱蒜,也可以蘸糖醋汁,还可以伴生
菜叶,都各有风味。因肥 肠成圆形,似射箭的斑指一样,故名。

\end{recipe}

% vim: filetype=tex noautoindent
% vim: fileencoding=utf-8 formatoptions+=m
% vim: textwidth=78 tabstop=4 shiftwidth=4 softtabstop=4
