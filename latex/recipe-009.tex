\begin{recipe}{芙蓉肉片}

\ingredients

\ingredient{猪肉}{半斤}
\ingredient{鸡蛋清}{三个}
\ingredient{面包粉}{四钱}
\ingredient{干豆粉}{一两}
\ingredient{味精}{二分}
\ingredient{葱}{三分}
\ingredient{姜}{二分}
\ingredient{蒜}{二分}
\ingredient{酱油}{二钱}
\ingredient{醋}{二钱}
\ingredient{白糖}{三钱}
\ingredient{胡椒面}{一分}
\ingredient{料酒}{五钱}
\ingredient{盐}{三分}
\ingredient{清汤}{二两}
\ingredient{化猪油}{半斤耗二两五}

\preparation

\step 鸡蛋清一个,与冷清汤同放碗中,用竹筷搅打均匀,上笼蒸熟为白芙蓉,愈嫩愈好
(如蒸过火成蜂窝状即老了)。另将鸡蛋清二个加干豆粉和成蛋清豆粉。葱切细花,姜、
蒜均切成碎米。

\step 猪肉选用夹心肉半斤(即前腿贴着扇形骨的肉,纤维无横顺之分,最嫩),原有厚
度约八分,先切成一寸二宽的块,再横着切成半分厚的片,即成为一寸二长、八分宽、约
二分厚的薄片,共切三十片。将肉片放入碗中,加胡椒面、盐、味精、料酒,一同拌匀渍
十分钟,使之入味。然后将蛋清豆粉倒入,翻搅合匀,使全部醮裹在肉片上。

\step 将锅用旺火烧热,使全锅温度均匀,放入猪油,即端离火口,左右上下转动,使油
均匀涂于锅内(以免贴肉片时焦糊粘锅)。然后将肉片伸展开,在其一面蘸上面包粉(将
面包揉搓成为粉)贴于锅上。待全部贴好后持锅于旺火上,左右来回移动烘烤,注意使受
热面均匀,不要烤糊。约烘一分钟,肉片着锅面呈金黄色,能随锅活动起来时,即将另锅
烧至冒大烟的猪油约四两(要先准备好)倒入,并把锅端起摆动一下,让油淌到贴在锅边
的肉片上约二秒钟(经沸油浪后,未着锅的一面肉,由半熟而烫熟,吃起来更嫩),泌去
油,摆入盘中。

\step 用旺火将锅烧至冒烟,放入猪油,加葱花、姜、蒜米稍煵一下,烹入料酒,随即放
入兑好的酱油、醋、白糖、水豆粉、味精、清汤,再用汤瓢推搅几下,烧沸后淋在盘中肉
片上。然后将白芙蓉从笼中取出,用调羹分五下舀在淋好汁的肉片上即成。

\features

此菜颜色黄白鲜明,味带甜酸,香酥脆嫩,别具风味。

\end{recipe}

% vim: filetype=tex noautoindent
% vim: fileencoding=utf-8 formatoptions+=m
% vim: textwidth=78 tabstop=4 shiftwidth=4 softtabstop=4
