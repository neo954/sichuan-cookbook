\begin{recipe}{荷包鱿鱼}

\ingredients

\ingredient{水发鱿鱼}{四两}
\ingredient{鸡脯肉}{三两}
\ingredient{猪肥膘}{二两}
\ingredient{鸡蛋清}{四个}
\ingredient{胡椒面}{二分}
\ingredient{干豆粉}{八钱}
\ingredient{熟瘦火腿}{一两}
\ingredient{盐}{六分}
\ingredient{料酒}{五钱}
\ingredient{丝瓜}{一根}
\ingredient{猪瘦肉}{三两}
\ingredient{清汤}{一斤半}
\ingredient{酱油}{三钱}
\ingredient{味精}{三分}

\preparation

\step 水发鱿鱼用中段,片成二寸长、二寸宽的片子二十四片;鸡脯肉、肥膘等打成“鸡
糁”;将蛋黄二个摊成蛋皮;丝瓜刮去粗皮,用中段切成三节,将皮车下泹熟,用冷水浸
透;连同火腿、蛋皮共三样各用一半,分别切成黑瓜子大小的斜方形小片,用盘装好;其
余各一半分别切成细丝,另用盘装好;瘦猪肉砸成茸子待用。

\step 片好的鱿鱼用好汤煨过,滤起用净布沾干水气,先在每片鱿鱼面上用蛋清豆粉抹上
一个半月形,再用“鸡糁”涂一层在蛋清豆粉面上,仍成半月形。每个涂好后,将切好的火
腿、丝瓜皮、蛋皮细丝小片等,先小片后细丝顺着半月形的糁一路路安成花边,成荷包
状。安好后将鱿鱼逐个拣在大瓷盘内,上笼蒸三分钟取出。

\step 走菜时可另用一蒸碗将鱿鱼拣入(糁向下)上笼气热,翻入大碗内。炒锅内掺清
汤、加酱油、盐、味精、胡椒等用肉茸清好,灌入即成。

\features

形态美观,汤清味鲜,适宜夏季席桌的清汤菜。

\end{recipe}

% vim: filetype=tex noautoindent nojoinspaces
% vim: fileencoding=utf-8 formatoptions+=m
% vim: textwidth=78 tabstop=4 shiftwidth=4 softtabstop=4
