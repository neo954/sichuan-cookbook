\begin{recipe}{盐白菜冬算}

\ingredients

\ingredient{冬笋}{一斤半}
\ingredient{盐}{二分}
\ingredient{味精}{二分}
\ingredient{盐白菜}{一两五}
\ingredient{料酒}{一钱}
\ingredient{鸡汤}{四两}
\ingredient{鸡油}{六钱}
\ingredient{化猪油}{四两耗一两三}
\ingredient{水豆粉}{一钱}

\cooking

\step 冬笋选用尖细、根小、身短、俗名“鹰嘴子”的为原料,削去笋根约五分长(老的〉
不用,以左手中、食指及大姆指捏着去根的冬笋,笋根向上,笋尖向下抵着菜墩,右手持
刀从根部起用刀尖顺着笋由上而下直划一刀,约一分深(以把笋壳划穿为度〉,到距笋尖
约五分长时即用力将刀向 右一拐按住笋壳,左手将笋向左一扭,全部笋壳即顺刀脱 下,
再用刀修去笋衣,使之光洁。笋根处若有顶刀的地方可 再去掉一些(保持嫩度〉。然后
切成一寸二分长、六分宽、 一分厚的薄片。

\step 炒锅放在旺火上放入猪油烧至七成热,将冬笋倒入,用汤瓢翻搅,稍煸一下,即将
猪油泌去,将笋装入盘中;盐白菜将水挤干,去掉花椒,用刀切成细末,用原猪油同样在
锅内炒一下;随即将盘中的笋子倒入,再加入鸡汤、料酒、盐、味精,用汤瓢搅匀,烧约
三分钟,即用水豆粉勾芡。然后顺着锅淋下鸡油,起锅即成。

\features

冬笋为冬季时菜之一,吃法很多,如加金钩、宣腿、酱 烧都可。与盐白菜同烧,鲜笋中
更透出一种盐白菜的芳香 别有风味。

①盐白菜为四川民间冬季家常菜。其作法是:卷心白菜一斤洗净, 当中划成四瓣,晾干水
份,待菜帮柔软时用盐三钱二抹透,用手 握花椒四十粒,揉搓出香味后撒在菜心中,用
干净青石压渍五天 即成.可以生拌或烧汤用.

\end{recipe}

% vim: filetype=tex noautoindent
% vim: fileencoding=utf-8 formatoptions+=m
% vim: textwidth=78 tabstop=4 shiftwidth=4 softtabstop=4
