\begin{recipe}{锅贴鸡片}

\ingredients

\ingredient{鸡脯肉}{四两}
\ingredient{猪肥瞟肉}{七两五}
\ingredient{料酒}{二线}
\ingredient{鸡蛋清}{二个}
\ingredient{干豆粉}{六钱}
\ingredient{熟瘦火腿}{三钱}
\ingredient{酱油}{二钱}
\ingredient{生菜}{三两}
\ingredient{香油}{三钱}
\ingredient{蕃芬酱}{三钱}
\ingredient{醋}{三钱}
\ingredient{白糖}{二两五}
\ingredient{化猪油}{一两五约耗四钱}
\ingredient{葱白}{六钱}
\ingredient{姜}{二钱}
\ingredient{甜酱}{三钱}

\cooking

\step 鸡脯肉用刀片成长一寸五、宽一寸二的薄片二十四张(越薄越好片好后用姜、葱段、料酒、酱油拌合渍起(约一刻钟猪肥膘肉在沸汤锅内煮约一小时(熟透微粑、捞起晾冷,在案板上用刀开成长一寸五、宽一寸二的长方形

肉块,再把刀放平片成厚一分的肉片二十四张,并用刀尖把 每片的四角及中心轻轻戳一个孔;瘦火腿先切成细丝,再横 着切成细末;鸡蛋清二个和干豆粉用碗调成蛋清豆粉。

\step 将肥膘肉二十四片平铺在盘内,用净布在热水中浸湿拧干后将肉片上的油沾干,将蛋清豆粉抹在每片肉的上面一层,并放上火腿末;渍起的鸡片去掉姜广葱,把每片展开,平放在火腿末上。

\step 炒锅在旺火上烧至七成热,放入猪油一浪,使锅内薄薄沾上一层油即泌去;将做好的鸡片逐个分开,猪肉向下贴在锅内,贴好后把锅置于微火上,并左右前后移动,以免煎糊;煎约八分钟,肉底呈金黄色时,将肉片化出的油用小铲轻轻向鸡片的周围浇淋,使鸡片逐渐变为浅黄色至熟;随后泌出多余的油,再淋上香油,将锅颠簸一下起锅,把鸡片堆在条盘的一端。

\step 生菜择用菜心及叶尖部分,淘洗干净,用白糖、醋、香油拌合〔生菜不能早拌,早拌要出水),镶在盘内的另一端;再将蕃茄酱淋在生菜上面;葱白切成一寸二长的段,甜酱用香油、白糖拌合调匀,分别放在盘的中部两边即成。

\notes

此菜颜色鲜美,入口脆嫩香酥,吃时蘸上一些葱酱,或 伴以生菜,更觉清香爽口。

\end{recipe}

% vim: filetype=tex noautoindent
% vim: fileencoding=utf-8
% vim: textwidth=78 tabstop=4 shiftwidth=4 softtabstop=4
