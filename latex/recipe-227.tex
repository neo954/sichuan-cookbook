\begin{recipe}[金钱豆腐]{白汁豆腐饼}

\ingredients

\ingredient{豆腐}{八个}
\ingredient{鸡蛋清}{三个}
\ingredient{面粉}{八钱}
\ingredient{干豆粉}{五钱}
\ingredient{味精}{三分}
\ingredient{盐}{三分}
\ingredient{丝瓜}{二两}
\ingredient{熟火腿}{二两}
\ingredient{化猪油}{二两五}
\ingredient{料酒}{一两}
\ingredient{清汤}{四两}
\ingredient{水豆粉}{三钱}
\ingredient{净白菜心}{四两}

\cooking

\step 豆腐去皮,水气滴干,用丝箩过细,加入鸡蛋清、面粉、 豆粉(过箩)、盐(二分)、料酒(五钱)、味精(二分)、 猪油〈一两五),在碗内用力搅匀,成“豆腐糁”待用。

\step 鸡蛋黄调匀,盛入碟内,上笼蒸熟,成蛋糕状。丝瓜 刮去祖茸,车下瓜皮,沮熟,在清水内漂过。分别用刀将蛋 糕、瓜皮改成二分宽旗子块。火腿切成二分见方的薄片。

\step 大平磁盘一个,先用猪油抹上,然后把搅匀的豆腐糁 用调羹舀成三十二份,在盘上拍压成直径一寸二、厚约三分 的爵形。再将火腿、,瓜皮、蛋糕等片嵌成金钱形,上笼蒸三

-分钟,成熟拣入盘内。用猪油将白菜心煸熟,加余下的盐、 味精、,料酒、清汤、水豆粉,勾成白汁,淋于菜上即成。

\notes

此菜形似金钱,质嫩味鲜。

\end{recipe}

% vim: filetype=tex noautoindent
% vim: fileencoding=utf-8
% vim: textwidth=78 tabstop=4 shiftwidth=4 softtabstop=4
