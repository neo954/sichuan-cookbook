\begin{recipe}{锅贴肚头}

\ingredients

\ingredient{肚头}{二个}
\ingredient{荸荠}{四个}
\ingredient{绍酒}{少许}
\ingredient{味精}{少许}
\ingredient{鸡蛋(用白)}{一个}
\ingredient{姜}{数小片}
\ingredient{肥漆肉}{二两五}
\ingredient{熟火腿}{六钱}
\ingredient{胡椒、食盐}{少许}
\ingredient{干豆粉}{四钱}
\ingredient{葱}{数小节}
\ingredient{净冬笋}{六钱}

\preparation

肚头去筋皮,切成骨牌般大、铜元般厚的块子,与食盐、绍酒、胡椒、味精、姜葱等一齐
拌和均匀。肥膘肉煮熟后切成与肚头一样大的块子,将荸荠、冬笋、火腿切成小片,再将
蛋白与豆粉拌成蛋清豆粉。

将肥膘肉抹干水气后,抹上蛋清豆粉。以荸荠、冬笋、火腿片摆在上面,又抹上蛋清豆粉,
再摆上一块肚头。锅烧辣后约下一汤瓢猪油,微烧后即倒去,留下少许猪油,移在小火上
煎炸上述片子。肚头朝上,肉片临锅,待肉煎成金黄色时,翻面将肚头那一面微煎,倒去
锅中所有的油,加进绍酒轻轻簸动,起锅即成。

\features

穌脆香美。

\end{recipe}

% vim: filetype=tex noautoindent nojoinspaces
% vim: fileencoding=utf-8 formatoptions+=m
% vim: textwidth=78 tabstop=4 shiftwidth=4 softtabstop=4
