\begin{recipe}{鸡淖脊髓}

\ingredients

\ingredient{猪脊髓}{三两五}
\ingredient{鸡脯肉}{二两五}
\ingredient{鸡蛋清}{四个}
\ingredient{干豆粉}{二钱五}
\ingredient{火腿}{五分}
\ingredient{化猪油}{二两五}
\ingredient{味精}{三分}
\ingredient{盐}{一钱五}
\ingredient{清汤}{三两五}
\ingredient{料酒}{三钱}

\cooking

\step 猪脊髓去筋,洗净;锅内放入沸水,加盐五分,然后将 脊髓放入;煮约十分钟,捞出稍晾,切为一寸长的段,装入碗中 以清水漂好待用。生鸡脯肉去筋,捶成茸。火腿切成细末。 豆粉以清水浸湿待用。

\step 清汤舀于碗内晾冷,将鸡茸放于汤内,用竹筷搅,再把 鸡蛋去壳,用蛋清连同浸湿的豆粉、盐、味精等一起加入鸡 茸内,搅拌均匀,成稀糊状。

\step 3少锅烧红,用布揩净锅底,把猪油入锅烧至九成热, 倾入调匀的鸡茸糊,翻搅炒熟即成鸡淖。将锅端离火口,将 鸡淖一半舀入盘中,其余一半留锅内待用。

I另用炒锅一口放在火上,加清汤、料酒烧沸;将漂好 的脊髓捞出滤尽水,入锅汆透,用汤瓢捞出,滤干水,舀入 剩余一半鸡淖锅中,同时加猪油在火上炒匀,倒入盘内鸡淖 上面;再把火腿末撒在鸡淖脊髓上面即成。

\notes

此菜味清淡,颜色洁白,下酒佐饭均宜。

\end{recipe}

% vim: filetype=tex noautoindent
% vim: fileencoding=utf-8
% vim: textwidth=78 tabstop=4 shiftwidth=4 softtabstop=4
