\begin{recipe}{鸡皮慈笋}

\ingredients

\ingredient{慈笋}[\footnotemark]{二斤}
\ingredient{味精}{三分}
\ingredient{白矾}{四钱}
\ingredient{化猪油}{一两三}
\ingredient{特级奶汤}{一斤}
\ingredient{葱}{二钱}
\ingredient{盐}{四分}
\ingredient{熟鸡皮}{一两三}
\ingredient{料酒}{三钱}
\ingredient{姜}{二钱}
\ingredient{鸡油}{三钱}
\ingredient{二汤}{一两三}

\preparation

\step 慈笋剥去外壳,用刀削去外表老茎,把前端较嫩的切成薄片,愈薄愈好。锅内加清
水一碗和白矾,烧沸放入慈笋片汆约驾钟,除去苦味,汆后漂于清水中(如吃笋的原味,
则不用白矾水汆)。白色熟鸡皮切成一寸大的菱形块,加料酒、二汤,入笼蒸二十分钟待
用。

\step 猪油入锅内烧热,放入姜、葱,加奶汤烧开后,拣去姜、葱,放入笋片,约烧五分
钟,加味精,再将蒸好的鸡皮放入,随后加鸡油,盛入汤盘即成。

\features

此菜清淡味鲜,宜于夏季吃用。

\footnotetext{
慈笋:产于四川盆地,长约数丈,夏季茂盛,称为慈竹。初生时出土约五寸,用以做菜,
味苦而鲜,一般称为慈竹笋。
}

\end{recipe}

% vim: filetype=tex noautoindent nojoinspaces
% vim: fileencoding=utf-8 formatoptions+=m
% vim: textwidth=78 tabstop=4 shiftwidth=4 softtabstop=4
