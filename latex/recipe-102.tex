\begin{recipe}{茉芋烧鸭}

\ingredients

\ingredient{仔鸭一只}{约二斤半}
\ingredient{酱油}{一两三}

\ingredient{水笨字}{一斤半}
\ingredient{盐}{六分}
\ingredient{豆瓣}{一两五}
\ingredient{料酒}{一两五}
\ingredient{花椒}{约十粒}
\ingredient{味精}{二分}
\ingredient{嫩姜片}{五钱}
\ingredient{胡椒面}{…分}
\ingredient{蒜片}{三钱}
\ingredient{水豆粉}{三钱}

\ingredient{蒜苗段(一寸艮)}{一两}
\ingredient{混合油}{三两}
\ingredient{清汤}{一斤}

\cooking

\step 将水茉芋切成长一寸六分、宽、厚各四分的长方条,用沸水在旺火上汆两次,去其石灰水味,然后漂入温水中。将仔鸭宰杀,退毛,剖腹,去脏,洗净,剔骨,头颈、翅尖和脚掌砍去不用,再砍成长一寸半、宽五分的长条块。

\step 将锅置于旺火上烧热,放入混合油,烧至冒烟时倒入鸭条,煸除水份,至鸭条现黄色斑点即铲起来。余油留锅中,放入花椒、豆瓣熵酥后,加入清汤,用漏瓢将花椒、豆瓣渣漏去。再将鸭条放入,加姜、蒜片、料酒、盐、酱油等,盖好。烧二十分钟左右,至鸭条已达七成炬时,将茉芋用筲箕捞出,滤水入锅,再加胡椒面搅勻盖好。继续烧二十分钟,至鸭条已粑,剩汁约三两时加入蒜苗段焖一下;随即放入味精,并将水豆粉加清水,调匀后倒入,勾芡起锅即成。

\notes

此菜滋汁红亮,味浓而鲜

\end{recipe}

% vim: filetype=tex noautoindent
% vim: fileencoding=utf-8
% vim: textwidth=78 tabstop=4 shiftwidth=4 softtabstop=4
