\begin{recipe}{红烧环喉}

\ingredients

\ingredient{环喉}{十五根}
\ingredient{水发兰片}{五钱}
\ingredient{化猪油}{半斤耗二两}
\ingredient{水豆粉}{三钱}
\ingredient{味精}{三分}
\ingredient{盐}{五分}
\ingredient{料酒}{一钱}
\ingredient{清汤}{四两}
\ingredient{胡椒面}{二分}
\ingredient{熟瘦火腿}{五钱}
\ingredient{酱油}{五钱}
\ingredient{鸡油}{五钱}
\ingredient{水发鸡松}{一两}

\cooking

\step 将环喉撕去油筋,再将内层翻出,放于墩子上,用刀尖侬次剗成蜈蚣的爪形,切成
一寸五的段,放入碗内,加淸汤入笼,大火蒸一小时半。又将火腿、兰片、鸡松切成一寸
二长、五分宽的片,放于盘内待用。
\step 将锅放于旺火上,倒入化油,烧至六成火候,将葱、姜投入油锅,稍炸,打去不
用,再投入火腿、鸡松、兰片,又将笼内的环喉取出投入锅内,加入料酒、胡椒、味精、
酱油、盐,炒动数转,再放入清汤,烧三分钟,勾入水豆粉起锅, 淋入鸡油,再入盘内即
成。

\notes

脆嫩可口,适宜秋夏。

\end{recipe}

% vim: filetype=tex noautoindent
% vim: fileencoding=utf-8
% vim: textwidth=78 tabstop=4 shiftwidth=4 softtabstop=4
