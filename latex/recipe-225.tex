\begin{recipe}{韭汁豆蕊}

\ingredients

\ingredient{韭菜叶}{一斤}
\ingredient{豆腐}{八1}
\ingredient{猪肥膘}{二两}
\ingredient{鸡蛋}{三个}
\ingredient{金钩}{五钱}
\ingredient{熟肥瘦火腿}{一两五}
\ingredient{水发口笨}{一两}
\ingredient{味精}{二分}
\ingredient{胡椒}{一分}
\ingredient{盐}{三分}
\ingredient{干豆粉}{一两}
\ingredient{面粉}{五钱.}
\ingredient{清汤}{八两}
\ingredient{化猪油}{二两}
\ingredient{香油}{三钱}

\preparation

\step 韭菜叶淘净加盐,揉搓挤出汁水,用纱布滤净,盛入碗:内。用丝箩把捏散的豆腐
过一道(先入热水内除去砠水味)。蛋清快速调成蛋泡。肥膘肉剁茸不见籽粒。火腿、金
钩、口茉均切成细颗待用。

\step 将干豆粉、盐、面粉、味精、胡椒,在碗内和匀。加蛋清、肥膘茸、豆腐茸,用力
向着一个方向搅匀成“豆腐糁”。另用二鱼碗一个先抹一点猪肥膘茸,糊上豆腐糁一层约二
分厚,再用豆腐糁将火腿、金钩、口茉细颗混合调匀、放入中心,上面放入其余豆腐糁,
上笼蒸三十分钟取出,翻入大圆盘。

\step 炒锅内将猪油烧热,倒入清汤、韭菜汁,加味精、盐烧沸,勾芡起锅搭香油淋上豆
腐糁即成。

\features

菜色白润,汤汁翠绿,清香可口,营养丰富9老年人多 喜爱。

\end{recipe}

% vim: filetype=tex noautoindent nojoinspaces
% vim: fileencoding=utf-8 formatoptions+=m
% vim: textwidth=78 tabstop=4 shiftwidth=4 softtabstop=4
