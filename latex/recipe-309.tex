\begin{recipe}{清炖牛肉汤}

\ingredients

\ingredient{黄牛肉}{二十五斤}
\ingredient{老姜(拍破)}{丰斤}
\ingredient{绍酒}{半斤}
\ingredient{味精}{二两五}
\ingredient{上等花椒}{一两六}
\ingredient{净鲜肥母鸡}{六斤}
\ingredient{食盆(炒后磨细)}{酌量}

\preparation

水牛肉质较粗,草气重,鲜味差,所以烛牛肉必须选用黄牛肉,而黄牛肉适宜于清炖的又
以后腿腿骨筋、千斤头、肋占、筋管、前腿腿骨筋较好。不要用净瘦肉。

将挑选好的黄牛肉用冷水漂二十分钟,切成约重二斤半的块子,放入特制的圆形白铁烛桶
(高一尺三寸、直径七寸五分)内。须分次放入,以便于分次除去泡沫。第一次放进十五
斤,同时下入沸水二十斤,在旺火上烧开后,撇去泡沫,并约隔半小时从桶底翻动牛肉,
以免粘锅。随即将其余十斤半牛肉放下,俟烧开,又打去泡沫,加进老姜、花椒、绍酒、
母鸡(去头及爪),再行烧开,打去泡沫,然后移微火上炖着。

炖桶移在微火上后,应经常保持使汤微开的火力。肉炖到五成熟时,将上下面的肉相互移
位,使接受热力均匀。到七成熟时,取出牛肉按纤维横切成一寸长、食指祖的一字条,拣
去不合格的皮筋,并按炤硬不同的程度分成三类,分装三个炖桶中。同时,用干净稀白布
滤去汤中老姜、花椒,也将汤分盛入三个炖桶内,在旺火上烧开后,置微火上炖圯为止。
前后共需五小时半。母鸡只取其汁,炖好后取出作别用。

吃时可配以蔬菜,冬春天用萝卜,夏秋天用冬瓜或瓠子瓜,切成一寸长、食指粗的一字
条,另用锅煮熟,吃时加在碗中,并酌加味精、食盐。

\features

汤色清彻,汤味鲜香,肉质细嫩,油而不腻。

\end{recipe}

% vim: filetype=tex noautoindent nojoinspaces
% vim: fileencoding=utf-8 formatoptions+=m
% vim: textwidth=78 tabstop=4 shiftwidth=4 softtabstop=4
