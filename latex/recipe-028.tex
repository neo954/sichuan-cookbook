\begin{recipe}{焦皮肘子}

\ingredients

\ingredient{净肘子}{一个(约二斤半)}
\ingredient{化猪油}{二两}
\ingredient{冰糖}{二两}
\ingredient{料酒}{二两}
\ingredient{酱油}{二两}
\ingredient{姜}{五钱}
\ingredient{葱}{五钱}
\ingredient{盐}{三分}
\ingredient{水豆粉}{二钱}
\ingredient{清汤}{三斤}
\ingredient{碎骨}{一斤}

\preparation

\step 将肘子残毛拈去,放在炭火上烧,至皮呈焦黑时,丢入热水内泡半点钟,至皮已软
后,取出,用刀全部刮去烧黑的一层,现出黄色时,入清水清洗两次待用。

\step 鼎锅置于旺火上,放入碎骨垫底,加汤,再投入肘子。汤一冲开,打净血泡,即端
离旺火放在小火上焙起。

\step 油在炒锅内用庇火烧热,下冰糖炒成汁,舀入少许鼎锅内的汤把糖汁冲散。加料
酒、酱油、葱(挽成结)、姜(拍破)、盐,用汤瓢荡匀翻入鼎锅。继续焙两小时,用筷
子将肘子拈入大圆盘摆好。滋汁泌入炒锅内,加味精,勾芡,收酽后淋在肘子上即成。

\features

颜色红亮,味浓香美,肥而不腻。

\end{recipe}

% vim: filetype=tex noautoindent nojoinspaces
% vim: fileencoding=utf-8 formatoptions+=m
% vim: textwidth=78 tabstop=4 shiftwidth=4 softtabstop=4
