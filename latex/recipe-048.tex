\begin{recipe}[菠饺银肺]{菠饺白肺}

\ingredients

\ingredient{猪肺一副}{约二斤}
\ingredient{盐}{一两五}
\ingredient{姜}{二分}
\ingredient{面粉}{二两五}
\ingredient{酱油}{一钱五}
\ingredient{化猪油}{一两}
\ingredient{猪肉(肥瘦)}{二两五}
\ingredient{清汤}{二斤}
\ingredient{鸡油}{一钱}
\ingredient{熟火腿}{六钱}
\ingredient{菠菜}{二斤}
\ingredient{味精}{二分}
\ingredient{水发口蘑}{六钱}
\ingredient{料酒}{五钱}
\ingredient{香油}{一钱}
\ingredient{熟鸡皮}{六钱}
\ingredient{葱白}{三钱}
\ingredient{胡椒面}{二分}

\preparation

\step 选择较白净无破洞的猪肺一副(可先检查,看是否有漏气的地方),用绳将肺把子
拴住,肺叶向下挂起,用大铁壶盛清水(有自来水的就用自来水笼头)慢慢由肺上端管口
灌入冲洗,待肺内水满膨胀,即用双手轻轻捧起肺叶向上倾倒一次血水。灌时若遇血块阻
塞停滞,可用一比肺管口略细、长约七寸的竹管插入肺管内用手轻拍,使之畅通再、灌
(但本倉用力过大,不然肺易破裂,裂后就不能洗白了)。这样反复灌洗,到肺内红色血
水全部随水漫出,猪肺全部变为白色为止。冲洗净后,把肺叶向上肺管向下用竹筲箕装
起,将水滤干,再照原来方法把肺挂起,滤净水,入沸水锅中煮熟,切成长一寸二分、宽
一寸、厚半分的片。

\step 猪肉去筋,在菜墩上用刀剁成细泥,加入香油、盐、冷清汤在碗内拌合成馅。火
腿、鸡皮切成一寸二分长、八分宽的片待用。

\step 玻菜淘洗予净,去茎留叶,在木瓢内用手搓揉成菜泥,用纱布包好挤出绿色菜汁,
和入干面粉内拌匀(汁不够时可略加清水用力揉成面团;再平均扯成二十四个小面剂,角
小面棍擀成圆形饺皮。把拌好的馅分摊在饺皮上对折包成半圆形,狡边甩手捏成细折,然
后用清水把饺子煮好,端离火口,用时再捞出。

\step 猪油放在炒锅内,于旺火上烧热,放入姜、葱白、料酒和清汤,开锅后用漏瓢将
姜、葱捞出不用,放入盐、胡椒面、味精,同时捞起锅内的水饺倒入汤锅内,再加入口
蘑、火腿、鸡皮、白肺片,至汤再开后淋入鸡油倒入大菜盘内即成。

\features

此菜用料平常,但烹制精细,不仅汤内饺子碧绿,肺片银白,两色相间,颜色美观,而且
菜的质地亦非常细嫩鲜美。“菠饺”在川菜中应用很广。如“菠铰海参”、“菠饺鱿鱼”
等等,均可如法炮制。

\end{recipe}

% vim: filetype=tex noautoindent nojoinspaces
% vim: fileencoding=utf-8 formatoptions+=m
% vim: textwidth=78 tabstop=4 shiftwidth=4 softtabstop=4
