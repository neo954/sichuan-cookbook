\begin{recipe}{烟熏子鸡}

\ingredients

\ingredient{白皮肥嫩母鸡}{一只}
\ingredient{茶叶}{少许}
\ingredient{食盐}{一两}
\ingredient{红酱油}{三钱}
\ingredient{葱}{三根}
\ingredient{小磨麻油}{三钱}
\ingredient{生姜}{一块}
\ingredient{花株}{约二十颗}
\ingredient{耢糟水}{六钱}

\cooking

鸡去头脚,洗干净,晾干水气,里外均匀地抹上食盐,

胸部因肉厚要适当多抹。隔约半天,把姜拍碎,葱子打成小结子,连同花椒放在鸡腹内,鸡的皮面以耢糟水抹上,用碗盛起,上笼蒸约二小时,蒸到软和时为度。在蒸的过程中,每隔二三十分钟须刷麻油一次。鸡蒸好,取出时将红酱油刷在皮面上,使颜色美观,然后用鲜柏树枝和少许茶叶燃起的烟子熏约半小时去掉腹内的姜、葱、花椒等,砍成约八分长的斜方形小块,按鸡形盛于盘内即成。

\notes

色泽金黄,味清香,鲜嫩。

\end{recipe}

% vim: filetype=tex noautoindent
% vim: fileencoding=utf-8
% vim: textwidth=78 tabstop=4 shiftwidth=4 softtabstop=4
