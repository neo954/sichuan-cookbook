\begin{recipe}{锅巴肉片}

\ingredients

\ingredient{大米锅巴}{五两}
\ingredient{水发青菌}{三钱}
\ingredient{猪腰柳肉}{三两八}
\ingredient{葱白}{二钱五}
\ingredient{菜油}{一斤半约耗一两八}
\ingredient{姜}{一钱五}
\ingredient{蒜}{一钱五}
\ingredient{水发玉兰片}{一两}
\ingredient{化猪油}{一两二}
\ingredient{水豆粉}{四钱}
\ingredient{酱油}{三钱}
\ingredient{清汤}{二两八}
\ingredient{白糖}{三钱}
\ingredient{醋}{三钱}
\ingredient{盐}{一分}

\preparation

\step 腰柳肉(背脊下面近猪腰处的肉)用刀片去白筋,横着肉纹切成长一寸半、宽八分
的薄片(愈薄愈好);用水豆粉、料酒、盐在碗内将肉片搅匀。玉兰片、青菌用刀片成长
一寸二、宽八分的薄片。姜、蒜去皮,切成三分见方、半分厚的片。葱白斜切成六分长的
段。清汤、白糖、醋、料酒、水豆粉、酱油、盐、味精等佐料在碗内调好。

\step 猪油放入炒锅内在旺火上烧至八成热,将肉片倒入拨散,即加入姜、葱、蒜片及玉
兰片、青菌等,翻搅一下,随将碗内调好的佐料倒入,用汤瓢搅匀,即炒成有汤的肉片。
盛入碗内待用。

\step 锅巴取不厚不糊的,用手掰成长宽约两寸的块。菜油放入锅内在旺火上烧沸,将锅
巴倒入,炸至呈金黄色时捞入盘中。盘底留沸油约六钱(淋肉片时发出声音才大),然后
将炒好的有汤肉片同时端出,待锅巴在桌上放定后,始将肉片往锅巴上淋下,这时“哗嚓”
一声,盘中立即喷起一种浓烈的糖醋香味。

\features

此菜肉片鲜嫩,锅巴酥脆,味甜酸而香,趁热取食,其味尤美。因淋肉片时,要在堂面上
发出响声,故又名“堂响肉片”。

\end{recipe}

% vim: filetype=tex noautoindent nojoinspaces
% vim: fileencoding=utf-8 formatoptions+=m
% vim: textwidth=78 tabstop=4 shiftwidth=4 softtabstop=4
