\begin{recipe}{清汤白菜}

(原名开水白菜)

\ingredients

\ingredient{黄秧白菜心}{一斤半}
\ingredient{胡椒面}{二分}
\ingredient{料酒}{二钱五}
\ingredient{盐}{一}
\ingredient{特级清汤}{二斤}

\cooking

\step 	将黄秧白菜四棵去净边叶,得心一斤半左右。每棵长 四寸半,四棵长短一致,大小均勻,每棵切成对半。然后用 清水洗净,漂入清水中待用。

\step 	炒锅放于大旺火炉上,倒入开水四斤,再放入白菜心 煮。煮时不要盖锅,否则白菜便要变色。煮至八分火时捞出, 用清水过二、三次,务使菜心冷透。然后将白菜有次序地摆 入蒸碗中,上加胡椒面、料酒、盐和特级清汤,入笼用大旺 火蒸四分钟。

\step 炒锅放炉上,倒入特级清汤,加胡椒面、料酒、盐烧 开。同时将蒸笼内的白菜出笼,泌去蒸碗中的汤不用,将下 余特级清汤均匀地淋在白菜上,务使全部淋透,再将清汤泌 去不用,将白菜翻入大汤碗中,最后将锅中加好味的特级清 汤舀入汤碗中上席。

\notes

此菜味道鲜美,颜色与生鲜菜无异,看来如同一碗开水 内放着几棵生白菜,故有“开水白菜”之称。

①八分火即指白菜将把未把的时候,

\end{recipe}

% vim: filetype=tex noautoindent
% vim: fileencoding=utf-8
% vim: textwidth=78 tabstop=4 shiftwidth=4 softtabstop=4
