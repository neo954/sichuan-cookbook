\begin{recipe}[名开水白菜]{清汤白菜}

\ingredients

\ingredient{黄秧白菜心}{一斤半}
\ingredient{胡椒面}{二分}
\ingredient{料酒}{二钱五}
\ingredient{盐}{一}
\ingredient{特级清汤}{二斤}

\preparation

\step 将黄秧白菜四棵去净边叶,得心一斤半左右。每棵长四寸半,四棵长短一致,大小
均匀,每棵切成对半。然后用清水洗净,漂入清水中待用。

\step 炒锅放于大旺火炉上,倒入开水四斤,再放入白菜心煮。煮时不要盖锅,否则白菜
便要变色。煮至八分火时捞出,用清水过二、三次,务使菜心冷透。然后将白菜有次序地
摆入蒸碗中,上加胡椒面、料酒、盐和特级清汤,入笼用大旺火蒸四分钟。

\step 炒锅放炉上,倒入特级清汤,加胡椒面、料酒、盐烧开。同时将蒸笼内的白菜出笼
,泌去蒸碗中的汤不用,将下余特级清汤均匀地淋在白菜上,务使全部淋透,再将清汤泌
去不用,将白菜翻入大汤碗中,最后将锅中加好味的特级清汤舀入汤碗中上席。

\features

此菜味道鲜美,颜色与生鲜菜无异,看来如同一碗开水 内放着几棵生白菜,故有“开水白
菜”之称。

①八分火即指白菜将把未把的时候,

\end{recipe}

% vim: filetype=tex noautoindent nojoinspaces
% vim: fileencoding=utf-8 formatoptions+=m
% vim: textwidth=78 tabstop=4 shiftwidth=4 softtabstop=4
