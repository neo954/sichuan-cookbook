\begin{recipe}{鹅黄肉}

\ingredients

\ingredient{肥瘦肉}{半斤}
\ingredient{料酒}{五钱}
\ingredient{葱花}{二钱}
\ingredient{鸡蛋}{五个}
\ingredient{胡椒面}{二分}
\ingredient{鱼辣椒}{五钱}
\ingredient{干豆粉}{一两}
\ingredient{菜油}{二斤耗二两}
\ingredient{醋}{五钱}
\ingredient{酱油}{三钱}
\ingredient{味精}{二分}
\ingredient{白糖}{三钱}
\ingredient{盐}{二钱}
\ingredient{姜米}{二钱}
\ingredient{蒜米}{二钱}

\preparation

\step 鸡蛋三个调匀,先在锅内摊成蛋皮二张。肥瘦肉剁成细末,加料酒、葱花、姜米、
酱油、盐、味精、豆粉、胡椒面、鸡蛋一个,共拌匀成馅。

\step 蛋皮铺开全抹上蛋清豆粉,将拌好的馅裹成六条蛋卷,用手拍成宽六分、厚二分的
扁形卷,用刀在卷的半面切成丝,半面不切(如切蛰卷样),然后再按一寸距离切为三十
段。

\step 菜油烧至八成火,将切过的蛋卷入油锅内炸熟,捞起盛入盘内。

\step 锅内留适当的油,再放鱼香味作料,勾成二流芡滋汁,淋上即成。

\features

为肉制品中之多过程操作法,皮酥馅嫩,色鲜味浓,过 去常走于海参便饭行菜。

\end{recipe}

% vim: filetype=tex noautoindent nojoinspaces
% vim: fileencoding=utf-8 formatoptions+=m
% vim: textwidth=78 tabstop=4 shiftwidth=4 softtabstop=4
