\begin{recipe}{芙蓉杂烩}

\ingredients

\ingredient{酥肉}{二两五}
\ingredient{熟猪肚}{二两}
\ingredient{熟猪舌}{一两五}
\ingredient{熟火腿}{一两}
\ingredient{尖刀丸子}{一两五}
\ingredient{水发响皮}{二两}
\ingredient{水发笋子}{二两}
\ingredient{水发鸡松}{一两}
\ingredient{鸡蛋}{两个}
\ingredient{姜葱}{各五钱}
\ingredient{料酒}{五分}
\ingredient{盐}{三分}
\ingredient{味精}{二分}
\ingredient{胡椒}{三分}

\preparation

\step 炸酥、放响、川笋子、抠鸡松、剁丸子馅。

\step 酥肉、肚、舌、火腿、响皮、笋子、鸡松,分别切、片成一寸五长、三至四分宽条
片,丸子刮成尖刀。

\step 碗底用火腿、肚、舌各一片摆好,两头各摆鸡松一片,碗周围按酥肉、火腿、肚、
舌、鸡松各一片间隔镶满,再按先后放丸子、响皮、笋子分层垫底加姜葱、料酒上笼;芙
蓉蛋单蒸。走菜时吃味、灌汤,芙蓉蛋舀在周围。

\features

内容多样,整齐美观,色明汤清,一般作吃饭菜。

\end{recipe}

% vim: filetype=tex noautoindent nojoinspaces
% vim: fileencoding=utf-8 formatoptions+=m
% vim: textwidth=78 tabstop=4 shiftwidth=4 softtabstop=4
