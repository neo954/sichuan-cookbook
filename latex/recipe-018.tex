\begin{recipe}{芙蓉杂烩}

\ingredients

\ingredient{酥肉}{二两五}
\ingredient{熟猪肚}{二两}
\ingredient{熟偖舌}{一11两五}
\ingredient{熟火腿}{一两}
\ingredient{尖刀元子}{一一两五}
\ingredient{水发响皮}{二两}
\ingredient{水发笋子}{二两}
\ingredient{水发鸡松}{~一两}
\ingredient{鸡蛋}{两个}
\ingredient{姜葱}{各五钱}
\ingredient{料酒}{五分}
\ingredient{盐}{三分}
\ingredient{味精}{二分}
\ingredient{胡椒}{三分}

\cooking

\step 	炸酥、放响、川笋子、抠鸡松、剁元子馅。

\step 	酥肉、肚、舌、火腿、响皮、笋子、鸡松,分别切、片 成一寸五长、三至四分宽条片,元子刮成尖刀。

\step 碗底用火腿、肚、舌各一片摆好,两头各摆鸡松一 片,碗周围按酥肉、火腿、肚、舌、鸡松各一,间隔镶满, 再按先后放元子、响皮、笋子分层垫底加姜葱1料酒上笼; 芙蓉蛋单蒸。走菜时吃味、灌汤,芙蓉蛋园在周围。

\notes

内容多样,整齐美观,色明汤清,一般作吃饭菜。

\end{recipe}

% vim: filetype=tex noautoindent
% vim: fileencoding=utf-8
% vim: textwidth=78 tabstop=4 shiftwidth=4 softtabstop=4
