% BSD 3-Clause License
%
% Copyright (c) 2023 Quux System and Technology. All rights reserved.
%
% Redistribution and use in source and binary forms, with or without
% modification, are permitted provided that the following conditions are met:
%
% 1. Redistributions of source code must retain the above copyright notice, this
%    list of conditions and the following disclaimer.
%
% 2. Redistributions in binary form must reproduce the above copyright notice,
%    this list of conditions and the following disclaimer in the documentation
%    and/or other materials provided with the distribution.
%
% 3. Neither the name of the copyright holder nor the names of its
%    contributors may be used to endorse or promote products derived from
%    this software without specific prior written permission.
%
% THIS SOFTWARE IS PROVIDED BY THE COPYRIGHT HOLDERS AND CONTRIBUTORS "AS IS"
% AND ANY EXPRESS OR IMPLIED WARRANTIES, INCLUDING, BUT NOT LIMITED TO, THE
% IMPLIED WARRANTIES OF MERCHANTABILITY AND FITNESS FOR A PARTICULAR PURPOSE ARE
% DISCLAIMED. IN NO EVENT SHALL THE COPYRIGHT HOLDER OR CONTRIBUTORS BE LIABLE
% FOR ANY DIRECT, INDIRECT, INCIDENTAL, SPECIAL, EXEMPLARY, OR CONSEQUENTIAL
% DAMAGES (INCLUDING, BUT NOT LIMITED TO, PROCUREMENT OF SUBSTITUTE GOODS OR
% SERVICES; LOSS OF USE, DATA, OR PROFITS; OR BUSINESS INTERRUPTION) HOWEVER
% CAUSED AND ON ANY THEORY OF LIABILITY, WHETHER IN CONTRACT, STRICT LIABILITY,
% OR TORT (INCLUDING NEGLIGENCE OR OTHERWISE) ARISING IN ANY WAY OUT OF THE USE
% OF THIS SOFTWARE, EVEN IF ADVISED OF THE POSSIBILITY OF SUCH DAMAGE.
%
\begin{recipe}{芙蓉杂烩}

\ingredients

\ingredient{酥肉}{二两五}
\ingredient{熟猪肚}{二两}
\ingredient{熟猪舌}{一两五}
\ingredient{熟火腿}{一两}
\ingredient{尖刀圆子}{一两五}
\ingredient{水发响皮}{二两}
\ingredient{水发笋子}{二两}
\ingredient{水发鸡松}{一两}
\ingredient{鸡蛋}{两个}
\ingredient{姜、葱}{各五钱}
\ingredient{料酒}{五分}
\ingredient{盐}{三分}
\ingredient{味精}{二分}
\ingredient{胡椒}{三分}

\preparation

\step 炸酥、放响、汆笋子、抠鸡松、剁圆子馅。

\step 酥肉、肚、舌、火腿、响皮、笋子、鸡松,分别切、片成一寸五长、三至四分宽条
片,圆子刮成尖刀。

\step 碗底用火腿、肚、舌各一片摆好,两头各摆鸡松一片,碗周围按酥肉、火腿、肚、
舌、鸡松各一片间隔镶满,再按先后放圆子、响皮、笋子分层垫底加姜葱、料酒上笼;芙
蓉蛋单蒸。走菜时吃味、灌汤,芙蓉蛋舀在周围。

\features

内容多样,整齐美观,色明汤清,一般作吃饭菜。

\end{recipe}

% vim: filetype=tex noautoindent nojoinspaces
% vim: fileencoding=utf-8 formatoptions+=m
% vim: textwidth=78 tabstop=4 shiftwidth=4 softtabstop=4
