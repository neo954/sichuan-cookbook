\begin{recipe}{蹄燕鸽蛋}

\ingredients

\ingredient{千大白蹄筋}{十六根}
\ingredient{鸽蛋}{十二个}
\ingredient{鸡汤}{一斤半}
\ingredient{胡椒}{二分}
\ingredient{味精}{五分}
\ingredient{料酒}{三钱}
\ingredient{鸡脯茸子}{四两}
\ingredient{化猪油}{一斤耗二两}
\ingredient{熟瘦火腿}{三钱}

\cooking

\step 将蹄筋用清水洗净、晾干。将锅放于炉上,放上猪油,将蹄筋放入锅内,用温油浸炸,有八成泡性时捞起晾冷;再用清水冰起泡把3又用七、八成温水泡起,泌去油水;再甩温水泡三、四次,蹄筋的油已除尽即可。将每根蹄筋对破,边子另用,每半边片成二片,同样将十六根蹄筋片完〔但薄的就不片);片后放在冷水内然后捞起用刀尖剗不能将蹄筋劁断再放入碗内用开水过两次;泌去;再放下开水,并放入草碱四分,用盘子盖住碗发上十五分钟;蹄筋粑了再用热水换上两次,又用冷水冰起。

\step 将丨1%蛋煮熟,去壳,放入一"个盘内装好;再将火腿切成丝.共切十几条即可3

\step 将蹄筋碗内的冷水泌去,再用开水烫后泌去,用二汤过一次;并将鸽蛋烫热,将蹄筋放入大海碗内,鸽蛋镶于周围。

\step 将锅放于旺火上,倒入好鸡汤,烧至微开,再将鸡脑茸用清水解散倒入锅内,肉沫浮起,打尽杂质,再放下味精、胡椒、料酒,烧开后,倒入装蹄筋的大海碗内即成。

\notes

清爽、鲜美、四季可口。

\end{recipe}

% vim: filetype=tex noautoindent
% vim: fileencoding=utf-8
% vim: textwidth=78 tabstop=4 shiftwidth=4 softtabstop=4
