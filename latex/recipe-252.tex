\begin{recipe}{蹄燕鸽蛋}

\ingredients

\ingredient{千大白蹄筋}{十六根}
\ingredient{鸽蛋}{十二个}
\ingredient{鸡汤}{一斤半}
\ingredient{胡椒}{二分}
\ingredient{味精}{五分}
\ingredient{料酒}{三钱}
\ingredient{鸡脯茸子}{四两}
\ingredient{化猪油}{一斤耗二两}
\ingredient{熟瘦火腿}{三钱}

\cooking

\step 将蹄筋用清水洗净、晾干。将锅放于炉上,放上猪油, 将蹄筋放入锅内,用温油浸炸,有八成泡性时捞起晾冷;再 用清水冰起泡把3又用七、八成温水泡起,泌去油水;再甩 温水泡三、四次,蹄筋的油已除尽即可。将每根蹄筋对破, 边子另用,每半边片成二片,同样将十六根蹄筋片完〔但薄的 就不片);片后放在冷水内然后捞起用刀尖剗不能将蹄筋劁 断再放入碗内用开水过两次;泌去;再放下开水,并放入 草碱四分,用盘子盖住碗发上十五分钟;蹄筋粑了再用热水 换上两次,又用冷水冰起。

\step 将丨1%蛋煮熟,去壳,放入一"个盘内装好;再将火腿切 成丝.共切十几条即可3

\step 	将蹄筋碗内的冷水泌去,再用开水烫后泌去,用二汤过 一次;并将鸽蛋烫热,将蹄筋放入大海碗内,鸽蛋镶于周围。

\step 	将锅放于旺火上,倒入好鸡汤,烧至微开,再将鸡脑茸 用清水解散倒入锅内,肉沫浮起,打尽杂质,再放下味精、 胡椒、料酒,烧开后,倒入装蹄筋的大海碗内即成。

\notes

清爽、鲜美、四季可口。

\end{recipe}

% vim: filetype=tex noautoindent
% vim: fileencoding=utf-8
% vim: textwidth=78 tabstop=4 shiftwidth=4 softtabstop=4
