\begin{recipe}{五色卷}

\ingredients

\ingredient{鲜鱼肉}{三两}
\ingredient{肥肉}{二两}
\ingredient{鸡蛋}{三个}
\ingredient{冬笋}{一苞}
\ingredient{水豆粉}{二钱}
\ingredient{干豆粉}{三钱}
\ingredient{盐}{四分}
\ingredient{料酒}{五钱}
\ingredient{味精}{二分}
\ingredient{特级清汤}{一斤半}
\ingredient{红萝卜、白萝卜、青笋}{各一根}

\preparation

\step 选直径六分大的红萝卜、青笋,连同白萝卜各洗净去皮,先用刀切方正,切片成三
寸长、六分宽的薄片各六片,在沸水内泹软,捞入清水内漂起。冬笋剥壳,两头切正,在
沸水内煮熟,在清水内冷却后捞起,用刀车成大薄片漂在清水内。鸡蛋黄摊成蛋皮(不要
摊老)呈浅黄色。鸡蛋清、干豆粉调成蛋清豆粉。以上材料待用。

\step 鱼肉、肥肉、蛋清等搅成“鱼糁”,分成三十份,做时将青笋等片捞起,粘干水份,
一片一片地抹上蛋清豆粉,再舀上一份鱼糁在片的一端,随即裹成卷形,裹紧粘稳,交头
处向下。

\step 蛋皮和笋片尽材料同样裹成粗细成与青笋等一样大小的长卷,一并拣入盘内。上笼
蒸五分钟取下,蛋皮冬笋卷各切成六分长的节子,蒸溢出来的糁两头修齐,使之整齐光
洁。走菜前装入碗内气热,临走时翻入碗内,灌入烧好的特级清汤即成。

\features

色分五彩,菜浮汤面,味美汤鲜,席桌中适宜走中菜。

\end{recipe}

% vim: filetype=tex noautoindent nojoinspaces
% vim: fileencoding=utf-8 formatoptions+=m
% vim: textwidth=78 tabstop=4 shiftwidth=4 softtabstop=4
