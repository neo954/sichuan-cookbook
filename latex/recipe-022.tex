\begin{recipe}{麻圆肉}

\ingredients

\ingredient{猪保肋肉}{一斤}
\ingredient{干豆粉}{一两}
\ingredient{鸡蛋}{二个}
\ingredient{面粉}{三钱}
\ingredient{白糖}{四两}
\ingredient{菜油}{一斤半耗一两}
\ingredient{熟芝麻}{一两}

\preparation

\step 割去保肋痩肉,铲去肉皮,成猪肥膘。先在汤锅内煮透心(已熟)捞起,切成四分
见方的颗,再入沸水内汆一次,去其浮油,捞起晾干水气,即在连蛋黄的蛋豆粉内裹上一
层,即成待用的肉丸。

\step 菜油烧至八成火,将肉丸入油锅炸至浅黄色捞起。油泌尽后,在锅内放入白糖及少
许清水,炒至糖汁起大泡时加入芝麻,即倒入炸过的肉丸,边炒边将锅几颠,使肉丸裹糖
起锅。敞风,糖即干,盛盘。

\features

以前多用于一般席桌的碟子上,以便客人包回家去。味香甜,很象糖果店“麻圆果子”,
故名。

\end{recipe}

% vim: filetype=tex noautoindent nojoinspaces
% vim: fileencoding=utf-8 formatoptions+=m
% vim: textwidth=78 tabstop=4 shiftwidth=4 softtabstop=4
