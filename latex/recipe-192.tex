\begin{recipe}{凤尾鱼翅}

\ingredients

\ingredient{鱼翅}{一两五}
\ingredient{鸡脯肉}{二两}
\ingredient{鸡蛋}{三个}
\ingredient{肥瞟}{一两五}
\ingredient{熟火腿}{一两}
\ingredient{丝瓜皮}{一恨}
\ingredient{水豆粉}{五线}
\ingredient{特级清汤}{二斤}
\ingredient{盐}{四分}
\ingredient{料酒}{三分}
\ingredient{味精}{二分}
\ingredient{胡椒}{二分}

\cooking

\step 鸡脯、肥膘分别砸茸,加清水、蛋清、盐、味精、水豆粉搅成“鸡糁”;鱼翅先在沸
水内煮数分钟,使之干净柔软,盛入蒸碗内,掺沸水上笼蒸半小时取出,去净杂质,再放
入料酒、盐、清汤汆一次,捞出晾干水气待用;鸡蛋一个,摊成蛋皮。

\step 火腿、丝瓜皮、蛋皮各切成一寸五长的细丝,连同鱼翅分别摆开。

\step 用二十四根调羹,各抹油少许,即将火腿、丝瓜皮、蛋皮、鱼翅各丝共拈数根配成
色,靠着调羹把子斜放,并将

“鸡糁”舀于调羹内,使各丝的头子盖入约二分长度粘稳, 上笼蒸约五分钟即熟,取出全
部倒在一个盘内晾冷成半制品。

\step 走菜时,先将半制品馏热,锅内烧清汤吃味,再用鸡茸清扫两次,即将馏热的半制
品梭下;汤再沸时,盛入大碗内上席。

\features

色味俱备,质嫩汤清,夏季最宜。

\end{recipe}

% vim: filetype=tex noautoindent
% vim: fileencoding=utf-8 formatoptions+=m
% vim: textwidth=78 tabstop=4 shiftwidth=4 softtabstop=4
