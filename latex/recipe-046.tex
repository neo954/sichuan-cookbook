\begin{recipe}{肥肠豆沙汤}

\ingredients

\ingredient{猪肥肠四段}{约四斤}
\ingredient{化猪油}{二两}
\ingredient{千施豆}{七两}
\ingredient{苏打}{五分}

\ingredient{味精}{三分}
\ingredient{盐}{七钱五}
\ingredient{料酒}{三钱}
\ingredient{纯鸡汤}{二片}
\ingredient{姜(去皮拍松)}{二钱}
\ingredient{葱白}{三钱}

\ingredient{花椒}{约十五粒}

\cooking

\step 取下肥肠的中段(不粗不细)四段,将盐在肥肠上用力揉搓,再于清水中淘洗干净,除去粘液涎水。再换清水洗一次,把肥肠有油的一面翻出来,撕扯掉汕上粘的杂质脏物(不要撕掉油)。这样反复清洗几次,洗至肠臼,涩手,无臭味为止;再把有油的一面翻到内面去。肥肠洗净后在开水锅内煮约一刻钟捞出,用刀修去两头约一分厚,装入大蒸碗内,加葱白、花椒、料酒、盐、姜等佐料和清水四两(以没

过肥肠为度),上笼蒸约三小时,至肠皮起皱折为合适。

\step 干豌豆用温水泡十二小时(水要没过豌豆一寸),即泡胀,、然后倒入竹筲箕内滤去水。随后把苏打放入豌豆内抄匀,倒在锅内加水,放在微火上烧,焙两小时,豌豆已至极烂,即倒入筲箕内,/把水滤干,倒入木瓢内(目水的瓢),然后双手持漏瓢在豌豆上用力压,边压边移,豌豆泥就由漏瓢孔内冒出,透不过孔的,即为豌豆壳不用。

\step 猪油放入炒锅内,于旺火上烧至七成热,即将亘泥炒翻沙,加入盐及鸡汤,用汤瓢背将豆沙按散。开锅后撇去浮袜,随后将笼内的肥肠取出,切成四分长的段,倒入锅内同煮。开锅后撇去浮油,舀入放好味精的碗内即成(碗内再放少许葱末、姜汁水亦可)。

\notes

此汤是由民间小吃“肠肠汤”发展而成的。肠略呈黄 色,味鲜而浓,肠肥软而嫩,利口不腻。吃时用调羹,豆沙 入口更觉酥香,筵席上作佐饭汤菜,颇受欢迎。暑天可稍加 泡青菜帮子同熬。

\end{recipe}

% vim: filetype=tex noautoindent
% vim: fileencoding=utf-8
% vim: textwidth=78 tabstop=4 shiftwidth=4 softtabstop=4
