% BSD 3-Clause License
%
% Copyright (c) 2023, 2024 Quux System and Technology. All rights reserved.
%
% Redistribution and use in source and binary forms, with or without
% modification, are permitted provided that the following conditions are met:
%
% 1. Redistributions of source code must retain the above copyright notice, this
%    list of conditions and the following disclaimer.
%
% 2. Redistributions in binary form must reproduce the above copyright notice,
%    this list of conditions and the following disclaimer in the documentation
%    and/or other materials provided with the distribution.
%
% 3. Neither the name of the copyright holder nor the names of its
%    contributors may be used to endorse or promote products derived from
%    this software without specific prior written permission.
%
% THIS SOFTWARE IS PROVIDED BY THE COPYRIGHT HOLDERS AND CONTRIBUTORS "AS IS"
% AND ANY EXPRESS OR IMPLIED WARRANTIES, INCLUDING, BUT NOT LIMITED TO, THE
% IMPLIED WARRANTIES OF MERCHANTABILITY AND FITNESS FOR A PARTICULAR PURPOSE ARE
% DISCLAIMED. IN NO EVENT SHALL THE COPYRIGHT HOLDER OR CONTRIBUTORS BE LIABLE
% FOR ANY DIRECT, INDIRECT, INCIDENTAL, SPECIAL, EXEMPLARY, OR CONSEQUENTIAL
% DAMAGES (INCLUDING, BUT NOT LIMITED TO, PROCUREMENT OF SUBSTITUTE GOODS OR
% SERVICES; LOSS OF USE, DATA, OR PROFITS; OR BUSINESS INTERRUPTION) HOWEVER
% CAUSED AND ON ANY THEORY OF LIABILITY, WHETHER IN CONTRACT, STRICT LIABILITY,
% OR TORT (INCLUDING NEGLIGENCE OR OTHERWISE) ARISING IN ANY WAY OUT OF THE USE
% OF THIS SOFTWARE, EVEN IF ADVISED OF THE POSSIBILITY OF SUCH DAMAGE.
%
\begin{recipe}{肥肠豆沙汤}

\ingredients

\ingredient{猪肥肠四段}{约四斤}
\ingredient{化猪油}{二两}
\ingredient{干豌豆}{七两}
\ingredient{苏打}{五分}
\ingredient{味精}{三分}
\ingredient{盐}{七钱五}
\ingredient{料酒}{三钱}
\ingredient{纯鸡汤}{二斤}
\ingredient{姜(去皮拍松)}{二钱}
\ingredient{葱白}{三钱}
\ingredient{花椒}{约十五粒}

\preparation

\step 取下肥肠的中段(不粗不细)四段,将盐在肥肠上用力揉搓,再于清水中淘洗干
净,除去粘液涎水。再换清水洗一次,把肥肠有油的一面翻出来,撕扯掉油上粘的杂质脏
物(不要撕掉油)。这样反复清洗几次,洗至肠白,涩手,无臭味为止;再把有油的一面
翻到内面去。肥肠洗净后在开水锅内煮约一刻钟捞出,用刀修去两头约一分厚,装入大蒸
碗内,加葱白、花椒、料酒、盐、姜等佐料和清水四两(以没过肥肠为度),上笼蒸约三
小时,至肠皮起皱褶为合适。

\step 干豌豆用温水泡十二小时(水要没过豌豆一寸),即泡胀,然后倒入竹筲箕内滤去
水。随后把苏打放入豌豆内抄匀,倒在锅内加水,放在微火上烧,㸆两小时,豌豆已至极
烂,即倒入筲箕内,把水滤干,倒入木瓢内(舀水的瓢),然后双手持漏瓢在豌豆上用力
压,边压边移,豌豆泥就由漏瓢孔内冒出,透不过孔的,即为豌豆壳不用。

\step 猪油放入炒锅内,于旺火上烧至七成热,即将豆泥炒翻沙,加入盐及鸡汤,用汤瓢
背将豆沙按散。开锅后撇去浮沫,随后将笼内的肥肠取出,切成四分长的段,倒入锅内同
煮。开锅后撇去浮油,舀入放好味精的碗内即成(碗内再放少许葱末、姜汁水亦可)。

\features

此汤是由民间小吃“肠肠汤”发展而成的。肠略呈黄色,味鲜而浓,肠肥软而嫩,利口不
腻。吃时用调羹,豆沙入口更觉酥香,筵席上作佐饭汤菜,颇受欢迎。暑天可稍加泡青菜
帮子同熬。

\end{recipe}

% vim: filetype=tex noautoindent nojoinspaces
% vim: fileencoding=utf-8 formatoptions+=m
% vim: textwidth=78 tabstop=4 shiftwidth=4 softtabstop=4
