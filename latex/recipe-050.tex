\begin{recipe}{四上玻璃肚}

\ingredients

\ingredient{猪肚头三个}{一斤}
\ingredient{花椒}{约二十粒}
\ingredient{酱油}{一两五}
\ingredient{箩粉}{五两}
\ingredient{姜}{二钱}
\ingredient{醋}{六钱}
\ingredient{蕃芬}{五两}
\ingredient{大縣}{二钱}
\ingredient{全工酱油}{少许}
\ingredient{熟油辣椒}{三钱}
\ingredient{白糖}{二分}
\ingredient{盐}{少许}
\ingredient{葱叶}{二钱}
\ingredient{香油}{五钱}
\ingredient{草碱}{六钱}
\ingredient{清汤}{少许}

\preparation

\step 选鲜猪肚,只取肚头部分,清洗干净;用刀将两面的筋缠及边沿修去,成长方形,
随着形式用刀将肚头片成板薄的片(越薄越好)。片的要求:要薄、要匀、要不片烂。
\step 用清水少许将草碱溶化于肚片上造匀,在盆内浸渍半小时后,用沸水冲入盖严;烫
焖十分钟揭盖,即将碱水泌去,另换清水;每隔五分钟换一次,如是换四次,直到将碱味
去掉为止。此时肚片则变为细嫩、柔软、半透明,如玻璃体状。
\step 箩粉用刀切成大姜糖块形(即大斜方形),用盐、香油少许拌过;蕃茄先用沸水烫
后,撕皮去蒂,片成薄片。
\step 葱叶、花椒加盐少许,在菜墩上用刀共同铡成细末,加白酱油、香油及红酱油各少
许,兑成椒麻调料;白酱油、香 油及清汤少许,兑成白油调料;熟油辣椒,加白糖、香
油、 白酱油、醋少许,兑成红油调料;生姜去皮,切成姜米,加醋、白酱油、香油少
许,兑成姜汁调料。以上四种调料分别 盛入四个汤杯。
\step 走菜时肚片在开水内冒后,滤干水份放入大盘中间,周围镶與蕃莼片、箩粉技,相
互间隔摆好,连同四个调料汤杯,随幕上席。

\features

躍色调和,质地脆嫩,菜以四个不同口味的调料蘸食, 故名“四上“。

\end{recipe}

% vim: filetype=tex noautoindent nojoinspaces
% vim: fileencoding=utf-8 formatoptions+=m
% vim: textwidth=78 tabstop=4 shiftwidth=4 softtabstop=4
