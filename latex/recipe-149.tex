\begin{recipe}{锅贴鱼片}

\ingredients

\ingredient{净鲜鱼肉}{八两}
\ingredient{鲜笋}{二两}
\ingredient{净生菜}{二两}
\ingredient{白糖}{五分}

\ingredient{盐}{三分}
\ingredient{醋}{二分}
\ingredient{香油}{五钱}
\ingredient{化猪油}{一'两}
\ingredient{熟猪肥瞟}{一斤}
\ingredient{熟火腿}{一两}
\ingredient{鸡蛋清}{三个}
\ingredient{千豆粉}{一两五}

\cooking

\step 把肥膘片成一寸五长、九分宽、一分半厚,共二十四片,用刀尖逢中剗两刀。鱼片切成长一寸五、宽九分、厚一分,共二十四片,用盐、料酒少许拌和均匀。鲜笋切成宽五分、长一寸二的薄片;火腿剁成蒙子;鸡蛋合千豆粉搅成蛋清豆粉;生菜淘洗干净漂起,临走菜时捞起滤干水气,加白糖、盐、醋、香油拌匀。

用条盘一个,将肥膘全部铺起,用净布入热水透过拧 干,在肥膘上分别沾去油质一至二次后,匀净地抹上蛋清豆 粉。先将笋片贴在肥膘的半边,再贴上火腿蒙子在其余半 边。然后把剩余的蛋清豆粉合鱼片拌勻,一片一片地盖在铺 好笋子、火腿蒙的肥膘上盖好。

\step 先将炒锅在火上把锅的周围炕热,下油一两浪匀后,提离火口泌尽油;分别拣起鱼片将肥膘的一面贴在锅上,先贴锅边后贴锅底;贴完后放在火上坑,随时用手将锅车动,使火候匀净;炕至肥膘成鸭黄色时,将就烤出的油将鱼片浪熟后,将余油泌去,加香油起锅,盛入白色条盘,镶上生菜即上席。

\notes

脆、酥、嫩、香,佐酒好菜。

\end{recipe}

% vim: filetype=tex noautoindent
% vim: fileencoding=utf-8
% vim: textwidth=78 tabstop=4 shiftwidth=4 softtabstop=4
