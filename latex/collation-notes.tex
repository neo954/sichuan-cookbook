% BSD 3-Clause License
%
% Copyright (c) 2023 Quux System and Technology. All rights reserved.
%
% Redistribution and use in source and binary forms, with or without
% modification, are permitted provided that the following conditions are met:
%
% 1. Redistributions of source code must retain the above copyright notice, this
%    list of conditions and the following disclaimer.
%
% 2. Redistributions in binary form must reproduce the above copyright notice,
%    this list of conditions and the following disclaimer in the documentation
%    and/or other materials provided with the distribution.
%
% 3. Neither the name of the copyright holder nor the names of its
%    contributors may be used to endorse or promote products derived from
%    this software without specific prior written permission.
%
% THIS SOFTWARE IS PROVIDED BY THE COPYRIGHT HOLDERS AND CONTRIBUTORS "AS IS"
% AND ANY EXPRESS OR IMPLIED WARRANTIES, INCLUDING, BUT NOT LIMITED TO, THE
% IMPLIED WARRANTIES OF MERCHANTABILITY AND FITNESS FOR A PARTICULAR PURPOSE ARE
% DISCLAIMED. IN NO EVENT SHALL THE COPYRIGHT HOLDER OR CONTRIBUTORS BE LIABLE
% FOR ANY DIRECT, INDIRECT, INCIDENTAL, SPECIAL, EXEMPLARY, OR CONSEQUENTIAL
% DAMAGES (INCLUDING, BUT NOT LIMITED TO, PROCUREMENT OF SUBSTITUTE GOODS OR
% SERVICES; LOSS OF USE, DATA, OR PROFITS; OR BUSINESS INTERRUPTION) HOWEVER
% CAUSED AND ON ANY THEORY OF LIABILITY, WHETHER IN CONTRACT, STRICT LIABILITY,
% OR TORT (INCLUDING NEGLIGENCE OR OTHERWISE) ARISING IN ANY WAY OUT OF THE USE
% OF THIS SOFTWARE, EVEN IF ADVISED OF THE POSSIBILITY OF SUCH DAMAGE.
%
\begingroup%
\vbadness=10000%
\small%
\setlength{\parskip}{0\baselineskip plus .05\baselineskip}%
\setlength{\parindent}{1.945945em}%
\begin{center}%
{%
	\Large\rmfamily\bfseries%
	\hbadness=10000\makebox[5em][s]{校勘记}%
}%
\end{center}%

\vspace{.6875\baselineskip}%
% Reduce the \textwidth 3bp
\begin{list}{}{%
	\setlength{\topsep}{0pt}%
	\setlength{\leftmargin}{1.5bp}%
	\setlength{\rightmargin}{1.5bp}%
	\setlength{\listparindent}{\parindent}%
	\setlength{\itemindent}{\parindent}%
	\setlength{\parsep}{\parskip}%
}%
\item[]%

本书原为成都市饮食公司所编纂。成都市饮食公司成立于一九五六年,是一家专营川菜、
成都小吃的特色餐饮企业。原书未公开出版,仅以内部资料发行于一九七二年七月。据编
印说明中记述,本书系参照《中国名菜谱第七辑》和《重庆名菜谱》修改整理而成。四川
省蔬菜水产饮食服务公司编纂,一九七七年七月发行的内部资料《四川菜谱》收集了两百
六十四个菜品,有少部分来自于本书。成都饮食公司一九八八年九月再版重印本书,仍以
内部资料发行。

今以《四川菜谱》成都市饮食公司一九七二年七月版为底本。校以《四川菜谱》成都市饮
食公司一九八八年九月版。《四川菜谱》四川省蔬菜水产饮食服务公司一九七七年七月
版、《中国名菜谱第七辑四川名菜点》商业部饮食服务业管理局一九六二年六月中国财政
经济出版社版,和《重庆名菜谱》重庆市饮食服务公司一九六〇年五月重庆人民出版社
版,亦尽可能采用。《四川菜谱》一九七二年七月版,省略称为《四》;《四川菜谱》一
九八八年九月版,省略称为《八》;《四川菜谱》四川省蔬菜水产饮食服务公司一九七七
年七月版,省略称为《七》;《中国名菜谱第七辑四川名菜点》,省略称为《中》;《重
庆名菜谱》,省略称为《重》。

校勘基本方针。保留方言用字。修正异体字。对已经弃用的同音简化字和类推简化字,恢
复原字。修正误植、错字、别字。

\vspace{.5\baselineskip plus .5\baselineskip minus .5\baselineskip}%
{\noindent\null\hfill\pgfornament[width=64mm]{86}\hfill\null}%
\vspace{1\baselineskip plus .5\baselineskip minus .5\baselineskip}%

\recipelist{2,7,8,9,20,22,24,28,29,31,36,37,38}等多篇。“滗”,《四》本作“泌”。
径改。“滗”,滤。

\recipelist{4,24,71,79,92,94,100,109,112}等多篇。“搛”,《四》本作“扦”。径改。
“搛”,(用筷子)夹。

\reciperef{6}。“锅巴”,《四》本作“锅粑”。据《中》本改。

\recipelist{6,9,58,91,93,99,197,230,260}。“煳”,《四》本作“糊”。径改。

\reciperef{7}。《四》本无脚注{\footnotesize\circled{1}}。据《中》本补。

\recipelist{8,11,16,17,35,54,73,78,109}等多篇。“醪糟”,各本皆作“𰪿糟”。径改。
“𰪿”系“𫃑”之类推简化字。

\reciperef{9}。“使全部蘸裹在肉片上”。“蘸”,《四》本作“醮”。据《中》本改。

\recipelist{15,26,35,73,95,96}等多篇。“鱼碗”,一种盛菜或装汤的大碗。其尺寸比斗
碗大,约为七至十一寸,碗深度比斗碗稍浅。也叫小鱼钵。“斗碗”,也是指大碗。旧时民
间指较大的土碗。

\recipelist{16,66,115,164}。“腌”,《四》本作“醃”。径改。

\reciperef{17}。“瓤”,各本皆作“穰”。径改。

\recipelist{18,19}。“尖刀圆子”,《四》本《八》本《七》本作“尖刀元子”。《中》本
作“尖刀丸子”。径改。

\recipelist{18,279}。“汆”,《四》本作“川”。径改。

\recipelist{19,20,56,195,254}所附共十一副插图。《四》本《八》本插图,线条边缘
参差不齐模糊不清。据《中》本插图重制。

\recipelist{22,33}。“肉圆”,《四》本作“肉元”。径改。

\recipelist{24,48,55,75,76,84,92,107,111}等多篇。“口蘑”,《四》本作“口茉”。据
《中》本《重》本改。

\recipelist{26,93,96,110,111,219,244}。“三叠水”,菜肴的一种传统装盘形式。多用
于冷菜。先取菜肴的片(或条),按顺序摆成两行,然后再在两行中部覆盖一行即成。要
求刀工处理时注意厚薄均匀,长短一致,并按刀口等距排列,整齐入盘。

\recipelist{26,95,207,262}。“风车”(风车形),传统装盘形式之一。因形似风车,故
名。多用于冷菜摆拼。用荤素原料改成厚薄一致的长方片,将色泽和荤素岔开,按刀面等
距、整齐地装入圆盘;装盘时,原料要下宽上窄、前搭后、顺时针方向镶摆一圈;圆盘中
部可用其他冷菜堆摆成形。

\recipelist{28,29,33,163,205,206,208,258,271}。“圆”,《四》本作“元”。径改。

\recipelist{29,30,31,32,96,177,179,265,269,293,295,310,311}。“铝锅”,各本皆作
“锑锅”。\xeCJKnobreak{}径改。

\reciperef{30}。“捡去泡沫”,《四》本《八》本作“检去泡沫”。径改。“杆杖”,《四》
本作“杆仗”。径改。

\recipelist{33,34,42,68,88,89,93,97,117,127,151}等多篇。“茨菰”,《四》本作“慈
菇”。径改。

\recipelist{33,34,147,188,191}。“圆子”,《四》本作“元子”。径改。

\reciperef{37}。“扳指”,《四》本作“斑指”,《中》本《七》本作“班指”。径
改。“……至肥肠起着皱折为适合。”“皱折”,《四》本作“绉折”。据《八》本改。

\recipelist{39,48,49}。“擀”,《四》本作“扞”。据《八》本改。

\reciperef{40}。“磉磴”。
“磉”,{s\v{a}ng}{ㄙ\kern-.333333emㄤ\kern-.75em\raise.5ex\hbox{\v{}}\kern.25em}
柱下石。
“磴”,{d\`{e}ng}{ㄉ\kern-.333333emㄥ\kern-.75em\raise.5ex\hbox{\`{}}\kern.25em}
石阶。

\recipelist{41,42,71,121,257}。“剞”,《四》本作“𠟤”。径改。

\recipelist{41,265,270}。“臊”,《四》本作“骚”。径改。

\recipelist{44,49,87,106,117,206,248}。“烂”,《四》本作“滥”。径改。

\recipelist{45,64}。“煳辣”,《四》本作“糊辣”。径改。

\reciperef{48}。“水龙头”,《四》本作“水笼头”。径改。

\recipelist{50,79,85,91,158,214,220,230,256,260,274}。“番茄”,各本皆作“蕃茄”。
径改。

\recipelist{52,261}。“环喉”,猪或牛的主动脉血管。

\recipelist{55,69,153}。“须”,《四》本作“鬚”。据《八》本改。

\recipelist{55,186,188,189,223,261,288,293}。“珧柱”,《四》本作“𧎼蛀”。《中》
本作“𧎼柱”。径改。

\recipelist{55,296}。“坛”,《四》本作“罈”。《中》本作“罎”。径改。

\reciperef{56}。“将烤池内的红杠炭捡于烤池四周”。“捡”,《四》本《八》本《中》本
皆作“检”。径改。“在池中捡红杠炭一块”。“捡”,《四》本作“检”,《八》本作“拣”。据
《中》本改。

\reciperef{61}。“窝油”,一种传统手工技艺酿造酱油。因酱胚产生的原汁液是从盛曲料
容器慢慢滴入窝内,故名窝油。又称白酱油。

\reciperef{63}。各本皆作“劖(音沾……)”。误。据《康熙字典》,“劖”,锄衔切,
{ch\'{a}n}{ㄔ\kern-.333333emㄢ\kern-.75em\raise.5ex\hbox{\'{}}\kern.25em}。

\recipelist{67,68,82,92,118,127,150,255,280,301}。“熘”,《四》本作“溜”。径改。

\reciperef{69}。“子姜”,《四》本作“仔姜”。径改。

\recipelist{72,78,102,108,246,253}。“煺毛”,《四》本作“退毛”。径改。

\reciperef{73}。“箅子”,《四》本作“篦子”,《八》本作“篾子”。径改。

\reciperef{74}。脚注{\footnotesize\circled{1}}中“……盛产于成都近郊。”《中》本作
“……盛产于成都附近山崖中。”“三菌”,鸡𭎂(㙡){z\={o}ng}%
{ㄗ\kern-.333333emㄨ\kern-.333333emㄥ}菌。

\reciperef{81}。“淖”,{n\`{a}o}%
{ㄋ\kern-.333333emㄠ\kern-.75em\raise.5ex\hbox{\`{}}\kern.25em}\,泥也。

\recipelist{84,219}。“化鸡油”,《四》本作“鸡化油”。乙正。

\reciperef{87}。“擀”,《四》本作“扞”,《八》本作“撖”。径改。

\recipelist{93,123,137,192,251,292}。“趖”,《四》本作“梭”。径改。“趖”,
{su\={o}}\,{ㄙ\kern-.333333emㄨ\kern-.333333emㄛ}[方言]移动。

\reciperef{94}。“㮟”,{k\={a}}\,{ㄎ\kern-.25emㄚ}\,(紧)夹,扎,刺。

\reciperef{95}。《八》本标题作“刷把鸡丝(汤)”。

\reciperef{100}。“叫花鸡”,《四》本作“叫化鸡”,径改。“𫃕”,%
{c\'{i}}\,{ㄘ\kern-.75em\raise.5ex\hbox{\'{}}\kern.25em}\,粘稠。

\reciperef{101}。“鸡翘”,鸡屁股。

\reciperef{102}。“魔芋”,《四》本作“茉芋”。径改。

\reciperef{106}。“锅内掺清汤”。“掺”,《四》本作“渗”。径改。

\recipelist{111,295}。“罐”,《四》本作“罆”。据《八》本改。

\recipelist{120,121,123,223,257,262}。“肫”,《四》本作“𬂁”。径改。“肫”,鸟类的
胃。

\recipelist{120,129,130,131,132,133,134,135,136,137}等多篇,“鳃”。各本皆作
“腮”。径改。

\reciperef{121}。《四》本标题作“川双脆”。径改。

\reciperef{142}。“甲鱼”,《四》本全篇皆作“足鱼”。据《中》本《七》本改。“肉𤆵可
口”,《中》本作“肉烂可口”。“但须注意甲鱼肉不可与苋菜同食”。“甲鱼肉”,《中》本
作“龟鳖肉”。

\reciperef{143}。《四》本中脚注标记{\footnotesize\circled{1}}在“原料”段落中“鲢
鱼头一个约四、五斤”之后。据《中》本改至标题之后。“竹篾箅”,《四》本作“竹篾篦”。
据《中》本改。“即用筷子将鱼肉轻轻拨下”,《四》本作“……轻轻试一下”。据《中》本
改。脚注{\footnotesize\circled{1}}中“而不是一般池塘养的鲢鱼”,《四》本无此句。
据《中》本补。之后“但四川人……”,《四》本无“但”字。据《中》本补。脚注
{\footnotesize\circled{1}}中“身圆口大”,《四》本作“身团口大”。据《中》本改。

\reciperef{144}。“酱油、白糖、醋、味精、水豆粉、料酒、清汤先兑成滋汁一碗。”“白
糖”,《八》本作“冰糖”。

\recipelist{147,150}。“鱼圆”,《四》本作“鱼元”。径改。

\reciperef{148}。“咖喱”,《四》本作“咖哩”。径改。

\recipelist{157,173,178,182,183,184,185,193,263}。“煨”,《四》本作“喂”。径改。

\reciperef{158}。“用刀切成与虾肉一样大的丁颗”,《四》本无“一样”二字。文意不通。
径补。

\recipelist{162,248,256,257,258}。“吐司”,《四》本作“土司”。径改。“吐司”,烤面
包片。[英~toast]

\reciperef{163}。“鲜虾仁”,《四》本“鲜”前衍一“尽”字。径删。

\reciperef{180}。“带丝”,《四》本作“代丝”。径改。“带丝”,海带丝。

\reciperef{183}。“擀”,《四》本作“扞”。径改。

\reciperef{187}。“旗子块”,《四》本作“棋子块”。径改。

\reciperef{194}。“铝盘”,《四》本作“锑盘”。径改。

\recipelist{202,204}。“红苕”,亦称番薯、红薯、甘薯。

\reciperef{204}。“青糖”,《四》本作“清糖”,径改。“青糖”,粗制赤(红)糖。

\recipelist{204,207}。“橘红”,《四》本作“桔红”。径改。

\reciperef{205}。“圆子”,《四》本作“元子”。径改。“入沸水潦过心”。“潦”,用开水
略煮。

\reciperef{207}。“百合”,《四》本作“白合”。径改。

\reciperef{212}。“大拇指”,《四》本作“大姆指”。据《中》本改。

\reciperef{217}。“再放入炒好的馅于萝卜腹内,用盖盖好……”\\%
“用盖盖好”,《四》本《八》本作“用盖好”。径补。

\recipelist{223,261}。“什锦”,《四》本作“什景”。径改。

\recipelist{225,226}。“胆水”,《四》本作“𥑲水”。径改。“胆水”,盐卤。

\reciperef{228}。“四张豆腐皮一叠,如是再重四张,共为两叠。”“叠”,《四》本作
“迭”。径改。“如是”,《四》本作“如果”。径改。

\reciperef{233}。《四》本“八分火”之后无脚注标记{\footnotesize\circled{1}}。据
《八》本《中》本补。

\reciperef{234}。“椿芽”,《四》本作“春芽”。径改。

\reciperef{244}。“带皮”,海带。亦称海藻、昆布、海布。

\reciperef{245}。“盐六分”,《四》本作“盐一分”。据《中》本改。《四》本无脚注%
{\footnotesize\circled{1}\circled{2}\circled{3}\circled{4}}。据《中》本补。

\reciperef{253}。《中》本标题作“子母会”。“因鸽子与鸽蛋同配一菜,故名子母会
(烩)。”《四》本括号部分在句号之后。《中》本无括号部分内容。据《标点符号用法》
\textsc{gb/t 15834--2011},将括号部分移至句号之前。

\reciperef{262}。《四》本原料部分列有“干粉~五钱”,制作方法中并未提及。径删。

\reciperef{264}。“抽去……扦子骨……入抽去扦子骨处”。“扦”,~《四》本作“杄”,《八》
本作“{\bfseries\scalebox{.65}[1]{林}\kern-.12em\scalebox{.55}[1]{千}}”。径改。

\recipelist{264,276,302}。“仔鸡”,《四》本作“子鸡”。径改。

\reciperef{267}。“扳指”,《四》本作“斑指”。径改。“……以防炸煳。”“煳”,《四》本
作“胡”。径改。

\recipelist{267,308}。“汆一道”,《四》本作“川一道”。径改。

\reciperef{268}。“阴米”。将糯米淘净,用水稍浸泡,捞出沥干水分,入笼用旺火蒸熟,
取出晾干即成。

\reciperef{269}。“竹荪”,《四》本作“竹参”。径改。

\reciperef{273}。各本标题皆作“汗蒸鱼”。径改。

\reciperef{292}。“莴笋”,《四》本作“窝笋”。径改。“瓢儿白”,上海白菜。亦称上海
青、苏州青、青江菜、小棠菜、青梗白菜、青江白菜、 汤匙菜等。

\recipelist{303,307}。“仔公鸡”,《四》本作“子公鸡”。径改。

\end{list}
\endgroup%

% vim: filetype=tex noautoindent nojoinspaces
% vim: fileencoding=utf-8 formatoptions+=m
% vim: textwidth=78 tabstop=4 shiftwidth=4 softtabstop=4
