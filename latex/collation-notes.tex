% BSD 3-Clause License
%
% Copyright (c) 2023 Quux System and Technology. All rights reserved.
%
% Redistribution and use in source and binary forms, with or without
% modification, are permitted provided that the following conditions are met:
%
% 1. Redistributions of source code must retain the above copyright notice, this
%    list of conditions and the following disclaimer.
%
% 2. Redistributions in binary form must reproduce the above copyright notice,
%    this list of conditions and the following disclaimer in the documentation
%    and/or other materials provided with the distribution.
%
% 3. Neither the name of the copyright holder nor the names of its
%    contributors may be used to endorse or promote products derived from
%    this software without specific prior written permission.
%
% THIS SOFTWARE IS PROVIDED BY THE COPYRIGHT HOLDERS AND CONTRIBUTORS "AS IS"
% AND ANY EXPRESS OR IMPLIED WARRANTIES, INCLUDING, BUT NOT LIMITED TO, THE
% IMPLIED WARRANTIES OF MERCHANTABILITY AND FITNESS FOR A PARTICULAR PURPOSE ARE
% DISCLAIMED. IN NO EVENT SHALL THE COPYRIGHT HOLDER OR CONTRIBUTORS BE LIABLE
% FOR ANY DIRECT, INDIRECT, INCIDENTAL, SPECIAL, EXEMPLARY, OR CONSEQUENTIAL
% DAMAGES (INCLUDING, BUT NOT LIMITED TO, PROCUREMENT OF SUBSTITUTE GOODS OR
% SERVICES; LOSS OF USE, DATA, OR PROFITS; OR BUSINESS INTERRUPTION) HOWEVER
% CAUSED AND ON ANY THEORY OF LIABILITY, WHETHER IN CONTRACT, STRICT LIABILITY,
% OR TORT (INCLUDING NEGLIGENCE OR OTHERWISE) ARISING IN ANY WAY OUT OF THE USE
% OF THIS SOFTWARE, EVEN IF ADVISED OF THE POSSIBILITY OF SUCH DAMAGE.
%
\begingroup%
\vbadness=10000%
\small%
\begin{center}%
{%
	\Large\rmfamily\bfseries%
	\hbadness=10000\makebox[5em][s]{校勘记}%
}%
\end{center}%

\begin{list}{}{%
	\setlength{\topsep}{0pt}%
	\setlength{\leftmargin}{.529166mm}%
	\setlength{\rightmargin}{.529166mm}%
	\setlength{\listparindent}{\parindent}%
	\setlength{\itemindent}{\parindent}%
	\setlength{\parsep}{\parskip}%
	\addtolength{\textheight}{2.800042mm}%
}%
\item[]%

\vspace{0\baselineskip plus .5\baselineskip}%

本书原为成都市饮食公司所编纂。成都市饮食公司成立于一九五六年,是一家专营川菜、
成都小吃的特色餐饮企业。原书未公开出版,仅以内部资料发行于一九七二年七月。据编
印说明中记述,本书系参照《中国名菜谱第七辑》和《重庆名菜谱》修改整理而成。四川
省蔬菜水产饮食服务公司编纂,一九七七年七月发行的内部资料《四川菜谱》收集了两百
六十四个菜品,约有半数来自于本书。成都饮食公司一九八八年九月再版重印本书,仍以
内部资料发行。

今以《四川菜谱》成都市饮食公司一九七二年七月版为底本。校以《四川菜谱》成都市饮
食公司一九八八年九月版。《四川菜谱》四川省蔬菜水产饮食服务公司一九七七年七月
版、《中国名菜谱第七辑四川名菜点》商业部饮食服务业管理局一九六二年六月中国财政
经济出版社版,和《重庆名菜谱》重庆市饮食服务公司一九六〇年五月重庆人民出版社
版,亦尽可能采用。《四川菜谱》一九七二年七月版,省略称为《四》;《四川菜谱》一
九八八年九月版,省略称为《八》;《四川菜谱》四川省蔬菜水产饮食服务公司一九七七
年七月版,省略称为《七》;《中国名菜谱第七辑四川名菜点》,省略称为《中》;《重
庆名菜谱》,省略称为《重》。

校勘基本方针。保留异体字、方言用字。对已经弃用的同音简化字和类推简化字,恢复原
字。修正误植、错字、别字。

\vspace{1\baselineskip}%

二、七、八……全书各篇,“滗”。《四》本《八》本作“泌”。径改。“滗”,滤。

四、二四、七一……全书各篇,“搛”。《四》本《八》本作“扦”。径改。“搛”,(用筷子)
夹。

六、锅巴肉片。“锅巴”,《四》本《八》本作“锅粑”。据《中》本改。

七、水煮肉片。《四》本《八》本无脚注{\footnotesize\circled{1}}。据《中》本补。

八、一一、一六……全书各篇,“醪糟”,各本皆作“𰪿糟”。径改。“𰪿”系“𫃑”之类推简化
字。

九、芙蓉肉片。“使全部蘸裹在肉片上”。“蘸”,《四》本《八》本作“醮”。据《中》
本改。

一七、大南瓜蒸肉。各本皆作“掏尽内穰”。据《现代汉语词典》,“穰”今作“瓤”。

一八、一九,“尖刀圆子”。《四》本《八》本《七》本作“尖刀元子”。《中》本作“尖刀
丸子”。径改。

一八、二七九,“汆”。《四》本《八》本作“川”。径改。

一九、二〇、五六、一九五、二五四所附共十一副插图。《四》本《八》本插图,线条边
缘参差不齐模糊不清。据《中》本插图重制。

二一、八四、一七五、二二七、二四八,“棋子块”。《四》本《八》本作“旗子块”。径
改。

二二、三三,“肉圆”。《四》本《八》本作“肉元”。径改。

二四、四八、五五……全书各篇,“口蘑”。《四》本《八》本作“口茉”。据《中》本《重》
本改。

二八、二九、三三、一六三、二〇五、二〇六、二〇八、二五八、二七一,“圆”。《四》
本《八》本作“元”。径改。

二九、三十、三一、三二、九六、一七七、一七九、二六五、二六九、二九三、二九五、
三一〇、三一一,“铝锅”。各本皆作“锑锅”。径改。

三〇、芝麻肘子。“捡去泡沫”,《四》本《八》本作“检去泡沫”。径改。“杆杖”,《四》
本《八》本作“杆仗”。径改。

三三、三四、一四七、一八八、一九一,“圆子”。《四》本《八》本作“元子”。径改。

三四、四二、六八、八八、八九、九三、九七、一一七、一二七、一五一、一五四、一六
〇、一六三、二〇〇、二〇二、二〇六、二三七、二四〇、二五〇,“慈姑”。《四》本
《八》本作“慈菇”。径改。

三七、炸扳指。“扳指”,《四》本《八》本作“斑指”,《中》本《七》本作“班指”。径
改。“……至肥肠起着皱折为适合。”“皱折”,《四》本作“绉折”。据《八》本改。

三九、四八、四九,“擀”。《四》本作“扞”。据《八》本改。

四〇、锅贴腰片。“磉磴”。
“磉”,{s\v{a}ng}{ㄙ\kern-.25emㄤ\kern-.75em\raise.5ex\hbox{\v{}}\kern.25em}
柱下石。
“磴”,{d\`{e}ng}{ㄉ\kern-.25emㄥ\kern-.75em\raise.5ex\hbox{\`{}}\kern.25em}
石阶。

四一、四二、七一、一二一、二五七,“𠟤”。疑误。

四一、二六五、二七〇,“臊”。《四》本《八》本作“骚”。径改。

四四、四九、八七、一〇六、一一七、二〇六、二四八,“烂”。《四》本《八》本作
“滥”。径改。

四八、菠饺白肺。“水龙头”,《四》本《八》本作“水笼头”。径改。

五〇、七九、八五……全书各篇,“番茄”。各本皆作“蕃茄”。径改。

五二、六三、一四九、二五四,“劖”。《八》本作
“{\bfseries\raise.19em\hbox{\scalebox{.675}[.75]{免}}\kern-.675em%
\lower.215em\hbox{\scalebox{.95}[.55]{⺀}}\kern-.405616em%
\scalebox{.65}[1]{刂}}”,系已弃用之类推简化字。

五五、罈子肉。《八》本目录中作“罐子肉”。误。《中》本目录中作“坛子肉”,内页
中作“罎子肉”。

五五、六九、一五三,“须”。《四》本作“鬚”。据《八》本改。

五五、一八六、一八八……全书各篇,“珧柱”。《四》本《八》本作“𧎼蛀”。《中》本作
“𧎼柱”。径改。

五六、烤酥方。“将烤池内的红杠炭捡于烤池四周”。“捡”,《四》本《八》本《中》本皆
作“检”。径改。“在池中捡红杠炭一块”。“捡”,《四》本作“检”,《八》本作“拣”。据
《中》本改。

六三、辣子鸡丁。各本皆作“劖(音沾……)”。误。据《康熙字典》,“劖”,锄衔切,
{ch\'{a}n}{ㄔ\kern-.25emㄢ\kern-.75em\raise.5ex\hbox{\'{}}\kern.25em}。

六七、六八、八二、九二、一一八、一二七、一五〇、二五五、二八〇、三〇一,“熘”。
《四》本《八》本作“溜”。径改。

七三、旱蒸全鸡。“箅子”,《四》本作“篦子”,《八》本作“篾子”。径改。

七四、三菌炖鸡。脚注{\footnotesize\circled{1}}中“……盛产于成都附近山崖中。”
《四》本《八》本作“……盛产于成都近郊。”据《中》本改。

八一、雪花鸡淖。“淖”,
{n\`{a}o}{ㄋ\kern-.25emㄠ\kern-.75em\raise.5ex\hbox{\`{}}\kern.25em} 泥也。

八七、牡丹鸡片。“擀”,《四》本作“扞”,《八》本作“撖”。径改。

九四、五彩鸡片。“㮟”,《八》本作“扞”。误。“㮟”,{k\={a}}{ㄎ\kern-.25emㄚ}
(紧)夹,扎,刺。

九五、刷把鸡丝。《八》本标题作“刷把鸡丝(汤)”。“上笼汽三分钟”。“汽”,《八》
本作“气”。误。

一〇〇、泥糊鸡。“𫃕”,{c\'{i}}{ㄘ\kern-.75em\raise.5ex\hbox{\'{}}\kern.25em}
粘稠。

一〇一、包烧鸡。“鸡翘”,鸡屁股。

一〇二、魔芋烧鸭。“魔芋”,《四》本《八》本作“茉芋”。径改。

一〇六、银杏鸭脯。“锅内掺清汤”。“掺”,《四》本《八》本作“渗”。径改。

一二〇、一二一、一二三、二二三、二五七、二六二,“𬂁”。“肫”之异体字。鸟类的胃。

一二〇、一二九、一三〇……全书各篇,“鳃”。各本皆作“腮”。径改。

一二一、汆双脆。《四》本《八》本标题作“川双脆”。径改。

一四二、红烧甲鱼。“甲鱼”,《四》本《八》本全篇皆作“足鱼”。据《中》本《七》本
改。“肉𤆵可口”,《中》本作“肉烂可口”。“但须注意甲鱼肉不可与苋菜同食”。“甲鱼
肉”,《中》本作“龟鳖肉”。

一四三、黄焖大鲢鱼头。《四》本《八》本中脚注标记{\footnotesize\circled{1}}在
“原料”段落中“鲢鱼头一个约四、五斤”之后。据《中》本改至标题之后。“竹篾箅”,
《四》本《八》本作“竹篾篦”。据《中》本改。“即用筷子将鱼肉轻轻拨下”,《四》本
《八》本作“……轻轻试一下”。据《中》本改。脚注{\footnotesize\circled{1}}中“而不
是一般池塘养的鲢鱼”,《四》本《八》本无此句。据《中》本补。之后“但四川人……”,
《四》本《八》本无“但”字。据《中》本补。脚注{\footnotesize\circled{1}}中“身圆
口大”,《四》本《八》本作“身团口大”。据《中》本改。

一四四、五柳鱼。“酱油、白糖、醋、味精、水豆粉、料酒、清汤先兑成滋汁一碗。”“白
糖”,《八》本作“冰糖”。

一四七、一五〇,“鱼圆”。《四》本《八》本作“鱼元”。径改。

一四八、椒盐鱼卷。“咖喱”,《四》本《八》本作“咖哩”。径改。

一五七、一七三、一七八、一八二、一八三、一八四、一八五、一九三、二六三,“煨”。
《四》本《八》本作“喂”。径改。

一六三、五色虾球。“鲜虾仁”,《四》本《八》本“鲜”前衍一“尽”字。径删。

一六九、佛手蜇卷。“蜇”,《八》本作“蛰”。误。

一八三、菠饺鱼肚。“擀”,《四》本《八》本作“扞”。径改。

一九四、佛手海参。“白铝盘”,《四》本《八》本作“白锑盘”。径改。

二〇二、二〇四,“红苕”。亦称番薯、红薯、甘薯。

二〇四、蜜汁苕蛋。“青糖”,《四》本《八》本作“清糖”,径改。“青糖”,粗制赤(红)
糖。

二〇四、二〇七,“橘红”。《四》本《八》本作“桔红”。径改。

二〇五、糯米圆子。“圆子”,《四》本《八》本《中》本作“元子”。径改。“入沸水潦过
心”。“潦”,疑误。

二〇七、八宝酿藕。“百合”,《四》本《八》本作“白合”。径改。

二一二、盐白菜冬笋。“大拇指”,《四》本《八》本作“大姆指”。据《中》本改。

二二三、二六一,“什锦”。《四》本《八》本作“什景”。径改。

二二八、格花豆腐皮。“四张豆腐皮一叠,如果再重四张,共为两叠。”“叠”,《四》本
《八》本作“迭”。径改。

二三三、清汤白菜。《四》本“八分火”之后无脚注标记{\footnotesize\circled{1}}。据
《八》本《中》本补。

二四五、酸辣虾羹汤。“盐六分”,《四》本《八》本作“盐一分”。据《中》本改。《四》
本《八》本无脚注{\footnotesize\circled{1}\circled{2}\circled{3}\circled{4}}。
据《中》本补。

二五三、子母烩。《中》本标题作“子母会”。“因鸽子与鸽蛋同配一菜,故名子母会
(烩)。”《四》本《八》本括号部分在句号后面。《中》本无括号部分内容。据《标点
符号用法》\textsc{gb/t 15834--2011},将括号部分移至句号之前。

二六四、酿两样。“抽去……扦子骨……入抽去扦子骨处”。“扦”,~《四》本作“杄”,《八》
本作“{\bfseries\scalebox{.65}[1]{林}\kern-.12em\scalebox{.55}[1]{千}}”。径改。

二六七、炸扳指。“扳指”,《四》本《八》本《重》本作“斑指”。径改。“……以防炸糊。”
“糊”,《四》本《八》本《重》本皆作“胡”。径改。

二六七、三〇八,“汆一道”。《四》本《八》本作“川一道”。径改。

二六九、清汤竹荪肝膏。“竹荪”,《四》本《八》本《重》本皆作“竹参”。径改。

二七三、旱蒸鱼。各本标题皆作“汗蒸鱼”。径改。

二九二、奶汤素烩。“莴笋”,《四》本《八》本作“窝笋”。径改。

\end{list}
\endgroup%

% vim: filetype=tex noautoindent nojoinspaces
% vim: fileencoding=utf-8 formatoptions+=m
% vim: textwidth=78 tabstop=4 shiftwidth=4 softtabstop=4
