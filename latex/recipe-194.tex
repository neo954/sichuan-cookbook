\begin{recipe}{佛手海参}

\ingredients

\ingredient{水发海参}{四两}
\ingredient{鸡脯肉}{二两五}
\ingredient{盐}{三分}
\ingredient{鸡蛋清}{四个}
\ingredient{千豆粉}{五钱}
\ingredient{味精}{二分}
\ingredient{料酒}{二钱}
\ingredient{胡椒}{二分}
\ingredient{净瘦肉}{三两}
\ingredient{熟火腿}{一两}
\ingredient{丝瓜}{一根}
\ingredient{清汤}{一斤半}
\ingredient{猪肥膘}{一两五}

\cooking

\step 选大小均匀的海参,平片成长约三寸的片(十六片),再逢中腰断成三十二片,用
刀修成前宽八分、后宽六分。在宽的一头用刀划成五个五分长的齿,再将五个齿的尖上修
成箭头形,投入好汤,加胡椒、味精、盐、料酒,煮二、三分钟,打起待用。

鸡脯肉、肥猪膘、鸡蛋清、豆粉,合搅成“糁”。鸡蛋7青 合干豆粉搅成蛋清豆粉。选约五
寸长的丝瓜刮去粗皮,洗净, 车成两张丝瓜皮,在沸水内煮过,捞入冷水浸透,将水气
沾 干,切成二粗丝,一寸五长计三十二丝。熟火腿与丝瓜丝同样 切成三十二丝。猪瘦肉
打成肉茸,用清水发起解散待用。

\step 海参用净布沾干水气,每片上抹遍蛋清豆粉,铺在墩子上,将搅好的糁分放在三十
二个海参片的窄的一端;再将火腿丝、丝瓜丝各一根横起,分别镶在三十二个糁上V然后
将海参片放在手上,一个一个地把五齿翻转来贴在糁上。齿的距离一分宽。摆在抹油的白
锑盘内,上笼蒸三分钟,取出放入大薄碗内待用。

\step 清汤在旺火上用瘦肉茸扫清,加入其余的盐、料酒、胡椒、味精,浪匀即倒入海参
碗内即成。

\features

色形美观,清淡可口,春夏适宜。

\end{recipe}

% vim: filetype=tex noautoindent
% vim: fileencoding=utf-8 formatoptions+=m
% vim: textwidth=78 tabstop=4 shiftwidth=4 softtabstop=4
