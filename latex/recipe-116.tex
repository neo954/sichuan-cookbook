\begin{recipe}[八宝全鸭]{糯米全鸭}

\ingredients

\ingredient{肥鸭一只}{三斤}
\ingredient{糕米}{二两}
\ingredient{湘莲子}{一两玉}
\ingredient{芡实}{九钱}
\ingredient{苡仁}{九钱}
\ingredient{熟火腿}{一两五}
\ingredient{金钩}{二钱}
\ingredient{扁豆仁}{九钱}
\ingredient{口蘑}{九钱}
\ingredient{盐}{一钱五}
\ingredient{料酒}{四钱}
\ingredient{胡椒面}{二分五}
\ingredient{菜油二斤}{耗两}
\ingredient{草碱}{一钱}

\cooking

湘莲子放入碗中,倒入草碱和勻,渍五分钟I然后加 水倒入锅中,放炉火上烧至水快开时,用刷把戳去莲子皮, 携出放入清水中,再用手援去余皮;随后削去顶帝,抽去 心。扁豆淘洗干净,加水倒入锅中煮十五分钟,剥去皮。糯 米淘洗干净,用清水漂五分钟。苡仁、芡实去净杂质,用温 : 水泡十五分钟。金钩加温水浸透使发胀。口茉用温水泡十分

钟,淘净泥沙,掐去足,用刀切成三分方丁。选用去皮肥瘦火 腿,同样切成三分方丁。以上八种原料,分别去干水份,一 起放入大碗中,加料酒、盐、胡椒面拌勻,上笼蒸三十分钟 出笼,称为“八宝”。

\step 鸭子宰杀后,去毛,洗净,切去足;将鸭放于案板上,在鸭颈上(鸭胸上两寸处)顺着颈项开一口,长约二寸半;在咽喉刀口处切断颈颡骨,使鸭头与鸭皮整个相连;再从刀口处剔出项骨。然后将鸭子尾部向下立放案板上,将鸭皮连肉翻着往下退,同时从颈颡骨尽头处开始用刀剔除骨头,一直剔到鸭尾;除两翅不剔骨外,其余骨头须全部剔出不用(在剔骨过程中必须保持鸭肉的完整,不要弄破鸭皮)。然后将鸭翻过来(因为在剔骨时鸭皮翻在里面,肉在外面,故须翻过来),使鸭皮朝外,成为一只无骨的完整全鸭。再从蒸笼内取出八宝掐全部装入鸭腹内,在刀口处将鸭颈皮打一个结,以免腹中的馅漏出。然后放入汤锅中烫三分钟后捞出,再入清水中,摘尽茸毛。用料酒、盐、胡椒面等和匀,抹满鸭的全身。将鸭腹朝下,鸭背向上放于大蒸碗中,鸭头放于鹎背两翅盼中间,上笼蒸一点半钟,出笼晾干水气。

\step 菜油在锅内烧至将见油烟时,将鸭子放入油中炸五分钟,炸至皮酥,并呈金黄色时起锅。将鸭照原形摆入盘中上席。

\notes

此菜色泽金黄,形如全鸭、大方美观,外酥内粑,属席 桌大菜之一。

\end{recipe}

% vim: filetype=tex noautoindent
% vim: fileencoding=utf-8
% vim: textwidth=78 tabstop=4 shiftwidth=4 softtabstop=4
