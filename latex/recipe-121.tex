\begin{recipe}{川双脆}

\ingredients

\ingredient{肚子}{一个}
\ingredient{月君肝}{四个}
\ingredient{猪瘦肉}{三两}
\ingredient{清汤}{一斤半}
\ingredient{味精}{三分}
\ingredient{胡椒面}{三分}
\ingredient{盐}{三分}
\ingredient{酱油}{二钱}
\ingredient{香菜一}{五钱}

\preparation

\step 扯下靠肚尖子一寸长的肚头,撕洗干净,再把油筋剽干净,片成长一寸五分、宽六
分的片子两片;靠两长边约二分宽处划穿两根长约一寸二的直线,逢中再割一刀,以便入
味。腊肝切成两瓣,剽平两边,逢中划透约六分长的直线; 将肚片顺折穿入腊肝裂缝,
再将肚片的一端折转,穿入肚片 另一端的裂缝中,使之相互穿稳不脱,用清水漂起。香
菜掐留叶子,与胡椒面分别装入小碟。

\step 双脆泌去清水装入荷叶碗,汤在锅内烧沸,用汤瓢舀汤在碑内,将双脆冒一下,连
续冒三次,双脆已有五成熟,然后先搭盐用瘦肉扫汤,等汤清好后再将双脆冒一下,这时
双脆已有七成熟,汤内加味精,起锅加酱油提色,灌入双脆碗内,随同香菜、胡椒面碟子
上席。

\features

清淡脆嫩,味美汤鲜。

\end{recipe}

% vim: filetype=tex noautoindent
% vim: fileencoding=utf-8 formatoptions+=m
% vim: textwidth=78 tabstop=4 shiftwidth=4 softtabstop=4
