\begin{recipe}{锅贴豆腐}

\ingredients

\ingredient{石膏豆腐}{六块}
\ingredient{猪肥膘肉}{七两五}
\ingredient{鸡蛋}{三个}
\ingredient{干豆粉}{四钱}
\ingredient{味精}{二分}
\ingredient{蕃恭酱}{六钱}
\ingredient{生菜}{二两}
\ingredient{盐}{二分}
\ingredient{面粉}{二分}
\ingredient{熟瘦火腿}{三钱}
\ingredient{化猪油}{二两}
\ingredient{香油}{三钱}
\ingredient{醋}{三钱}
\ingredient{白糖}{三钱}

\preparation

仫猪肥膘肉在汤锅内煮八分熟,捞出晾冷,切成一寸二 长、一寸宽、一分厚的片二十四
片,用刀尖在每片的四角及 中心轻轻划穿(这样贴时肉片才不会卷起)生豆腐放入丝 箩
内搅散、滤去渣,取豆腐汁装入碗内。鸡蛋清用竹後在八 寸盘内用力向着一个方向快速
搅打,搅打至全部变为雪白成. 团的细泡,将筷子一只插入细泡内不倒时为止。推盐、干
豆 粉、面粉、味精和鸡蛋泡都加入豆腐汁的碗内,搅匀后再加 入化猪油搅匀,成为“豆
腐糁”。用鸡蛋清与干豆粉调成蛋: 清豆粉。

\step 熟肥猪肉片二十四片分开摊在盘内,用拧干的热布帕把油沾干,将每个肉片上抹上
一层蛋清豆粉;再用调羹将豆腐糁舀在上面,平摊约三分厚(四边稍薄,贴时才不会流)
。鸡蛋黄(三个)搅散.倒入猪油抹过的炒锅内(用猪油很少,使蛋不巴锅即可),于旺
火上摊成蛋皮。蛋皮及痩火腿先各切成细丝,再切成细末,按在摊好的豆腐糁上,每片放
上火腿及蛋皮末各一半,然后在盘内摆好,上笼蒸约五分钟即熟,取出。

\step 炒锅放在旺火上烧至七成热,倒入猪油将锅转动,使锅内粘上一层油,即泌去油,
随将蒸好的豆腐(肥肉片向下)逐个贴在锅内约煎八分钟。煎时将锅左右前后移动,以.
免糊锅。煎至肉底呈现金黄色、豆腐呈浅黄色时,淋入香油, 将炒锅转动一下即起锅。
把有火腿末的一面向上摆在盘的一 端;将淘洗干净的生菜掐叶去根(长约一寸二分),
用糖、 醋、香油拌合好后放在盘的另一端;然后将蕃茄酱淋在生菜 面上即成。

\features

此菜在浅黄色豆腐上微微透出火腿及蛋黄皮的红黄两 色,镶以绿色生菜,色调美观,食
之松脆香嫩清爽可口。

\end{recipe}

% vim: filetype=tex noautoindent nojoinspaces
% vim: fileencoding=utf-8 formatoptions+=m
% vim: textwidth=78 tabstop=4 shiftwidth=4 softtabstop=4
