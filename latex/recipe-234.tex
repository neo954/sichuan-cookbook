\begin{recipe}{春芽烘蛋}

\ingredients

\ingredient{鸡蛋}{五个}
\ingredient{水豆粉}{三钱}
\ingredient{咮精}{一分}
\ingredient{春芽尖}{三钱}
\ingredient{盐}{四分}
\ingredient{化猪油}{二两五}

\preparation

\step 将鸡蛋打入大碗中,再将水豆粉与清水和匀加入。甩竹筷在碗内搅打约一分钟,待
它匀和时,加入盐、味精,并将春芽尖叶剁成碎末一并放入,再继续搅打匀和。

\step 将锅置于旺火上烤热后,放入猪油二两,烧至冒烟即移于微火上〔火上薄薄盖一层
炭末,不要燃明火,以免把蛋:烤糊),将搅好的鸡蛋倒入锅中,取碗一个扣在锅中鸡蛋:
上,再分次将猪油半两从锅的周边淋下,约烘十分钟,用竹签插入鸡蛋内提起验看,已干
即是烘好。然后将扣碗换圆盘:复于锅中。泌去余油后,用汤瓢将烘蛋翻扣在盘中即成。

\features

此菜系烘烤而成,故色泽金黄,味咸香而鲜美,皮酥里 嫩。冬末舂初季节樁树始发新芽
,鲜嫩清香,故宜于此时食用)

\end{recipe}

% vim: filetype=tex noautoindent nojoinspaces
% vim: fileencoding=utf-8 formatoptions+=m
% vim: textwidth=78 tabstop=4 shiftwidth=4 softtabstop=4
