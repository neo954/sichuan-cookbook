\begin{recipe}{鸭腰莼菜}

\ingredients

\ingredient{苑菜}{半瓶}
\ingredient{鸭腰}{八两}
\ingredient{水发口笨}{三钱}
\ingredient{特级清汤}{一斤二两}
\ingredient{盆}{三分}
\ingredient{胡椒}{二分}
\ingredient{料酒}{少许}
\ingredient{味精}{二分}

\cooking

\step 先将莼菜淘去涎质,用开水汆过,放入碗内;再将鸭腰洗净,大的对剖,用清水烧
沸捞起用冷水冰起,撕去朦皮,放在清汤内烧开待用。

\step 将锅放于旺火上,倒入特级清汤,放入胡椒、料酒、盐烧开,将鸭腰及莼菜分别用
漏瓢在二汤内冒透,对镶于二碗内;再将锅内的清汤加入味精,起锅倒入二碗内即成。

\notes

此菜汤鲜味美,脆嫩叮口,适宜春夏二季。

\end{recipe}

% vim: filetype=tex noautoindent
% vim: fileencoding=utf-8 formatoptions+=m
% vim: textwidth=78 tabstop=4 shiftwidth=4 softtabstop=4
