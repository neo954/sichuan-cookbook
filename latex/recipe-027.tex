% BSD 3-Clause License
%
% Copyright (c) 2023, 2025 Quux System and Technology. All rights reserved.
%
% Redistribution and use in source and binary forms, with or without
% modification, are permitted provided that the following conditions are met:
%
% 1. Redistributions of source code must retain the above copyright notice, this
%    list of conditions and the following disclaimer.
%
% 2. Redistributions in binary form must reproduce the above copyright notice,
%    this list of conditions and the following disclaimer in the documentation
%    and/or other materials provided with the distribution.
%
% 3. Neither the name of the copyright holder nor the names of its
%    contributors may be used to endorse or promote products derived from
%    this software without specific prior written permission.
%
% THIS SOFTWARE IS PROVIDED BY THE COPYRIGHT HOLDERS AND CONTRIBUTORS "AS IS"
% AND ANY EXPRESS OR IMPLIED WARRANTIES, INCLUDING, BUT NOT LIMITED TO, THE
% IMPLIED WARRANTIES OF MERCHANTABILITY AND FITNESS FOR A PARTICULAR PURPOSE ARE
% DISCLAIMED. IN NO EVENT SHALL THE COPYRIGHT HOLDER OR CONTRIBUTORS BE LIABLE
% FOR ANY DIRECT, INDIRECT, INCIDENTAL, SPECIAL, EXEMPLARY, OR CONSEQUENTIAL
% DAMAGES (INCLUDING, BUT NOT LIMITED TO, PROCUREMENT OF SUBSTITUTE GOODS OR
% SERVICES; LOSS OF USE, DATA, OR PROFITS; OR BUSINESS INTERRUPTION) HOWEVER
% CAUSED AND ON ANY THEORY OF LIABILITY, WHETHER IN CONTRACT, STRICT LIABILITY,
% OR TORT (INCLUDING NEGLIGENCE OR OTHERWISE) ARISING IN ANY WAY OUT OF THE USE
% OF THIS SOFTWARE, EVEN IF ADVISED OF THE POSSIBILITY OF SUCH DAMAGE.
%
\begin{recipe}{芙蓉肉糕}

\ingredients

\ingredient{肥膘肉}{五两}
\ingredient{红糖}{三两}
\ingredient{白糖}{二两}
\ingredient{鸡蛋}{二个}
\ingredient{菜油}{一斤半耗二两}
\ingredient{红色素}{少许}
\ingredient{干豆粉}{二两}
\ingredient{青糖}{少许}

\preparation

\step 肥膘肉煮熟,切成长一寸五的二粗丝,下沸水锅汆过(去油),用漏瓢滤起,在盘
内晾干水气,再将蛋清豆粉与肉丝调拌均匀。

\step 将肉丝用手拌散,撒入七成火候的油锅,随即用漏瓢打起。撕散,再入八成火候的
油锅,炸脆,起锅待用。

\step 将红糖熬起泡时加青糖,将肉丝倒入。随手将锅提起来造转,再倒进搪瓷方盘。用
铲或刀按平约七分厚,面上撒一层胭脂糖(白糖用色素调成)。晾冷后切成姜糖块(即斜
方块),装入条盘。

\features

此菜约六七十年的历史,可上热食,亦可上冷盘,又可上海参席,脆甜可口,解酒佳
肴。

\end{recipe}

% vim: filetype=tex noautoindent nojoinspaces
% vim: fileencoding=utf-8 formatoptions+=m
% vim: textwidth=78 tabstop=4 shiftwidth=4 softtabstop=4
