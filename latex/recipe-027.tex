\begin{recipe}{芙蓉肉糕}

\ingredients

\ingredient{肥瞟肉}{五两}
\ingredient{红糖}{三两}
\ingredient{白糖}{二两}
\ingredient{鸡蛋}{二个}
\ingredient{菜油}{一斤半耗二两}
\ingredient{红色素}{少许}
\ingredient{干豆粉}{二两}
\ingredient{青糖}{少许}

\cooking

\step 肥膘肉煮熟,切成长一寸五的二祖丝,下沸水锅汆过(去油),用漏瓢滤起,在盘
内晾干水气,再将蛋清豆粉与肉丝调拌均匀。
\step 将肉丝用手拌散,撒入七成火候的油锅,随即用漏瓢打起。撕散,再入八成火候的
油锅,炸脆,起锅待用。
\step 将红糖熬起泡时加青糖,将肉丝倒入。随手将锅提起来造转,再倒进搪瓷方盘。用
铲或刀按平约七分厚,面上撒一层胭脂糖(白糖用色素调成)。晾冷后切成姜糖块(即斜
方块),装入条盘。

\notes

此菜约六、七十年的历史,可上热食,亦可上冷盘,又 可上海参席,脆甜可口,解酒佳
肴。

\end{recipe}

% vim: filetype=tex noautoindent
% vim: fileencoding=utf-8
% vim: textwidth=78 tabstop=4 shiftwidth=4 softtabstop=4
