\begin{recipe}{烤酥方}

\ingredients

\ingredient{生猪肉}{(约十五斤)一方}
\ingredient{蒜}{四两}
\ingredient{葱}{一斤}
\ingredient{柴}{十斤}
\ingredient{私炭}{十五斤}
\ingredient{甜酱}{四两}
\ingredient{香油}{一两}

\cooking

\step 选料及处理选厚膘连皮带肋骨肉一方,横、长一 尺,宽九寸。肉皮须平
整没有凸凹者。先去净 茸毛,刮洗干净。然后 肉皮向下、排骨向上放
于案板上,用直径二分 粗的尖长竹签,在排骨 缝中瘦肉上刺上若干气
眼。刺的深度以接近肉 皮为合适,但不要把肉
皮刺破(图5)。然后将它揩干水份,用铁质二股烤叉一
把,由排骨之下,肥肉之上的痩肉中叉进去(图6),叉尖 伸出肉方外约一尺左右。

图5 刺气眼

图6 叉肉

\step 出坯用干柴十斤,先用 二斤放炉内烧燃,炉中火苗燎出
炉口一尺至二尺〈须经常保持这 种火苗,火苗不均匀时加柴),
随即手拿叉柄,将肉方的皮向着 火苗,排骨向上(不要着火)在
火苗上燎。一面调剂火候,一面 手拿叉柄左右拧动。拧动的角度
是:先把肉方拧至肉皮与地平成 八十五度后,'再拧过去成同样的
角度。这样反复来回拧动。拧动 的速度是:中间稍稍快,两端稍
停。着重燎肉方的四周和四角,

燎至肉皮上的毛眼中好像在沸腾。同时猪皮上比较粗老的皮
被烤成一层很薄的黑壳,自行整张地脱落于炉火中〈若有的
地方没有落,就说明那里的火候还不够),然后将肉方挪开
炉火,用净布擦净叉尖,取下放入烫水中,用布帕洗净。通
过出坯,肉皮比原肉皮薄,保留的肉皮约半分厚,皮上微微
现出蜂窝形状的花纹,带牙黄色,本道工序即成。待本菜上
席前二十分钟再进行烤酥的工序。

图7 烤酥

\step 烤酥用青砖二十四块,在平坦的地坪上嵌一烤池,
用细煤渣灰五斤平铺在烤池底,将杠炭八斤烧红,平放(不
要立放,立放便要伸出 火苗,烤酥时严忌火 苗)在烤池之中。再将
肉方由原叉眼中叉好, 放于烤池上,肉皮向 火,排骨向上(图7〕,
同样手拿叉柄左右拧 动、拧动的角度与出坯 相同,但速度稍快。烤
至肉方出油时,将烤池 内的红杠炭检于烤池四 周(前后两端多放一
些,去净池心的火星, 以免肉方的油滴于火上,引起火苗),继续拧动着烤。此时肉
方皮上的油很多,为了使它在肉皮上流来流去,把皮烫酥,
而不掉下来,故拧动速度宜再快些。烤至肉皮成金黄色时,
可以用刀尖试验一下是否酥泡。试法是:用手指捏着刀叶,
只留一分长的刀尖在外,将肉方拿离烤池,用刀尖在肉皮上

连锥几下,如发出酥泡的响声为合格。然后将香油三钱刷于
酥皮之上,用净布擦净叉尖,取下酥方,酥皮向上放于大圆 盘中。

\step 铲皮上席用刀尖在酥皮下平铲,将铲下的酥皮切成
一寸半长、七分宽的条方片,照原样摆于肉方之上。另外,
给每位顾客配汤盃一个,内盛加味特级清汤①(二两三
寸手碟一个,内盛蒜片、甜酱(甜酱中加香油二钱搅匀)、
葱白段(一寸半长)三段;再配五寸平盘一个,内盛点心两 块,连同穌方一同上席。

\step 吃法一般只吃酥方皮,不吃肉。肉可用来做回锅 肉,别有风味。

\notice

烤方需要一定技术,操作中稍不注意,就会发生质量事
故。易于发生的质量事故及其补救方法如下:

\step 方皮鼓泡鼓泡在“出坯”和“烤酥”中都会发生。
其原因是气眼刺得不好,或拧动角度太大。补救方法是:把
肉方拿离烤池,用尖头竹签在鼓泡的附近由痩肉中刺眼放
气,若鼓泡太大,可用刀将鼓泡的皮割去,抹上蛋清豆粉, 稍抹厚点,再继续烤。

\step 硬皮发生硬皮的原因是“出坯”时火候不匀,或烤
酥时拧动角度太小。补救方法是:将肉方拿离烤池,在池中
检红杠炭一块〈炭的大小与硬皮相等),逼近硬皮处烤至微
焦,再用小刀将烤焦处之皮刮去一层,继续上烤池烤。

\step 烂皮发生烂皮的原因是出坯时拧动速度太慢,以致
肉皮被烧烂。补救方法:用蛋清豆粉抹烂皮处,厚厚地抹一 层,再继续上烤池烤。

\step 漏油在选料或刺气眼没有注意,皮上有眼就会漏
油。在烤酥时,可用蛋清团粉抹严,再继续烤。

\notes

此菜一般是高贵筵席的配菜,色彩金黄,美观大方,吃
时酥香,脂肪特多,而爽口不腻。

①特级清汤

(一)原 料

老母鸡	一只(三斤〕	老肥鸭	一只(三斤〕
火腿蹄子	一只	火腿棒子骨	一斤
排骨	二斤	鸡脯肉	三个
猪生瘦肉	五斤	清水	三十斤

(二)制作方法

1.	将鸡、鸭宰杀,去毛,去内脏9蹄子、棒子骨刮洗干净。生瘦肉、蟢 脯肉都分别用刀背捶成茸。

2,	汤锅放炉上,将棒子骨、排骨、蹄子、鸡、鸭依次放入汤锅中,倒入 清水二十五斤,用旺火烧开,去净泡沫,炖一小时而后捞出,漂于温热水 中。用猪肉茸一斤,加清水一斤兑匀调散,倒入汤锅中;待瘦肉和泡沫浮起 时,用漏瓢打净。将鸡、鸭、蹄、骨等用热水洗净,放于汤锅中,用微火焙半 小时,将鸡、鸭捞出另作别用,再将各骨捞出漂溫水中。然后将猪肉茸四斤, 加清水三斤调散倒于汤锅中;待肉茸浮起时,用漏瓢把它挤成四至五个肉饼. 再将汤面浮油吹净,将各骨用温水洗净,轻轻放入汤锅中;随后将肉饼放于各 骨之上,用小微火焙。此时汤已变色,很象料酒颜色。鸡茸内加清水一斤调散. 待使用清汤时,再将鸡倒入汤中,待泡沫和肉茸浮起后,将泡沫和肉茸打去, 即成清汤.

\end{recipe}

% vim: filetype=tex noautoindent
% vim: fileencoding=utf-8
% vim: textwidth=78 tabstop=4 shiftwidth=4 softtabstop=4
