\begin{recipe}{烧牛头方}

\ingredients

\ingredient{水牛脑顶肉(不宜用黄牛)}{六斤}
\ingredient{冰糖色}{七钱}
\ingredient{母鸡汤(炒菜柏)}{三杓}
\ingredient{肉汤(炒菜杓)}{三杓}
\ingredient{鲜菜(或心)}{一斤}
\ingredient{绍酒}{一两二}
\ingredient{生鸡油}{一两二}
\ingredient{姜、葱}{少许}
\ingredient{食盐}{少许}
\ingredient{小磨麻油}{少许}
\ingredient{猪油}{五钱}

\preparation

烧牛头系取用其皮子。先将脑顶肉在炉火上将毛烧掉,

刮洗干净,用清水以文火炖约五小时,取出削净毛眼和肉取其皮子。再将皮子用清水以文
火炖至七成火候时(约需六小时),即取出洗净,切成骨牌块状,用沸水连续微煮三次
(厨称出水三次),以去其胶质和骚气。

猪油在旺火上煎辣,即放入葱姜快炒一、二铲,掺进肉汤,俟汤沸即取出葱姜不要。随即
放入牛皮移置文火上烧约一小时,捞起滤干汤汁,置锑锅内,加进鸡汤、绍酒、生鸡油、
食盐和冰糖色等,以文火烧熟为止(约一小时)。起锅时取出生鸡油渣不要,牛皮转入另
一耳锅里用武火收稠汤汁,淋上小磨麻油,盛于盘内,配上刚炒好的鲜菜即成。

\features

味浓厚可口,质糯而不粘,色泽光亮,营养丰富,具有特殊风味。

\end{recipe}

% vim: filetype=tex noautoindent nojoinspaces
% vim: fileencoding=utf-8 formatoptions+=m
% vim: textwidth=78 tabstop=4 shiftwidth=4 softtabstop=4
