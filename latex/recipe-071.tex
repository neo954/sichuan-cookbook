\begin{recipe}{贵州鸡}

\ingredients

\ingredient{生鸡脯肉}{六两}
\ingredient{干海椒}{五钱}
\ingredient{鸡蛋}{二个}
\ingredient{姜}{二钱}
\ingredient{葱}{五钱}
\ingredient{蒜}{一钱}
\ingredient{料酒}{五钱}
\ingredient{酱油}{三钱}
\ingredient{盐}{三分}
\ingredient{干豆粉}{八钱}
\ingredient{白糖}{五分}
\ingredient{味精}{二分}
\ingredient{化猪油}{一斤耗三两}
\ingredient{清汤}{一两}

\preparation

\step 先将鸡脯肉用刀两边相对剳成交差形的花纹,再切成四分见方的鸡丁,放入料酒、
盐拌和均匀,放入碗内待用。

\step 将干海椒去把,放在开水内发胀,春茸,再将姜蒜切成片,葱切成马耳朵形,分开
放入碗内待用。

\step 将鸡蛋及豆粉拌成蛋清豆粉,鸡丁同蛋清豆粉拌匀;再将锅放于旺火上烧红,放入
化猪油,烧至三成火后用手将调好的鸡丁抖入锅内,用竹筷滑散,扦入盘内。撇去锅内的
余油,留一两五油在锅内,放入姜、葱、蒜片及海椒茸,炒散;再放入料酒、味精、盐、
白糖、清汤、酱油,炒匀;再将盘内的鸡丁,倒入锅内,稍拨几转起锅即成。

\features

此菜色红、鲜嫩、味浓,适宜酒肴。

\end{recipe}

% vim: filetype=tex noautoindent nojoinspaces
% vim: fileencoding=utf-8 formatoptions+=m
% vim: textwidth=78 tabstop=4 shiftwidth=4 softtabstop=4
