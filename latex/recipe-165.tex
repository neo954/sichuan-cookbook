\begin{recipe}{红烧雪猪}

\ingredients

\ingredient{鲜雪猪肉}[\footnotemark]{一方三斤}
\ingredient{鸡肉}{一斤}
\ingredient{肥瘦火腿}{二两五}
\ingredient{蘑菇}{一两}
\ingredient{冬笋}{半斤}
\ingredient{特级清汤}{二斤}
\ingredient{八角}{一钱}
\ingredient{草果}{一钱}
\ingredient{姜}{一两}
\ingredient{大葱}{二两}
\ingredient{料酒}{三两}
\ingredient{𰪿糟}{一两五}
\ingredient{花椒}{十余颗}
\ingredient{胡椒}{十余颗}
\ingredient{咸红酱油}{一两三}
\ingredient{盐}{六分}
\ingredient{冰糖汁}{六钱}
\ingredient{味精}{三分}

\preparation

\step 蘑菇用温水泡十分钟,刮净泥沙,淘洗干净,切成约三分大、五分长的条块。冬笋
去壳,去老根,只用嫩尖,横切成三分厚、五分长的条块。火腿刮洗干净,切成三分厚、
一寸六分长的条块。猪肉和鸡肉先去净茸毛,刮洗干净,切成二寸见方的块。姜(拍松)
和大葱洗净,连同花椒、胡椒、草果、八角等用稀眼布包好待用。

\step 炒锅放炉上,放入清水,将雪猪肉、鸡块一起放入锅 中煮二、三分钟,去净血水
泡沫,然后捞出,在清凉水中漂 冷后,清洗干净待用(最好多汆几次)。

\step 砂锅放炉上,放入特级清汤,再将料酒、𰪿糟、姜葱布包、红白酱油、盐等全部放
入烧开。然后将切好的鸡块、蘑菇、冬笋、火腿、雪猪肉等放入,用旺火烧开,再用草纸
固封锅口。随后改用微火煨二小时,这时各种原料都已𤆵烂,撕去锅口草纸。取出葱、姜
布包不用;用漏瓢将菜舀入大圆盘中,雪猪盖在面上。再将炒锅中的原汁加味精,用旺火
熬酽,淋于菜上即成。

\features

此菜色红发亮,肉肥味浓,脂肪特多。

\footnotetext{
雪猪:俗名土拔鼠。产于四川高原和西藏地区。肉白嫩,脂肪多,故当地居民称为“雪
猪”。
}

\end{recipe}

% vim: filetype=tex noautoindent nojoinspaces
% vim: fileencoding=utf-8 formatoptions+=m
% vim: textwidth=78 tabstop=4 shiftwidth=4 softtabstop=4
