\begin{recipe}{麻酥鸡}

\ingredients

\ingredient{净鸡肉}{八两}
\ingredient{菜油}{一干耗三两}
\ingredient{生菜}{二两}
\ingredient{芝麻}{二两}

\ingredient{慈菇}{二两}
\ingredient{白糖}{二分}
\ingredient{料酒}{五钱}
\ingredient{鸡蛋清}{二个}
\ingredient{醋}{二分}
\ingredient{盐}{一钱}
\ingredient{干豆粉}{一两}
\ingredient{胡椒面}{二分}
\ingredient{香油}{二分}

\cooking

\step 鸡肉剔去粗筋,慈菇去皮淘净,分别用刀剁成细泥;芝麻用温热水泡十五分钟后,用手搓揉,使壳在水面浮起,用漏瓢打去,将余下芝麻在热锅上炕成浅黄色至熟。鸡蛋清调成蛋清豆粉,同鸡泥、慈菇泥加盐、胡椒面、香油、料酒共拌成谄。

\step 先用芝麻一半在九寸盘内铺匀,再放拌好的焰于盘内,用手拍压平整约四分厚,剩余一半芝麻均匀地撒在上面成半制品。

\step 炒锅在中火上先放菜油四两,烧至四成热,即将盘内加工好的半制品由锅边轻轻梭下,炸至进皮,泌去油,将锅一簸,翻面再炸。为了保持四至五成的油温,用瓢子逐渐加添新油,这样使油浸透、炸酥而又避免炸糊。

\step 起锅时用刀改成长一寸五、厚约二分的片,用条盘盛装,摆成二叠水,一端懷入拌好的生菜,吃时蘸椒盐、葱酱。

\notes

香酥脆嫩,风味别致。

\end{recipe}

% vim: filetype=tex noautoindent
% vim: fileencoding=utf-8
% vim: textwidth=78 tabstop=4 shiftwidth=4 softtabstop=4
