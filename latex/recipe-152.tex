\begin{recipe}{包烧鱼}

\ingredients

\ingredient{活敛鱼一条}{(约一斤二两)}
\ingredient{鸡蛋清}{三个}
\ingredient{姜}{二钱}
\ingredient{化猪油}{一两五}
\ingredient{酱油}{一钱五}
\ingredient{泡辣椒}{二根}
\ingredient{鲜肉}{二两}
\ingredient{生鲜菜}{一束}
\ingredient{干豆粉}{七钱}
\ingredient{香油}{七钱}
\ingredient{猪网油}{一斤}
\ingredient{料酒}{1一'两五}
\ingredient{盐}{一分}
\ingredient{芽菜}{三钱}
\ingredient{葱}{二钱}

\cooking

\step 鲤鱼去鳞去腮,剖腹去脏,切去头尖、尾尖,鱼的二面划上梯块形,揩干水。把料
酒、酱油、盐、姜、葱等同放碗内拌匀,将鱼浸渍约五分钟,取出揩干待用。

\step 芽菜洗净沙渣;猪肉选用肥瘦的,洗净,去皮;泡辣椒去籽,分别剁成细茸末。锅
内放入猪油,烧红,将肉末加入,炒熟后塞入鱼腹。竹筷一支前端削尖,从鱼嘴平穿插入
鱼尾,以免烧时鱼身下垂。

\step 猪网油平铺开,用刀修整齐。鸡蛋清与干豆粉混合调 匀成为蛋清豆粉。把网油除
留出约八寸长、五寸宽的面积 外,其余全部涂上蛋清豆粉。然后将穿好的鱼用网油包三
至 四层,未涂豆粉部分的网油包最内一层。包好后用小叉一柄 从鱼腹部刺进(估计两端
分量相等〕,由鱼背刺出,放杠炭火上翻转着烤约三十分钟,至外面呈金黄色时即已酥透
。然 后将叉尖揩净抽出,用刀划破网油将鱼取出,抽出竹筷;网 油最内一层摘去不用,
其余切成二寸长、八分宽的长方形 片,镶于盘中鱼侧;把鲜生菜洗净,切碎,镶摆盘中
另一角即成。

\notes

此菜味鲜而浓,酥脆芳香,宜于下酒。另外,烹制“包I 烧鸭”“包烧鸡”时也适用这种作法。

\end{recipe}

% vim: filetype=tex noautoindent
% vim: fileencoding=utf-8 formatoptions+=m
% vim: textwidth=78 tabstop=4 shiftwidth=4 softtabstop=4
