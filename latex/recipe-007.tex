\begin{recipe}{水煮肉片}

\ingredients

\ingredient{猪里脊肉}{五两}
\ingredient{混合油}{二两二}
\ingredient{干辣椒}{二钱}
\ingredient{姜}{二分}
\ingredient{花椒}{二分}
\ingredient{葱白}{三钱}
\ingredient{盐}{一分}
\ingredient{豆瓣}{三钱}
\ingredient{味精}{一分}
\ingredient{酱油}{三钱}
\ingredient{干豆粉}{三钱}
\ingredient{料酒}{一'钱}
\ingredient{白菜(嫩叶)}{二两}
\ingredient{鸡蛋清}{一一}
\ingredient{胡椒面}{一分五}
\ingredient{清汤}{半斤}

\cooking

\step 里脊肉放在菜墩上,横着肉纹切成长一寸、宽八分、 厚半分的薄片,加上盐、料酒及蛋清豆粉,同时和匀。混合 油放在锅内于旺火上烧至七成热,将花椒、干辣椒倒入炸成 金红色,捞起剁成碎末,泌去炸油不用。白菜淘洗干净,切 成比肉片稍大的碎块。姜去皮,切成二分见方的薄片。葱白 切成八分长的段。

\step 	混合油放在锅内于旺火上烧至刚冒烟时,将豆瓣放入 炒酥,依次迅速地放入白菜、葱、姜、清汤、酱油、胡椒面、 味精,用汤瓢稍一搅转,即用手将肉片抖散放入,随即用汤 瓢搅动让它散开,煮约三分钟便熟,舀在碗中。

\step 	将剁碎的干辣椒、花椒末撒在上面。再用汤瓢将混合 油烧沸淋于面上。它的作用是:用沸油把干辣椒、花椒末、 肉片再炸一下,使之有更浓的麻辣味。

\notes

此菜依据盛行于川南一带的水煮牛肉的制法,改用猪肉 为原料。色深味厚,香味浓烈,肉片鲜嫩,突出川味麻辣烫 的特点。肉片下锅不是直接用油炒,而是用汤煮出来的,故 名水煮。在菜肴中别具一种风格。

\end{recipe}

% vim: filetype=tex noautoindent
% vim: fileencoding=utf-8
% vim: textwidth=78 tabstop=4 shiftwidth=4 softtabstop=4
