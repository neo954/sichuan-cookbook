\begin{recipe}{樟茶鸭子}

\ingredients

\ingredient{肥公鸭一只}{三斤}
\ingredient{味精}{二分}
\ingredient{耢糟}{一两}
\ingredient{花茶}{一两}
\ingredient{香樟树叶}{一两}
\ingredient{葱}{二两五}
\ingredient{香油}{二钱}
\ingredient{花椒}{约二十粒}

\ingredient{胡椒面}{一'钱}
\ingredient{料酒}{一两}
\ingredient{菜油}{二斤耗二两}
\ingredient{組酱}{-一钱}

\ingredient{盐}{一两}
\ingredient{硝}{一钱}
\ingredient{柏枝}{四两}
\ingredient{锯末}{四两}

\cooking

\step 	将鸭宰杀去毛,清洗干净,再用铁皮夹子摘净茸毛, 然后放于案板上,用刀在颈项上接近鸭背处开一口,长约一 寸,由此处将鸭的软喉及鸭朦扯出。再在鸭背离尾部尽头处 横割一刀,约二寸长,由此刀口中将鸭的脏腹挖取干净,再 用水洗净,晾干水。葱去叶和须,用葱白,切成一寸半长的 段,盛盘中待用。

\step 	用碗将花椒、盐、胡椒面拌匀,抹于鸭腹之中。再将 料酒和耢糟拌匀,涂在鸭皮上。余下的抹于鸭腹之中。然后将 鹎放盆内醃十二小时,晾干水。

将花茶、锯末、柏枝、香樟树叶一起和匀,平均分为 三份。把木盆一个(高约四寸)放平地上,将茶叶、锯末等 一份,放入一土碗中,再从炉内夹一段烧红的炭放在茶叶、 锯末上面,将碗放入木盆中央。木盆口上平放一张稀眼铁丝 网,鸭子放铁丝网上,另用大盆一个盖着,用茶叶锯末所燃 起的烟熏。熏十分钟后,揭开大盆,取出土碗,加茶叶锯末 一份,另换烧红炭一段放回盆中,将鸭翻过来,盖上大盆, 再熏七分钟。然后再将大盆揭开,取出土碗,将余下一份茶 叶锯末放入,另换红炭,将碗放入木盆中,再将鸭色浅的一 面向下,盖上大盆熏五分钟。此时鸭色呈现出深黄色,将鸭 取出,放入大蒸碗中,上笼蒸三小时,出笼晾冷待用。

\step 锅放炉上,倒入菜油烧热,再将鸭放入炸五分钟,至 鸭皮已酥捞出。将鸭放于案板上,先剁下鸭颈,切成六分长 的段,放于大圆盘中央;再将鸭身劈成两半,每半顺切成两 块,再横切成一寸宽的块,鸭皮向上盖于盘中鸭颈块上,并

将鸭摆成原形。

\step 香油一钱与甜酱拌匀,平均分成二份,分别摆在盛鸭 的条盘两端。葱白段亦分成二份,照甜酱摆法摆入盘中,再 将余下的香油淋于鸭块上上席。

\notes

\notes

\end{recipe}

% vim: filetype=tex noautoindent
% vim: fileencoding=utf-8
% vim: textwidth=78 tabstop=4 shiftwidth=4 softtabstop=4
