\begin{recipe}{口袋豆腐}

\ingredients

\ingredient{豆腐}{十个}
\ingredient{特级奶汤}{一斤半}
\ingredient{胡椒面}{一分}
\ingredient{味精}{二分}
\ingredient{料酒}{一钱}
\ingredient{盐}{六分}
\ingredient{开水}{一斤}
\ingredient{二汤}{五两}
\ingredient{草碱}{二钱}
\ingredient{菜油}{二斤耗二两五}
\ingredient{鸡油}{一钱五}

\preparation

\step 豆腐去皮,切成一寸五长、六分见方的条方块,将所有的豆腐分成三份。

\step 用锅两口,分别放在两个火眼上,其中一口放入开水一斤,加草碱,其火候要保持
开水的热度又不至沸腾。另一口锅烧红,倒入菜油烧开,取豆腐一份放入,炸至金黄色时
用漏瓢捞出,投入草碱开水中泡四分钟,捞出放在清水中。再取豆腐一份入油中炸好捞
出,放入草碱开水中泡五分钟,捞入清水内。再取豆腐一份入油锅中炸好,放入草碱开水
中泡六分钟,捞出放入清水内。这种作法是使豆腐变成空心豆腐。但不能使豆腐全部空
完,可以用手指按一按,若空完则是碱大了,没有空则是碱小了。

\step 将豆腐在开水中汆一、二次后放入碗中,再用二汤汆一、二次。将特级奶汤舀入锅
中,加胡椒面、味精、料酒、盐等烧开,然后放入豆腐,搅匀起锅,盛在碗内。将鸡油淋
于汤中上席。

\features

此菜汤鲜、味浓、豆腐嫩。若加入鸡皮、笋衣、绿色鲜菜,即成为“三鲜口袋豆腐”。

\end{recipe}

% vim: filetype=tex noautoindent nojoinspaces
% vim: fileencoding=utf-8 formatoptions+=m
% vim: textwidth=78 tabstop=4 shiftwidth=4 softtabstop=4
