\begin{recipe}{鸡蒙兰花}

\ingredients

\ingredient{鸡脯肉}{四两}
\ingredient{鸡蛋清}{三个}
\ingredient{千豆粉}{五钱}
\ingredient{兰花}{三十件}
\ingredient{盐}{五分}
\ingredient{料酒}{二钱}
\ingredient{蜂精}{三分}
\ingredient{胡椒面}{二分}
\ingredient{特级清;务}{一斤半}
\ingredient{熟火腿}{二两}
\ingredient{肥膘肉}{一五}

\preparation

\step 兰花去茎抽心,火腿切成两寸长二祖丝三十根,每朵兰花内灌入一根,灌好备用。

鸡脯肉用刀背捶茸,剔去筋缠,平均分成两份;先用一 份加冷汤二两,兑起待用;鸡蛋清
调散;干豆粉用冷水调匀,

先倒入鸡茸内,用力搅匀搅稠;加肥膘肉再搅;再加盐、味精、 胡椒面、料酒少许,再搅
匀至不干不清时,即成“老糁”待用。

锅内清水烧沸,用手将灌好火腿的兰花,逐个在碗内 蒙上一层老糁;蒙时露出花尖部分
,边蒙边下入锅内,边浮 起即边打捞入清水内漂起。

\step 清汤在锅内烧沸,再将兑好的一份鸡茸将汤扫好,再将蒙好的兰花捞起滤干入锅,
汤再开后加味精、盐、料酒、胡椒面等目入碗内,起锅即成。

\features

色鲜味美,汤清香。

\end{recipe}

% vim: filetype=tex noautoindent
% vim: fileencoding=utf-8 formatoptions+=m
% vim: textwidth=78 tabstop=4 shiftwidth=4 softtabstop=4
