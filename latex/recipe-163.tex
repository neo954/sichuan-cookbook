\begin{recipe}{五色虾球}

\ingredients

\ingredient{尽鲜虾仁}{半斤}
\ingredient{熟瘦火腿}{一两}
\ingredient{金钩}{三钱}
\ingredient{水发木耳}{一两}
\ingredient{慈菇}{一两}
\ingredient{鲜青豆}{二两}
\ingredient{鸡蛋}{二个}
\ingredient{猪肥膘肉}{二两五}
\ingredient{水豆粉}{三钱}
\ingredient{胡椒面}{二分}
\ingredient{盐}{六分}
\ingredient{料酒}{二钱}
\ingredient{猪瘦肉}{二两}
\ingredient{味精}{三分}
\ingredient{特级清清}{一斤半}

\preparation

\step 先将虾仁淘洗干净,放于墩子上,用刀背捶成细茸,放于碗内。将猪肥膘肉剁成细
茸(如油脂),放于另一个硫内。再将猪瘦肉捶茸装入另碗用清水调散待用。

\step 将火腿、金钩、木耳(淘净后挤干水份)、慈菇(去皮剁细后用帕子挤干)、青
豆(用沸水沮后用清水冰冷,剁细后挤干。无青苴时可用其他代用,但要青色),分别剁
成细末,分开放入盘内。

\step 将捶好的虾仁用少量清水调散,放下二个蛋清,再放 下剁好的猪肥膘肉、盐、水
豆粉。每放下一样品种时都要搅 上数十转,只能顺着搅,不能反顺搅;大约搅至三、四
百转时挤成桂元形;置于水中能浮于水面,颜色为白而带宝色,不带一点杂质即成“糁”。

\step 将桂元形的糁,每六个在上面一种细末内滚一转,五种细末,共三十个虾球,滚好
入盘内上笼大火蒸至五分钟取出。

\step 将特级清汤倒入锅内,放入胡椒面、料酒、盐,投入捶好的莺子,汤微开,肉沫浮
起打尽,装入一个碗内无杂质。将蒸好的五色虾球翻入碗内,放下味精,将清好的汤倒入
大碗内即成。

\features

此菜颜色美观、鲜嫩、四季可口,而夏季最佳。

\end{recipe}

% vim: filetype=tex noautoindent nojoinspaces
% vim: fileencoding=utf-8 formatoptions+=m
% vim: textwidth=78 tabstop=4 shiftwidth=4 softtabstop=4
