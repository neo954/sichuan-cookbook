\begin{recipe}{红烧卷筒鸡}

\ingredients

\ingredient{净鸡脈脯肉}{七两五}
\ingredient{熟火腿}{二两}
\ingredient{剥净鲜笋}{二两}
\ingredient{水发鸡松}{二两}
\ingredient{鸡蛋清}{四个}
\ingredient{千豆粉}{一两五}

\ingredient{菜油}{二斤耗四两}
\ingredient{味精}{二分}
\ingredient{酱油}{五钱}
\ingredient{红酱油}{三钱}
\ingredient{料酒}{二钱}
\ingredient{盐}{二钱}
\ingredient{姜}{二钱}
\ingredient{清汤}{一斤半}
\ingredient{葱段}{二钱}

\cooking

\step 鸡腿从关节处下成两节,去骨,并鸡脯片成长约一寸三、宽八分的薄片二十四片至
二十八片,片好后将边沿修切 整齐;火腿、鲜笋、鸡松分别切成长七分的二粗丝二十四
根 至二十八根。

把每张鸡片摆好后,将火腿、鲜笋、鸡松丝各一根理 伸放于鸡片的一端,用手顺着裹成
卷形,卷尾的一端抹上蛋 清豆粉交口,每个卷再抹上一层蛋清豆粉。

\step 菜油二斤在锅内烧至八成热时,即将裹好蛋清豆粉的鸡卷逐个顺锅边放入油内,用
瓢子翻动,炸至见黄色后用漏瓢打起。

\step 用小锑锅一个,先在锅底放入猪骨或鸡骨数节,即将清汤、红白酱油、葱、姜、料
酒、盐及鸡卷依次放入锅内,用文火烧三十分钟,即将鸡卷拈入二鱼碗,摆成“三叠水”定
好;再将汁水倒入,去掉姜、葱、骨节,上笼蒸约十分钟取出,汁水泌入锅内9鸡卷翻入
盘中

\features

色泽金黄,肉质细嫩,味极鲜美。

\end{recipe}

% vim: filetype=tex noautoindent
% vim: fileencoding=utf-8 formatoptions+=m
% vim: textwidth=78 tabstop=4 shiftwidth=4 softtabstop=4
