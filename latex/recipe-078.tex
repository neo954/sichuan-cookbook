\begin{recipe}{福建仔鸡}

\ingredients

\ingredient{嫩仔鸡一只}{二斤半}
\ingredient{花椒}{六分}
\ingredient{姜}{二钱}
\ingredient{葱}{五钱}
\ingredient{料酒}{三钱}
\ingredient{盐}{七分}
\ingredient{酱油}{五钱}
\ingredient{白糖}{三钱}
\ingredient{胡椒面}{三分}
\ingredient{味精}{三分}
\ingredient{耢糟}{五钱}
\ingredient{香油}{一‘两}
\ingredient{清汤}{四两}
\ingredient{菜油}{一斤半耗二两}

\preparation

\step 仔鸡宰杀后去血退毛,清洗干净,从背脊部砍开,挖去内脏,去掉足爪不用,再清
洗一次。再用刀尖将鸡脯及鸡腿的肉剖开,把腿、翅、骨微微宰断,用手将盐、葱、姜、
酱油、耢糟、料酒、花椒等抹在鸡的内外,抹匀,渍一小时。

\step 将锅放在旺火上,倒下菜油,烧至八、九成热时将鸡取出搽干(原汁留用),投入
油锅内炸熟炸透;泌去余油,将抹鸡时的原汁一并倒入锅内,放下胡椒、白糖、香油、味
精、清汤,将锅内的滋汁收至一半为止,然后将鸡捞起放在墩子上宰成一寸二长、五分宽
的长方块,按照鸡的原形摆入盘内

(鸡脯向上)。将锅内滋汁泌入碗内,将碗内的姜、葱、花 椒捞出剁成细末后在滋汁碗
内调匀,淋于鸡脯的面上即成。

\features

此菜鲜嫩可口,甜咸中带有麻味,四季均宜。

\end{recipe}

% vim: filetype=tex noautoindent
% vim: fileencoding=utf-8 formatoptions+=m
% vim: textwidth=78 tabstop=4 shiftwidth=4 softtabstop=4
