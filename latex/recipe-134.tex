\begin{recipe}{凉粉鲫鱼}

\ingredients

\ingredient{鲫鱼三尾}{一斤}
\ingredient{口笨豆豉}{五钱}
\ingredient{白凉粉}{一斤}
\ingredient{芽菜}{三钱}
\ingredient{菜油}{一两}
\ingredient{花椒}{一钱}
\ingredient{大蒜}{二钱}
\ingredient{姜}{五钱}
\ingredient{葱}{一两}
\ingredient{料酒}{五钱}
\ingredient{芹菜}{一两}
\ingredient{盐}{二分}
\ingredient{熟油辣椒}{二两}
\ingredient{网油}{二两}
\ingredient{酱油}{一两}
\ingredient{味精}{三分}
\ingredient{香油}{三钱}

\cooking

\step 鱼剖腹、去鳞、挖腮,清洗干净,抹上料酒、盐,先用网油垫在大蒸碗内将鱼放上,网油抄拢,放花椒(数粒葱子、姜(拍松),上笼干蒸十五分钟。

\step 芹菜去叶留茎,连同葱子切成细花;大蒜、豆豉分别成细泥;芽菜剁成末;剩余的花椒与菜油在锅内制成花椒
油。

\step 油辣椒、豆豉、香油、酱油、蒜泥、芽菜末、葱花、

芳菜花、花椒油、味精等在碗内兑成调料。,

\step 凉粉切成四分见方的颗子,用清水一斤与凉粉一齐下锅水烧开时即将凉粉打起滤干,倒在调料碗内合转。::

\step 将笼内蒸好的鱼取出,揭去网油,拈在盘内摆好,’将-合好的凉粉倒在鱼上即成。

\notes

红亮、麻辣、香味浓,突出川味特点,地方风味浓厚。

\end{recipe}

% vim: filetype=tex noautoindent
% vim: fileencoding=utf-8 formatoptions+=m
% vim: textwidth=78 tabstop=4 shiftwidth=4 softtabstop=4
