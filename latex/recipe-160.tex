\begin{recipe}{软炸虾包}

\ingredients

\ingredient{鲜虾}{一斤}
\ingredient{豆油皮}{五张}
\ingredient{鸡蛋清}{二个}
\ingredient{肥瘦熟火腿}{一两}
\ingredient{白糖}{六分}
\ingredient{化猪油}{二斤耗二两五}
\ingredient{慈菇}{四个}
\ingredient{熟猪肥膘肉}{四两}
\ingredient{莲花白}{四两}
\ingredient{胡椒面}{一分}
\ingredient{嫩豌豆或嫩黄豆}{二两}
\ingredient{醋}{一钱}
\ingredient{盐}{一钱}
\ingredient{料酒}{二钱}
\ingredient{韭菜头}{五钱}
\ingredient{干豆粉}{五钱}
\ingredient{味精}{二分}
\ingredient{香油}{少许}

\preparation

\step 鲜虾去壳,挤出虾仁。嫩豌豆去皮。慈菇洗净去皮。猪肥膘肉、火腿去皮,都切成
一分方丁莲花白去筋去茎,切成一分宽的细丝,漂在清水中待用。豆油皮切成三寸见夺的
片。

\step 将鸡蛋清一个与干豆粉三钱在大碗中拌匀,再将虾仁、肉丁、火腿丁、豌豆等一起
放入,加盐、料酒、味精、胡椒茴等拌勾成为馅。

\step 将切好的豆油皮平铺大平盘中,用手指粘蛋清豆粉(用鸡蛋清一个和豆粉二钱调
成)均匀地抹在豆油皮上、每张豆油皮中央放馅二钱,将豆油皮包起来,每个虾包长约
七分、直径约五分。

\features

此菜皮酥内嫩,味鲜香,为下酒佳肴;配上鲜菜食用,更别有风味。

\end{recipe}

% vim: filetype=tex noautoindent nojoinspaces
% vim: fileencoding=utf-8 formatoptions+=m
% vim: textwidth=78 tabstop=4 shiftwidth=4 softtabstop=4
