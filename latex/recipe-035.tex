\begin{recipe}{软炸蒸肉}

\ingredients

\ingredient{猪肥膘一方}{一斤}
\ingredient{鲜豌豆(或黄豆)}{四两}
\ingredient{五香粉}{半分}
\ingredient{面包}{一个}
\ingredient{花椒面}{一分}
\ingredient{耢糟}{五钱}
\ingredient{酱油}{五钱}
\ingredient{豆腐乳水}{五钱}
\ingredient{姜米}{一钱}
\ingredient{白糖}{五钱}
\ingredient{葱花}{五钱}
\ingredient{菜油}{一斤半耗一两五}
\ingredient{鸡蛋}{二个}
\ingredient{料酒}{一两}
\ingredient{甜酱}{五钱}
\ingredient{盐}{二分}
\ingredient{大米粉}{二两}

\cooking

\step 猪肉刮洗干净,切成两寸长、一分半厚的片,装在大碗内。将花椒、五香粉、酱
油、豆腐乳水、耢糟、白糖、姜米、葱花、盐、料酒在碗内兑好调匀。豌豆洗净。面包揉
成细粉。鸡蛋打破搅勻。盐、花椒合成椒盐。
\step 将兑好的调料倒在肉片碗内造匀,码二十分钟,再加米粉、豌豆,拌匀后,将肉片
摆入洗净的二鱼碗底成一封书形,面上装豌豆,上笼蒸粑,取出翻在另外的二鱼碗,拈出
肉片晾冷,豌豆仍上笼馏起。
\step 晾冷的蒸肉两面抹上搅好的鸡蛋,再挨个地两面沾满面包粉,下入八成热的油锅
内,炸呈余黄色捞起,装在条盘中心。
\step 将豌豆取出,装在蒸肉条盘的一端,另端镶椒盐及白糖两味即成。

\notes

香酥味鲜,可甜可咸,别具风味。

\end{recipe}

% vim: filetype=tex noautoindent
% vim: fileencoding=utf-8
% vim: textwidth=78 tabstop=4 shiftwidth=4 softtabstop=4
