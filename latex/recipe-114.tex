\begin{recipe}{豆渣鸭子}

\ingredients

\ingredient{肥鸭一只}{约三斤}
\ingredient{豆渣}{七两}
\ingredient{盐}{二钱}
\ingredient{味精}{三分}
\ingredient{胡椒面}{少许}
\ingredient{化猪油}{半斤耗三两}
\ingredient{料酒}{七钱}
\ingredient{姜(拍破)}{三钱五}
\ingredient{葱(挽成结)}{三根}
\ingredient{花椒}{约十粒}

\ingredient{水豆粉}{三钱五}

\cooking

鸭子宰杀后去毛及翅足(只去翅弯及足弯以下部分), 剖腹去内脏,清洗干净,滴干血
水;以盐抹在鸭身内外,肉 厚处须多抹;并将葱、姜、花椒放在鸭腹之内,盛入蒸碗内

上笼蒸炤3

\step 将已蒸粑的鸭子取置盘中,将蒸鸭子的原汤汁倾入锅内,除去花椒、姜、葱不要,
放进味精、胡椒、料酒及盐,用汤瓢调匀,随即倾下水豆粉,调和均匀后即起锅待用。

\step 将豆渣用清水淘洗干净,挤干水份;锅内猪油用旺火烧热后,即投入豆渣翻炒,约
需十分钟,炒至豆渣呈黄色,而且不吸油只吐油时,泌去油,加进锅内的原汁拌匀,舀在
鸭子上即成。

另一作法:先将鸭子出一水,蒸粑去骨,然后上笼蒙上 纸蒸炤(其余操作过程同上)。

\features

鸭味鲜美,豆渣酥香。

\end{recipe}

% vim: filetype=tex noautoindent
% vim: fileencoding=utf-8 formatoptions+=m
% vim: textwidth=78 tabstop=4 shiftwidth=4 softtabstop=4
