\begin{recipe}{兰花土司}

\ingredients

\ingredient{土司}{一^旁}
\ingredient{肥瞟}{二两}
\ingredient{鸡蛋清}{四个}
\ingredient{捃肝}{四个}
\ingredient{鲜鱼}{八两}
\ingredient{盐}{四分}
\ingredient{味精}{二分}
\ingredient{干豆粉}{五钱}
\ingredient{香油}{五钱}
\ingredient{葱白}{一两}
\ingredient{料酒}{二钱}
\ingredient{白糖}{一钱}
\ingredient{菜油}{^一斤耗一两五}
\ingredient{鉗酱}{三钱}

\cooking

土司切成三分厚片子〈共四片〉,鲜鱼打成“鱼糁”, 每个土司片上先抹蛋清豆粉,再涂上鱼糁,用手抹平整。

\step 腊肝四个去掉两面皮子4每个割为六个鸡冠花的形式 (共二十四个〉,用盐、料酒少许,抹匀后分别插在土司上 (每片六个分两边插〉,上笼蒸五分钟取出晾冷,每片土司 再开为六片待用。

菜油在旺火上烧至七成火候,将土司倒入锅内炸至浅 黄色,将油泌去,淋上香油,簸匀起锅入盘。

\step 白糖、香油、葱白、甜酱,兑成葱酱,镶入盘另一端 入席。

\notes

美观,酥脆,席桌行菜,下酒最宜。

\end{recipe}

% vim: filetype=tex noautoindent
% vim: fileencoding=utf-8
% vim: textwidth=78 tabstop=4 shiftwidth=4 softtabstop=4
