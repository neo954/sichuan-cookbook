\begin{recipe}[金钱鸡塔]{鸡塔}

\ingredients

\ingredient{鸡脯肉(去皮)}{二两五}
\ingredient{白头韭菜}{二两五}
\ingredient{生猪肥膘肉}{一两二}
\ingredient{盐}{一分}
\ingredient{熟猪肥膘肉}{一方一斤}
\ingredient{醋}{一钱三}
\ingredient{熟瘦火腿}{二钱}
\ingredient{猪油}{一钱}
\ingredient{鸡蛋清}{四个}
\ingredient{清水}{一两}
\ingredient{干豆粉}{三钱}

\preparation

\step 取生鸡脯肉去筋,与生猪肥膘肉分别用刀背捶成茸;将鸡茸放在瓷盆中,加清水调
散拌匀,务使合为一体,再将猪肉茸放盆中调散拌匀,然后放入鸡蛋清二个,左手抓住盆
边,右手用力将各料搅动。搅时要注意顺着一定的方向搅,不能改变方向乱搅,否则便要
出次品或废品。搅至各料合为一体,颜色白而发亮,看不出一点杂质时,加盐,再如前法
搅六、七十下。随后加入清水再搅四、五十下,便成为“鸡糁”。

\step 熟肥膘肉切成一分厚,直径一寸二的圆形片二十四片。选用鲜红透亮的瘦火腿,先
切薄片,再切细丝,最后剁成极细的末。白头韭菜去叶不用,将白头切成三分长的段,漂
入清水中。鸡蛋清与干豆粉一起拌匀成为蛋清豆粉。

\step 将熟猪肥膘肉圆片二十四片平铺在大平盘中,用净布一方在沸水内浸透,挤干水轻
轻沾净肥肉上的油汁,再用手指粘蛋清豆粉在肥膘片上厚厚地抹上一层,然后将盆内“鸡
糁”做成直径六分大的圆珠,放在肥膘圆片上,再用手指粘少许凉水将圆珠抹得圆润光
滑,随即用手指取少许火腿末放在圆珠上使它粘稳,便成为鸡塔。

\step 将锅放在炉上烧烫,将鸡塔肥膘向下贴于锅中,烙至肥膘片成金黄色起锅,摆在大
条盘中央。将韭菜白由水中捞出,滤干水放入小碗中,加盐、醋和香油拌匀;摆入盛鸡塔
条盘中的两端,再将余下的香油淋于鸡塔上入席。

\features

川菜有糁、蒙、酿、贴四大烹调法,此菜制法为其中之一种,作为高贵筵席中“八大菜”
之一。形状美观,底黄、顶红、圆珠,颜色鲜艳。吃时酥香脆嫩,味鲜美,为佐酒佳,
肴。配鲜菜更别有风味。

\end{recipe}

% vim: filetype=tex noautoindent nojoinspaces
% vim: fileencoding=utf-8 formatoptions+=m
% vim: textwidth=78 tabstop=4 shiftwidth=4 softtabstop=4
