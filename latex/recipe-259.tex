% BSD 3-Clause License
%
% Copyright (c) 2023 Quux System and Technology. All rights reserved.
%
% Redistribution and use in source and binary forms, with or without
% modification, are permitted provided that the following conditions are met:
%
% 1. Redistributions of source code must retain the above copyright notice, this
%    list of conditions and the following disclaimer.
%
% 2. Redistributions in binary form must reproduce the above copyright notice,
%    this list of conditions and the following disclaimer in the documentation
%    and/or other materials provided with the distribution.
%
% 3. Neither the name of the copyright holder nor the names of its
%    contributors may be used to endorse or promote products derived from
%    this software without specific prior written permission.
%
% THIS SOFTWARE IS PROVIDED BY THE COPYRIGHT HOLDERS AND CONTRIBUTORS "AS IS"
% AND ANY EXPRESS OR IMPLIED WARRANTIES, INCLUDING, BUT NOT LIMITED TO, THE
% IMPLIED WARRANTIES OF MERCHANTABILITY AND FITNESS FOR A PARTICULAR PURPOSE ARE
% DISCLAIMED. IN NO EVENT SHALL THE COPYRIGHT HOLDER OR CONTRIBUTORS BE LIABLE
% FOR ANY DIRECT, INDIRECT, INCIDENTAL, SPECIAL, EXEMPLARY, OR CONSEQUENTIAL
% DAMAGES (INCLUDING, BUT NOT LIMITED TO, PROCUREMENT OF SUBSTITUTE GOODS OR
% SERVICES; LOSS OF USE, DATA, OR PROFITS; OR BUSINESS INTERRUPTION) HOWEVER
% CAUSED AND ON ANY THEORY OF LIABILITY, WHETHER IN CONTRACT, STRICT LIABILITY,
% OR TORT (INCLUDING NEGLIGENCE OR OTHERWISE) ARISING IN ANY WAY OUT OF THE USE
% OF THIS SOFTWARE, EVEN IF ADVISED OF THE POSSIBILITY OF SUCH DAMAGE.
%
\begin{recipe}{酿冬菇}

\ingredients

\ingredient{干冬菇}{三两}
\ingredient{盐}{一钱}
\ingredient{熟火腿}{六钱}
\ingredient{鸡蛋清}{四个}
\ingredient{黄秧白菜心}{一两}
\ingredient{葱白}{二钱}
\ingredient{干豆粉}{一两}
\ingredient{酱油}{四钱五}
\ingredient{姜}{二钱}
\ingredient{鸡脯肉}{二两五}
\ingredient{水豆粉}{六钱}
\ingredient{化猪油}{二两}
\ingredient{胡椒面}{二分五}
\ingredient{鸡油}{六钱}
\ingredient{猪肥膘肉}{一两五}
\ingredient{味精}{三分}
\ingredient{鸡汤}{四两}
\ingredient{料酒}{一两}
\ingredient{冬笋(净)}{一两}

\preparation

\step 干冬菇选菌盖直径约八分,大小均匀的二十四朵,用清水一大碗泡半小时,至泡胀
而未全胀时泌去水(另作他用)。换清水用手指在冬菇周围凸凹不平地方掏去泥沙,淘洗
干净,用刀齐盖底切去茎(作别用);然后放在沸水中煮约十分钟,端离火口浸泡十分钟
捞出;再换清水烧沸同样再煮一次,泌去煮水不用。此时菌盖直径已发胀至一寸二分左
右,则已干净胀透。

猪油倒入炒锅内,在旺火上烧至五成热,将葱白段、姜(去皮、拍松)、盐、料酒、鸡汤
依次放入,再加入冬菇、味精、胡椒面同烧五分钟,使冬菇入味,将汤泌去,把冬菇捞入
筲箕中,将水滤干后放盘中摊开,切去茎的一面,向上摆好待用。

\step 鸡脯肉去掉外面一层白膜,在菜墩上用刀背捶成茸,边捶边用刀口刮去肉内白筋,
捶至极细为止。猪肥膘肉用刀剁成肉泥(肉泥如同油脂,但不能化油;热天为了避免在剁
时化油,可先将肥膘肉放在冰箱内冰冻后再用,因化油要影响搅成糁)。大碗内放入清
水,把鸡茸投入用竹筷顺着一定方向搅打五分钟,要搅散搅匀;随后放入鸡蛋清继续用力
同样搅打十分钟;再加入盐、味精、料酒、水豆粉,同时加入肥肉泥,用力搅打五分钟成
稀糊状(用筷蘸一点滴在清水面上不沉底),即成鸡糁。

鸡蛋清二个与干豆粉调匀成为蛋清豆粉,在盘中摆好的每个冬菇上的凹处抹上一层,再用
调羹舀着鸡糁填满,随用调羹刮平。然后装入蒸笼内(有糁一面向上)上笼蒸五分钟,熟
后取出待用(临用时才取,以保持热度)。

\step 火腿、冬笋切成长一寸三、宽三分、厚半分的片。黄秧白菜心切成一寸半大小的
块。猪油放入炒锅内,在旺火上烧至五成热,即放入切好的火腿、冬笋、黄秧白菜,用汤
瓢翻搅两下,再迅速加入料酒、盐、酱油、鸡汤、味精、胡椒面等佐料,用汤瓢搅匀。同
时把蒸熟的冬菇翻入盘内(要使冬菇盖顶向上),即用水豆粉将汁勾芡,淋入鸡油,再把
汁全部淋于冬菇面上即成。

\features

此菜冬菇细嫩松脆,菌香味特浓,鲜美可口。

\end{recipe}

% vim: filetype=tex noautoindent nojoinspaces
% vim: fileencoding=utf-8 formatoptions+=m
% vim: textwidth=78 tabstop=4 shiftwidth=4 softtabstop=4
