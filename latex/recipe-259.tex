\begin{recipe}{酿冬截}

\ingredients

\ingredient{千冬菇}{三两}
\ingredient{盐}{一钱}
\ingredient{熟火腿}{六钱}
\ingredient{鸡蛋清}{四个}
\ingredient{黄秧白菜心}{一两}
\ingredient{葱白}{二钱}
\ingredient{干豆粉}{一两}
\ingredient{酱油}{四钱五}
\ingredient{姜}{二钱}
\ingredient{鸡脯肉}{二两五}
\ingredient{水豆粉}{六钱}
\ingredient{化猪油}{二两}
\ingredient{胡椒面}{二分五}
\ingredient{鸡油}{六钱}
\ingredient{猪肥礤肉}{一’两五}
\ingredient{味精}{三分}
\ingredient{鸡汤}{四两}
\ingredient{料酒}{一两}
\ingredient{冬笋(净)}{一两}

\cooking

\step 干冬菇选菌盖直径约八分,大小均匀的二十四朵,用 清水一大碗泡半小时,至泡胀而未全胀时泌去水(另作他 用)。换清水用手指在冬菇周围凸凹不平地方掏去泥沙,淘 洗干净,用刀齐盖底切去茎〔作别用);然后放在沸水中煮 约十分钟,端离火口浸泡十分钟捞出;再换清水烧沸同样再 煮一次,泌去煮水不用。此时菌盖直径已发胀至一寸二分左 右,则已干净胀透。

猪油倒入炒锅内,在旺火上烧至五成热,将葱白段、姜 (去皮、拍松)、盐、料酒、鸡汤依次放入,再加入冬菇、味 精、胡椒面同烧五分钟,使冬菇入味,将汤泌去,把冬菇捞 入筲箕中,将水滤干后放盘中摊开,切去茎的一面,向上摆 好待用。

\step 鸡脯肉去掉外面一层白膜,在菜墩上用刀背棰成茸,

边捶边用刀口刮去肉内白筋,捶至极细为止。猪肥膘肉用刀 剁成肉泥(肉泥如同油脂,但不能化油;热天为了避免在剁 时化油,可先将肥膘肉放在冰箱内冰冻后再用,因化油要影 响搅成糁)。大碗内放入清水,把鸡茸投入用竹筷顺着一定 方向搅打五分钟,要搅散搅匀;随后放入鸡蛋清继续用力同 样搅打十分钟;再加入盐、味精、料酒、水豆粉,同时加入 肥肉泥,用力搅打五分钟成稀糊状(用筷蘸一点滴在清水面 上不沉底),即成鸡糁。

鸡蛋清二个与干豆粉调匀成为蛋清豆粉,在盘中摆好的 每个冬菇上的凹处抹上一层,再用调羹舀着鸡糁填满,随用 调羹刮平。然后装入蒸笼内(有糁一面向上)上笼蒸五分 钟,熟后取出待用(临用时才取,以保持热度)。

\step 火腿、冬笋切成长一寸三、宽三分、厚半分的片。黄 秧白菜心切成一寸半大小的块。猪油放入炒锅内,在旺火上 烧至五成热,即放入切好的火腿、冬笋、黄秧白菜,用汤瓢 翻搅两下,再迅速加入料酒、盐、酱油、鸡汤、味精、胡椒 面等佐料,用汤瓢搅匀。同时把蒸熟的冬菇翻入盘内(要使 冬菇盖顶向上〕,即用水豆粉将汁勾芡,淋入鸡油,再把汁 全部淋于冬菇面上即成。

\notes

此菜冬菇细嫩松脆,菌香味特浓,鲜美可口。

\end{recipe}

% vim: filetype=tex noautoindent
% vim: fileencoding=utf-8
% vim: textwidth=78 tabstop=4 shiftwidth=4 softtabstop=4
