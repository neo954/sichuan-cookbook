\begin{recipe}{棋盘鱼肚}

\ingredients

\ingredient{黄鱼肚}{二两}
\ingredient{瘦肉}{三两}
\ingredient{鸡蛋清}{四个}
\ingredient{清汤}{一斤半}
\ingredient{胡椒面}{二分}
\ingredient{肥膘}{二两}
\ingredient{菠菜杆}{十根}
\ingredient{味精}{三分}
\ingredient{鸡脯}{四两}
\ingredient{熟火腿}{一两}
\ingredient{干豆粉}{三钱}
\ingredient{盐}{三分}
\ingredient{料酒}{一两}
\ingredient{水豆粉}{五钱}

\cooking

\step 鱼肚用猪油炸至透胀,先入冷水,再入温热水内漂净浮油,片成一寸宽、一寸五长
、二分厚的长方形片二十四片。火腿切成细丝。菠菜杆每根撕为六根细丝,用刀切成与鱼
肚相等长度。

鸡脯、肥膘分别砸茸,加鸡蛋清、水豆粉、盐少许, 搅匀成“鸡糁”。再将鱼肚用好汤喂
过,捞起挤干水份,再用干 帕子沾干,每个抹上蛋清豆粉(鸡蛋清一个,干豆粉三钱调
匀),再放上鸡糁,然后用菠菜杆细丝及火腿细丝,横竖在 “糁”上摆成棋盘形,放笼上
蒸五分钟,取出拣入碗内。走 菜时灌入扫好的清汤即成。

\notes

色形美观,菜美汤鲜.清淡适口。

\end{recipe}

% vim: filetype=tex noautoindent
% vim: fileencoding=utf-8 formatoptions+=m
% vim: textwidth=78 tabstop=4 shiftwidth=4 softtabstop=4
