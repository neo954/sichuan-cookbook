\begin{recipe}{清蒸青鳝}

\ingredients

\ingredient{青鳝'(十三两)}{一条}
\ingredient{特级清》务}{三斤}

\ingredient{水发竹荪①}{五朵}
\ingredient{生猪网油}{二两五}
\ingredient{熟猪肥肉}{半斤}
\ingredient{盐}{一钱}
\ingredient{胡椒面}{二分}
\ingredient{料酒}{四钱}
\ingredient{鸡蛋}{二个}
\ingredient{花椒}{数颗}

\cooking

\step 杀鳝:青鳝身上涎水很多,不易捉住。因此,在杀鳝时先在地上铺一堆细炭灰,将青鳝由水中捞出放入炭灰中,任它在炭灰中爬来爬去粘满炭灰,就无法逃跑了。这时即用左手按住鳝颈,右手拿一快刀,在鳝鱼眼后端距离约一分处,将鳝头一刀斩下,放尽血液。然后用水洗净鳝身的炭灰,放在瓦盆中。

\step 去皮:将热水(水温高低,可用手试一试。方法是:用手在水中抓三下,觉得烫手,但又烫得不很痛时为合适。饮食业叫它“三把水”。因为水温太高时,抓一二下就会烫手;水温太低则抓五、六下也不致烫手,这是厨师们的实际经验),舀十斤在鳝鱼盆中,鳝身被水一烫,鳝皮变色了。并旦有的地方鳝皮已离鳝肉,再用小刀轻轻地刮去鳝身的青色外皮,刮后现出一种银白色的油皮,皮上呈现岀美丽的花.纹(在去皮过程中切勿将油皮弄破,以免影响外表美观)9

\step 去鳝翅和切段:将鳝平放在案板上,用剪刀将鳝背"翅、鳝尾翅、鳝腮边的拨水翅全部剪去;然后由尾部起切段,每段长六分;切至鳝段中发现内脏时,便要改变刀法,不能再一刀切断,须在鳝身四周割一圈,将骨刺割断,但不要伤着内脏和弄破苦胆,再左手抓着鳝身,右手拿着鳝段向两头一扯,便将内脏抽出鳝段。照此作法,将整个鳝鱼弄成完:整的鳝段。

\step 定碗入笼:用蒸碗一个,先将网油铺于蒸碗中,再将

鳝鱼一段一段地立放在网油上,把料酒、胡椒面、盐和花椒 一起拌匀,淋于鳝段上。再将猪肉切成一寸长、一分半厚的 片,放在鳝段之上。用消毒草纸封固碗口,入蒸笼内,用旺 火蒸三十分钟即成。再用蒸碗一个,放入蛋清二个,和冷清 汤一起顺着一个方向搅匀,搅至起泡时入蒸笼内蒸十分钟,成 为白色芙蓉蛋。

\step 翻碗上席:将蒸好的青鳝出笼,将特级清汤一斤烧开,舀半斤倒入青鳝蒸碗中,稍待片刻,:泌去清汤,再将锅内下余清汤舀入蒸碗中,再泌去清汤。将青鳝翻扣在大圆盘中,揭去网油不用,将白芙蓉蛋划成四块,镶于青鳝四周。再将水发竹荪切成六分长的段,在二汤中汆热,镶于冑鳝四周。再将特级清汤放入锅中,加盐、胡椒面、味精、料酒等烧开、倒入圆盘中上席。

\notes

此系汤菜。青鳝肉肥美,特别鲜雠适口。

0水发竹荪:竹苏用水发二十分钟,洗净泥砂即成。

\end{recipe}

% vim: filetype=tex noautoindent
% vim: fileencoding=utf-8
% vim: textwidth=78 tabstop=4 shiftwidth=4 softtabstop=4
