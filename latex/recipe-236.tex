\begin{recipe}{熏牛肉}

\ingredients

\ingredient{腱子肉}{一斤}
\ingredient{白糖}{三钱}
\ingredient{盐}{二钱五}
\ingredient{五香面}{二分}
\ingredient{酱油}{三钱}
\ingredient{花椒面}{二分}
\ingredient{姜片}{二钱}
\ingredient{菜油}{一斤耗二两}
\ingredient{葱段}{二钱}
\ingredient{料酒}{二钱}
\ingredient{清汤}{三两}

\preparation

\step 选用腱子一斤,将附在肉上的筋剔除干净;横着肉參文切成长一寸、宽五分、厚半
分的薄片;用盐、料酒、葱、姜合牛肉拌匀,浸五分钟时间便入味。

\step 菜油在旺火上烧红,将牛肉倒入锅内炸,炸至牛肉水气一干,油锅内的水泡散尽,
立即把牛肉打起来。如果火大就把锅提开。(火色要看好,炸的时间久了牛肉要发硬,又
是绵的。)

\step 另用干净锅放油一两,在中火上烧热,相继放入牛肉、酱油、清汤、白糖、五香面
在锅内搅匀,待滋汁收干即起锅;起锅时放花椒面,扦在盘内,先将牛肉敞开散热,再用
柏枝熏一下,去掉葱、姜即成。

\features

味甜咸、酥香,适宜佐酒、下粥。用清水收滋汁可保持一星期不变味,适于作旅行菜。

\end{recipe}

% vim: filetype=tex noautoindent nojoinspaces
% vim: fileencoding=utf-8 formatoptions+=m
% vim: textwidth=78 tabstop=4 shiftwidth=4 softtabstop=4
