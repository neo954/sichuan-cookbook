\begin{recipe}{虫草鸭子}

\ingredients

\ingredient{肥鸭}{一只}
\ingredient{虫草}{六钱五}
\ingredient{姜}{七分}
\ingredient{料酒}{三钱}
\ingredient{葱白}{二钱}
\ingredient{盐}{一钱}
\ingredient{清汤}{二斤半}

\cooking

\step 活肥鸭一只(:宰后三斤左右),宰杀后放尽血,退毛, 去掉舌、掌(另作他用再在鸭的背面尾部横割一刀,挖 去内脏,然后用水清洗干净,下入煮锅内紧一下,并用铁抓 子把鸭抓住,在沸水里面连提两三次,使血腥水随着沸水淘 尽;捞起后齐嘴角切去鸭嘴,将翅膀翻向背上盘起。虫草用

温水泡一刻钟,用手轻轻搓洗,去其泥沙、杂质后捞出。

用筷子一根,把一端削成尖头竹签,长约三分、粗约 —分。将鸭腹部向上放好,用竹签斜起从鸭腹肚上穿成一个一 个的孔(深度约三分)。将虫草头部(粗的一端)一个个地插 入鸭腹戳好的孔内,尾部露在外面。插好后将鸭腹部向下装 入蒸碗内,加入料酒、葱、姜、清汤,用皮筋纸盖紧碗口, 上笼用旺火蒸三小时,至骨松、翅裂为度。

\step 蒸好后把鸭子腹部向上摆入另一大碗内,去掉姜、葱 不要,加入少许盐,将原汤注入即成。

\notes

虫草是一种名贵药材,是作为滋补营养品而用在蒸鸭、 炖鸭中的。

\end{recipe}

% vim: filetype=tex noautoindent
% vim: fileencoding=utf-8
% vim: textwidth=78 tabstop=4 shiftwidth=4 softtabstop=4
