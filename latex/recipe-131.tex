\begin{recipe}{豆腐鲫鱼}

\ingredients

\ingredient{活鲫鱼}{二条约七两五}
\ingredient{化猪油}{二两五}
\ingredient{酱油}{六钱}
\ingredient{盐}{六分}
\ingredient{料酒}{四钱}
\ingredient{蒜}{二钱五}
\ingredient{姜}{二钱}
\ingredient{𰪿糟}{六钱}
\ingredient{葱白}{二钱}
\ingredient{味精}{六分}
\ingredient{细豆瓣}{六钱}
\ingredient{甜酱}{二钱}
\ingredient{水豆粉}{三钱}
\ingredient{石膏豆腐}{十块}
\ingredient{清汤}{一斤半}

\preparation

\step 活鲫鱼用刀将头拍一下,使它昏过去,去鳞、开膛、去脏腹(注意不要弄破苦
胆)、挖掉腮,清洗干净后在鱼身两面各斜划三刀,在划口处抹上少许盐;姜、蒜均切成
半分厚、三分见方的薄片;葱白斜切成六分长的段(也可切成末子)。

\step 豆腐切成一寸半长、一寸宽、五分厚的长方形块,用开水在锅内煮五分钟后把水泌
去,再加入清汤一斤、盐六分移在微火上焰着待用。

\step 锅在旺火上放入猪油,烧至八成热时投入鲫鱼,稍煎至两面均呈浅黄色时,把炒
锅放斜,将鱼拨在锅边,用原油将刹,细的豆瓣炸酥,再将炒锅放正,把鱼拨还原,随
即加以料酒,并依次加入酱油、清汤、姜、蒜、葱白。这时就将焅好的豆腐泌去汤汁,
同鱼一起和烧十分钟左右,放入𰪿糟、甜酱、味精用汤瓢轻轻搅转,即用筷子将豆腐拨
开,把鱼拈出放在较大的窝盘内,随即用水豆粉勾芡,将豆腐连汁倒在鱼面上即成。

\features

此菜色泽金红、油亮,味浓厚,豆腐嫩而不烂,入口鲜美,不亚于鱼。下饭佐酒别有风味。

\end{recipe}

% vim: filetype=tex noautoindent nojoinspaces
% vim: fileencoding=utf-8 formatoptions+=m
% vim: textwidth=78 tabstop=4 shiftwidth=4 softtabstop=4
