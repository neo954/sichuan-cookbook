\begin{recipe}{豆腐鲫鱼}

\ingredients

\ingredient{活鯽鱼二条}{约七两五}
\ingredient{化猪油}{二两五}
\ingredient{酱油}{六钱}
\ingredient{盐}{六分}
\ingredient{料酒}{四钱}
\ingredient{蒜}{二钱五}
\ingredient{姜}{二钱}
\ingredient{耢糟}{六钱}
\ingredient{葱白}{二钱}
\ingredient{味精}{六分}
\ingredient{细豆瓣}{六钱}
\ingredient{,紺酱}{二钱}
\ingredient{水豆粉}{三钱}
\ingredient{石膏豆腐}{十块}
\ingredient{清汤}{一斤半}


\cooking

\step 活鲫鱼用刀将头拍一下,使它昏过去,去鳞、开膛、 去脏腹(注意不要弄破苦胆〉、挖掉腮,清洗干净后在鱼身 两面各斜划三刀,在划口处抹上少许盐;姜、蒜均切成半分

厚、三分见方的薄片;葱白斜切成六分长的段(也可切成末 子〉。

\step 豆腐切成一寸半长、一寸宽、五分厚的长方形块,用 开水在锅内煮五分钟后把水泌去,再加入清汤一斤、盐六分 移在微火上焰着待用。

锅在旺火上放入猪油,烧至八成热时投入鲫鱼,稍煎至 两面均呈浅黄色时,把炒锅放斜,将鱼拨在锅边,用原油将刹, 细的豆瓣炸酥,再将炒锅放正,把鱼拨还原,随即加以料酒, 并依次加入酱油、清汤、姜、蒜、葱白。这时就将锆好的豆腐 泌去汤汁,同鱼一起和烧十分钟左右,放入耢糟、甜酱、味精 用汤瓢轻轻搅转,即用筷子将豆腐拨开,把鱼拈出放在较大 的窝盘内,随即用水豆粉勾芡,将豆腐连汁倒在鱼面上即成。

\notes

此菜色泽金红、油亮,味浓厚,豆腐嫩而不烂,入口鲜 美,不亚于鱼。下饭佐酒别有风味。

\end{recipe}

% vim: filetype=tex noautoindent
% vim: fileencoding=utf-8
% vim: textwidth=78 tabstop=4 shiftwidth=4 softtabstop=4
