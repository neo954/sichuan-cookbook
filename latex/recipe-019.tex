\begin{recipe}{奶汤大杂烩}

\ingredients

\ingredient{千辑肉皮}{一两五}
\ingredient{胡椒}{三分}
\ingredient{熟猪肚}{一'两五}
\ingredient{火腿}{一两}
\ingredient{水豆粉}{六分}
\ingredient{猪肉}{六两}

\ingredient{鲜青菜}{一束}
\ingredient{熟猪舌}{一两五}
\ingredient{干豆粉}{七钱}
\ingredient{料酒}{七钱}
\ingredient{菜油}{一1斤半耗四两}
\ingredient{熟猪心}{一"两五}
\ingredient{干笋}{一两玉}
\ingredient{味精}{三分}
\ingredient{盐}{八分}
\ingredient{鸡蛋}{四个}
\ingredient{鸣松菌}{一-两}
\ingredient{特级奶汤}{一斤四两}

\cooking

\step 锅内倒入菜油一斤半烧热,将猪肉皮放入,约五分钟 炸软,连锅提起,移放微火上再浸五分钟,至色为深牙黄色 肘捞起晾冷,切成一寸长的条块,漂于清水中待用〈俗称响

\step 

\step 干笋洗净,用水浸泡,胀透为度;再每条用手撕成四 条,切成二寸的段。然后入沸水中煮透,去净硫磺质味,再 用清汤永一次,榜出待用。

\step 选用肥膘猪肉二两,洗净去皮,切成二寸长、四分宽 的片。鸡蛋(二个)、干豆粉(五钱)混合调为蛋清豆粉, 加盐(二分)与肉片拌勻。锅内菜油烧]:,把拌好的肉片放入 炸五分钟,至炸成深牙黄色时 措出,日足冷待用(俗称穌肉

图1 尖刀元子作法

选用肥瘦相连的猪肉二 两,洗净去皮,剁为肉茸,加 干豆粉(二钱)、鸡蛋(一个 去壳)、盐(三分)混合调匀,

拨一部分摊于左手上面(平摊 约四分厚、一寸半宽),右手 持刀斜刮左手的肉茸为上大下 小的三角条形(俗称鲫鱼背)共十六条,分开摆于盘內,入笼

蒸十分钟后取出晾冷待用(图1俗称尖刀元子)。 5,熟猪心、猪舌均切成一寸半长的薄片。

图2 如意蛋卷制作过程

甲、肉丝放入蛋皮两端 1.蛋皮;2,肉丝

乙、卷成两卷 丙、如意蛋卷

飞.鸡蛋一个,去壳搅散。炒锅烧热,锅中擦净,稍用菜油 涂匀,将蛋倾入,以手提耳锅转动,即摊成蛋皮,稍冷,蛋 皮即自行脱落。把肥瘦相连的猪肉二两洗净,剁成细茸,加 入水豆粉、盐、胡椒,调匀后铺于贴锅一面的蛋皮上(如图 2甲),两端向中央搓裹,成为两卷,中间相连(如图2乙), 接口处涂少许豆粉粘好。然后平放盘内入笼蒸十五分钟,取 出晾冷,用刀稍斜横切成一寸半宽的小块(俗称如意蛋卷)

(如图2丙:)。

\step 用平底大碗一个,先将汆好的笋条、响皮摆在碗底; 次将酥肉逐片摆成圆形,为第二层;又将尖刀元子十六条分 为每方四条摆成田字形,为第三层;再将猪心、猪舌象铺瓦 一样分别各摆半个碗,成圆形,为第四层;最后将蛋卷用刀 平铲放于上面。再用火腿,把特级奶汤、味精、料酒灌入摆 好的杂烩碗内,入笼蒸二小时取出。再用特级奶汤十两,加味 精、盐,舀入杂烩内。另选绿色鲜菜一束,淘洗烫熟,放入 碗内以衬其色即成。

\notes

此菜系汤菜合一,味鲜可口,佐酒下饭均宜。

\end{recipe}

% vim: filetype=tex noautoindent
% vim: fileencoding=utf-8
% vim: textwidth=78 tabstop=4 shiftwidth=4 softtabstop=4
