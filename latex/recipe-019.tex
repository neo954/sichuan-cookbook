\begin{recipe}{奶汤大杂烩}

\ingredients

\ingredient{千辑肉皮}{一两五}
\ingredient{胡椒}{三分}
\ingredient{熟猪肚}{一两五}
\ingredient{火腿}{一两}
\ingredient{水豆粉}{六分}
\ingredient{猪肉}{六两}
\ingredient{鲜青菜}{一束}
\ingredient{熟猪舌}{一两五}
\ingredient{干豆粉}{七钱}
\ingredient{料酒}{七钱}
\ingredient{菜油}{一斤半耗四两}
\ingredient{熟猪心}{一两五}
\ingredient{干笋}{一两玉}
\ingredient{味精}{三分}
\ingredient{盐}{八分}
\ingredient{鸡蛋}{四个}
\ingredient{鸡松菌}{一两}
\ingredient{特级奶汤}{一斤四两}

\preparation

\step
锅内倒入菜油一斤半烧热,将猪肉皮放入,约五分钟炸软,连锅提起,移放微火上再浸五
分钟,至色为深牙黄色肘捞起晾冷,切成一寸长的条块,漂于清水中待用(俗称响皮)。

\step 干笋洗净,用水浸泡,胀透为度;再每条用手撕成四条,切成二寸的段。然后入沸
水中煮透,去净硫磺质味,再用清汤永一次,榜出待用。

\step 选用肥膘猪肉二两,洗净去皮,切成二寸长、四分宽的片。鸡蛋(二个)、干豆粉
(五钱)混合调为蛋清豆粉,加盐(二分)与肉片拌匀。锅内菜油烧红,把拌好的肉片放
入炸五分钟,至炸成深牙黄色时措出,日足冷待用(俗称穌肉)。

图1 尖刀元子作法

\step 选用肥瘦相连的猪肉二两,洗净去皮,剁为肉茸,加干豆粉(二钱)、鸡蛋(一个
去壳)、盐(三分)混合调匀,拨一部分摊于左手上面(平摊约四分厚、一寸半宽),右
手持刀斜刮左手的肉茸为上大下小的三角条形(俗称鲫鱼背)共十六条,分开摆于盘內,
入笼蒸十分钟后取出晾冷待用(图1俗称尖刀元子)。

\step 熟猪心、猪舌均切成一寸半长的薄片。

图2 如意蛋卷制作过程

甲、肉丝放入蛋皮两端1. 蛋皮;2. 肉丝

乙、卷成两卷丙、如意蛋卷

\step 鸡蛋一个,去壳搅散。炒锅烧热,锅中擦净,稍用菜油涂匀,将蛋倾入,以手提耳
锅转动,即摊成蛋皮,稍冷,蛋皮即自行脱落。把肥瘦相连的猪肉二两洗净,剁成细茸,
加入水豆粉、盐、胡椒,调匀后铺于贴锅一面的蛋皮上(如图2甲),两端向中央搓裹,成
为两卷,中间相连(如图2乙),接口处涂少许豆粉粘好。然后平放盘内入笼蒸十五分钟,
取出晾冷,用刀稍斜横切成一寸半宽的小块(俗称如意蛋卷)(如图2丙)。

\step 用平底大碗一个,先将汆好的笋条、响皮摆在碗底;次将酥肉逐片摆成圆形,为第
二层;又将尖刀元子十六条分为每方四条摆成田字形,为第三层;再将猪心、猪舌象铺瓦
一样分别各摆半个碗,成圆形,为第四层;最后将蛋卷用刀平铲放于上面。再用火腿,把
特级奶汤、味精、料酒灌入摆好的杂烩碗内,入笼蒸二小时取出。再用特级奶汤十两,加
味精、盐,舀入杂烩内。另选绿色鲜菜一束,淘洗烫熟,放入碗内以衬其色即成。

\features

此菜系汤菜合一,味鲜可口,佐酒下饭均宜。

\end{recipe}

% vim: filetype=tex noautoindent nojoinspaces
% vim: fileencoding=utf-8 formatoptions+=m
% vim: textwidth=78 tabstop=4 shiftwidth=4 softtabstop=4
