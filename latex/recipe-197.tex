\begin{recipe}{糖粘羊尾}

\ingredients

\ingredient{净猪肥膘}{七两五}
\ingredient{千豆粉}{一两三}
\ingredient{鸡蛋}{二个}
\ingredient{白糖}{四两}
\ingredient{熟芝麻(研末)}{一钱五}
\ingredient{菜油}{一斤半耗一两五}

\cooking

\step 选猪肋条上的净肥膘肉一方,在开水锅内煮约一小时左右(肉熟透微把),措起臉
冷,切成长一寸二、宽厚各二分的肉条;把鸡蛋(连蛋黄)与干豆粉调成蛋糊;把切好后
的肉条在沸水内煮热,用拧干的热布帕将肉条水份沾干,在碗内与蛋糊拌合调匀,使肉条
都裹上一层蛋糊。

\step 菜油在旺火锅内烧至六成热时,即将裹好蛋糊的肉条逐个放入油内炸,并用汤瓢搅
动,炸成金黄色时捞出。

\step 另用无油的干净炒锅放入清水在旺火上烧开,把白糖放入,用小浐不停的搅造,待
水份逐渐蒸发,糖汁冒起大气泡时,即将炸好的肉条倒入,用手铲不断翻炒,同时加入熟
芝麻细末,再翻炒颠簸两下(若火过旺可将炒锅移离火口,以免糖糊),此时糖汁、芝麻
已沾满肉条,即起锅。

\notes

色呈金黄,外酥内嫩,油而不腻,香甜可口。

\end{recipe}

% vim: filetype=tex noautoindent
% vim: fileencoding=utf-8 formatoptions+=m
% vim: textwidth=78 tabstop=4 shiftwidth=4 softtabstop=4
