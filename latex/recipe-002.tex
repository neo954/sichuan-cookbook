\begin{recipe}{鱼香肉片}

\ingredients

\ingredient{净瘦肉}{五两}
\ingredient{细葱}{五分}
\ingredient{料酒}{二钱}
\ingredient{水发木耳}{一两}
\ingredient{混合油}{二两}
\ingredient{盐}{一分}
\ingredient{泡鱼辣椒}{一两五}
\ingredient{白糖}{二钱}
\ingredient{水豆粉}{五分}
\ingredient{姜}{二分}
\ingredient{酱油}{四钱}
\ingredient{大蒜}{二分}
\ingredient{醋}{一钱}

\cooking

\step 姜、蒜去皮,连同鱼辣椒分别剁成碎末,葱子切成细花。

\step 瘦肉切成长一寸二、宽八分的薄片,用料酒、水豆粉、盐与肉片拌和均匀。

\step 葱花、姜、蒜、鱼辣椒末、木耳、白糖、酱油、醋、水豆粉,加好汤少许,在碗内
兑成鱼香滋汁。

\step 炒锅在旺火上烧红后,将油舀入锅内泌去,再换温油入锅,随将肉片倒下,用瓢子
解散,即将鱼辣椒末倒下,把肉片煵起红色,即将兑好的滋汁倒入,急炒几下起锅。

\notes

鱼香味是四川独特口味之一,味兼甜、酸、咸、辣各味,菜内无鱼而富有浓烈的鱼味鲜
香,很多菜均宜,《鱼香烘蛋》、《鱼香油菜》等。

\end{recipe}

% vim: filetype=tex noautoindent
% vim: fileencoding=utf-8 formatoptions+=m
% vim: textwidth=78 tabstop=4 shiftwidth=4 softtabstop=4
