% BSD 3-Clause License
%
% Copyright (c) 2023 Quux System and Technology. All rights reserved.
%
% Redistribution and use in source and binary forms, with or without
% modification, are permitted provided that the following conditions are met:
%
% 1. Redistributions of source code must retain the above copyright notice, this
%    list of conditions and the following disclaimer.
%
% 2. Redistributions in binary form must reproduce the above copyright notice,
%    this list of conditions and the following disclaimer in the documentation
%    and/or other materials provided with the distribution.
%
% 3. Neither the name of the copyright holder nor the names of its
%    contributors may be used to endorse or promote products derived from
%    this software without specific prior written permission.
%
% THIS SOFTWARE IS PROVIDED BY THE COPYRIGHT HOLDERS AND CONTRIBUTORS "AS IS"
% AND ANY EXPRESS OR IMPLIED WARRANTIES, INCLUDING, BUT NOT LIMITED TO, THE
% IMPLIED WARRANTIES OF MERCHANTABILITY AND FITNESS FOR A PARTICULAR PURPOSE ARE
% DISCLAIMED. IN NO EVENT SHALL THE COPYRIGHT HOLDER OR CONTRIBUTORS BE LIABLE
% FOR ANY DIRECT, INDIRECT, INCIDENTAL, SPECIAL, EXEMPLARY, OR CONSEQUENTIAL
% DAMAGES (INCLUDING, BUT NOT LIMITED TO, PROCUREMENT OF SUBSTITUTE GOODS OR
% SERVICES; LOSS OF USE, DATA, OR PROFITS; OR BUSINESS INTERRUPTION) HOWEVER
% CAUSED AND ON ANY THEORY OF LIABILITY, WHETHER IN CONTRACT, STRICT LIABILITY,
% OR TORT (INCLUDING NEGLIGENCE OR OTHERWISE) ARISING IN ANY WAY OUT OF THE USE
% OF THIS SOFTWARE, EVEN IF ADVISED OF THE POSSIBILITY OF SUCH DAMAGE.
%
\begin{recipe}{绣球𧎼蛀}

\ingredients

\ingredient{𧎼蛀}{五两}
\ingredient{鸡脯肉}{二两}
\ingredient{肥膘}{一两五}
\ingredient{鸡蛋}{三个}
\ingredient{丝瓜皮}{一根}
\ingredient{熟火腿}{一两}
\ingredient{水豆粉}{五钱}
\ingredient{特级清汤}{二斤}
\ingredient{盐}{四分}
\ingredient{胡椒面}{二分}
\ingredient{味精}{二分}
\ingredient{料酒}{三钱}

\preparation

\step 𧎼蛀抠去玉带,淘洗干净,掺清水上笼蒸𤆵,取出挤干水晾冷,搓散成丝;鸡脯、
肥膘分别砸茸;鸡蛋清调匀,加清水、豆粉、味精、盐搅成“鸡糁”;用鸡蛋一个摊成蛋
皮。

\step 蛋皮、丝瓜皮、火腿分别切成五至六分长细丝,连同𧎼蛀丝在一个大盘内和匀抖
散,留三分之一在盘内均匀地铺开;即将“鸡糁”舀成二十四至二十八个圆子在上面;再将
其余三分之二的丝均匀地撒盖在圆子上面;然后再一个一个地拈入手中,将各丝连同圆子
团成绣球形;放入笼内蒸五分钟至熟取出。

\step 走菜时先将“绣球”拣入蒸碗内掺清汤少许,加胡椒、料酒上笼馏热;锅内掺清汤烧
开吃味,再将汤清扫两次;然后将馏热的“绣球”滗去汤汁倒入锅内,汤开时盛碗上席。

\features

色调美观,圆如彩球,味美肉鲜,清淡营养。

\end{recipe}

% vim: filetype=tex noautoindent nojoinspaces
% vim: fileencoding=utf-8 formatoptions+=m
% vim: textwidth=78 tabstop=4 shiftwidth=4 softtabstop=4
