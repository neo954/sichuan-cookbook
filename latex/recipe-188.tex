\begin{recipe}{绣球蟠蛀}

\ingredients

\ingredient{瑤蛀}{五两}
\ingredient{鸡脯肉}{二两}
\ingredient{肥膘}{一两五}
\ingredient{鸡蛋}{三个}
\ingredient{丝瓜皮}{一根}
\ingredient{熟火腿}{一两}
\ingredient{水豆粉}{五钱}
\ingredient{特级清汤}{二斤}
\ingredient{盐}{四分}
\ingredient{胡椒面}{二分}
\ingredient{味精}{二分}
\ingredient{料酒}{三钱}

\preparation

\step 鳐蛀抠去玉带,淘洗干净,掺清水上笼蒸粑,取出挤干水晾冷,搓散成丝;鸡脯、
肥膘分别砸茸;鸡蛋清调匀,加清水、豆粉、味精、盐搅成“鸡糁”;用鸡蛋一个摊成蛋皮
。

蛋皮、丝瓜皮、火腿分别切成五至六分长细丝,连同 蟠蛀丝在一个大盘内和匀抖散,留
三分之一在盘内均匀地铺 开厂即将“鸡糁”舀成二十四至二十八个元子在上面;再将 其余
三分之二的丝均匀地撒盖在元子上面;然后再一个一个 地拈入手中,将各丝连同元子团
成绣球形;放入笼内蒸五分 钟至熟取出。

\step 走菜时先将“绣球”拣入蒸碗内掺清汤少许,加胡椒、料濟上笼馏热;锅内掺清汤烧
开吃味,再将汤清扫两次;然后将馏热的“绣球”泌去汤汁倒入锅内,汤开时盛碗上席。

\features

色调美观,圆如彩球,味美肉鲜,清淡营养。

\end{recipe}

% vim: filetype=tex noautoindent
% vim: fileencoding=utf-8 formatoptions+=m
% vim: textwidth=78 tabstop=4 shiftwidth=4 softtabstop=4
