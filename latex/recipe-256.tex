\begin{recipe}{五彩土司}

\ingredients

\ingredient{土司}{一碎}
\ingredient{鲜鱼肉}{八两}
\ingredient{猪肥皤}{二两}
\ingredient{鸡蛋}{五个}
\ingredient{熟瘦火腿}{五钱}
\ingredient{水发木耳}{五钱}
\ingredient{花椒面}{二分}
\ingredient{豆粉}{三钱}
\ingredient{芝麻}{五钱}
\ingredient{绿色小菜}{一两}
\ingredient{香油}{五钱}
\ingredient{蕃茄酱}{五钱}
\ingredient{生菜}{三两}
\ingredient{白糖}{三钱}
\ingredient{酷}{三钱}
\ingredient{盐}{五分}
\ingredient{菜油}{二斤耗一两五}

\preparation

\step 鲜鱼剔骨,去皮,,选净细刺后,与肥膘分别用刀背砸茸搅成“鱼糁”。摊蛋皮一张
,连同火腿、木耳、绿色小菜分别切成各种细末,连芝麻共成五种细末各分装一边。

\step 土司修去周围的皮后,切成二分厚的片,先在一面抹上一层“蛋清豆粉”再放上“鱼
糁”约三分厚抹平,再将五种细末一样样摊开在墩子上,分颜色用刀口刮成一字条形,顺
着贴在“糁”上约一分厚,用手按平不脱,自成五色,每片同样贴好,上笼蒸五分钟取出。

\step 锅内掺油在旺火上烧至七成火,将土司放入,贴细末一面向上,炸成金黄色,将油
泌尽,淋上香油扞起;再用刀改成五分宽、一寸二长的条子(切时按五色切开使每片都现
彩色),装在条盘的一端,另一端放生菜。走菜时随椒盐一碟同上。

\features

色分五彩,美观;味兼香、酥、脆、嫩,为席桌行菜之一。

\end{recipe}

% vim: filetype=tex noautoindent nojoinspaces
% vim: fileencoding=utf-8 formatoptions+=m
% vim: textwidth=78 tabstop=4 shiftwidth=4 softtabstop=4
