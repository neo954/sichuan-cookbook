\begin{recipe}{苕菜狮子头}

\ingredients

\ingredient{冬干苕菜}{二两}
\ingredient{肥瘦肉}{各半斤}
\ingredient{慈菇}{三两}
\ingredient{金钩}{五钱}
\ingredient{肥火腿}{一两}
\ingredient{鲜青豆}{二两}
\ingredient{7鸟蛋}{二个}
\ingredient{干豆粉}{一两}
\ingredient{鸡油}{一两}
\ingredient{化猪油}{一斤耗三两}
\ingredient{料酒}{五钱}
\ingredient{盐}{六分}
\ingredient{胡椒}{一分}
\ingredient{味精}{三分}
\ingredient{姜、葱}{各三钱}
\ingredient{清汤}{二斤}

\preparation

\step 金钩用清水发胀,慈菇去皮,肥痩肉、火腿、鲜青豆等分别切成碎颗,鸡蛋清与豆
粉调成蛋清豆粉,同时盛入碗内,再放入盐、料酒、胡椒、味精等,用手拌匀,按四分之
一捏成四个扁形的元子,放入锅内(锅内有猪油),炸进皮肘,不见黄色即捞起。
\step 用小包罐一个洗净,将捞起的元子放下,加入清汤、姜、葱及剩余全部作料,放在
文火上烧二时。烧至半熟时将淘洗干净的苕菜放入,继续烧至熟透为止。走菜时用深圆盘
先将四个元子摆成四方形,将苕菜镶于四周围,将罐内的原汁灌入呈半汤形,淋入鸡油即
成。

\features

味道鲜美,清香。

\end{recipe}

% vim: filetype=tex noautoindent nojoinspaces
% vim: fileencoding=utf-8 formatoptions+=m
% vim: textwidth=78 tabstop=4 shiftwidth=4 softtabstop=4
