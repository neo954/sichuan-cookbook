% BSD 3-Clause License
%
% Copyright (c) 2023 Quux System and Technology. All rights reserved.
%
% Redistribution and use in source and binary forms, with or without
% modification, are permitted provided that the following conditions are met:
%
% 1. Redistributions of source code must retain the above copyright notice, this
%    list of conditions and the following disclaimer.
%
% 2. Redistributions in binary form must reproduce the above copyright notice,
%    this list of conditions and the following disclaimer in the documentation
%    and/or other materials provided with the distribution.
%
% 3. Neither the name of the copyright holder nor the names of its
%    contributors may be used to endorse or promote products derived from
%    this software without specific prior written permission.
%
% THIS SOFTWARE IS PROVIDED BY THE COPYRIGHT HOLDERS AND CONTRIBUTORS "AS IS"
% AND ANY EXPRESS OR IMPLIED WARRANTIES, INCLUDING, BUT NOT LIMITED TO, THE
% IMPLIED WARRANTIES OF MERCHANTABILITY AND FITNESS FOR A PARTICULAR PURPOSE ARE
% DISCLAIMED. IN NO EVENT SHALL THE COPYRIGHT HOLDER OR CONTRIBUTORS BE LIABLE
% FOR ANY DIRECT, INDIRECT, INCIDENTAL, SPECIAL, EXEMPLARY, OR CONSEQUENTIAL
% DAMAGES (INCLUDING, BUT NOT LIMITED TO, PROCUREMENT OF SUBSTITUTE GOODS OR
% SERVICES; LOSS OF USE, DATA, OR PROFITS; OR BUSINESS INTERRUPTION) HOWEVER
% CAUSED AND ON ANY THEORY OF LIABILITY, WHETHER IN CONTRACT, STRICT LIABILITY,
% OR TORT (INCLUDING NEGLIGENCE OR OTHERWISE) ARISING IN ANY WAY OUT OF THE USE
% OF THIS SOFTWARE, EVEN IF ADVISED OF THE POSSIBILITY OF SUCH DAMAGE.
%
\begin{recipe}{苕菜狮子头}

\ingredients

\ingredient{冬干苕菜}{二两}
\ingredient{肥瘦肉}{各半斤}
\ingredient{茨菰}{三两}
\ingredient{金钩}{五钱}
\ingredient{肥火腿}{一两}
\ingredient{鲜青豆}{二两}
\ingredient{鸡蛋}{二个}
\ingredient{干豆粉}{一两}
\ingredient{鸡油}{一两}
\ingredient{化猪油}{一斤耗三两}
\ingredient{料酒}{五钱}
\ingredient{盐}{六分}
\ingredient{胡椒}{一分}
\ingredient{味精}{三分}
\ingredient{姜、葱}{各三钱}
\ingredient{清汤}{二斤}

\preparation

\step 金钩用清水发胀,茨菰去皮,肥痩肉、火腿、鲜青豆等分别切成碎颗,鸡蛋清与豆
粉调成蛋清豆粉,同时盛入碗内,再放入盐、料酒、胡椒、味精等,用手拌匀,按四分之
一捏成四个扁形的圆子,放入锅内(锅内有猪油),炸进皮时,不见黄色即捞起。

\step 用小包罐一个洗净,将捞起的圆子放下,加入清汤、姜、葱及剩余全部作料,放在
文火上烧二时。烧至半熟时将淘洗干净的苕菜放入,继续烧至熟透为止。走菜时用深圆盘
先将四个圆子摆成四方形,将苕菜镶于四周围,将罐内的原汁灌入呈半汤形,淋入鸡油即
成。

\features

味道鲜美,清香。

\end{recipe}

% vim: filetype=tex noautoindent nojoinspaces
% vim: fileencoding=utf-8 formatoptions+=m
% vim: textwidth=78 tabstop=4 shiftwidth=4 softtabstop=4
