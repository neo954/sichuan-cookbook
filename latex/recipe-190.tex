\begin{recipe}{白酿海参}

(原金钱海参〉

\ingredients

\ingredient{水发海参}{七两}
\ingredient{鸡脯肉}{三两}
\ingredient{肥猪肉}{二两}
\ingredient{火腿}{一两}

\ingredient{鸡蛋}{二个}
\ingredient{千豆粉}{四钱}
\ingredient{盐}{五分}
\ingredient{料酒}{一钱}
\ingredient{味精}{二分}
\ingredient{清汤}{一斤}
\ingredient{黄秧白心}{半斤}
\ingredient{化猪油}{一两}
\ingredient{鸡油}{五钱}

\cooking

\step 	将水发海参洗净,用清汤汆一次捞起,将水揩干,放于 盘内;再将鸡脯肉及猪肥膘捶茸,搅成“鸡糁”,放入碗内待用。

\step 	将火腿切成二粗丝;蛋清和豆粉拌成蛋清豆粉,抹入 海参腹内,放在一个抹上一层油的平盘内;将鸡糁放入海参 腹内,火腿丝放入鸡糁当中;再将海参捏成圆形,仍放入抹 油的盘内,上笼大火蒸十分钟,取出晾冷;在墩子上切成三分 厚的圆片,用二碗一个摆入碗内,切余的两头摆于面上;将黄 秧白淘洗干净,去筋,沮粑,放于海参面上,再放入笼内馏起。

\step 	将锅放于旺火上,放入化猪油,烧至七成热时加入清汤 半斤、味精、料酒、胡椒、盐,烧开,放下豆粉,勾成二流芡; 再取出笼内的海参,翻入大盘内,将鸡油淋于锅内,起锅将 滋汁淋于海参面上即成。

\notes

粑软味浓,鲜美适口,宜于春冬。

\end{recipe}

% vim: filetype=tex noautoindent
% vim: fileencoding=utf-8
% vim: textwidth=78 tabstop=4 shiftwidth=4 softtabstop=4
