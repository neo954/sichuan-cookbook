\begin{recipe}[神仙鸭子]{南边鸭子}

\ingredients

\ingredient{鸭子一只}{净二斤}
\ingredient{水发口笨}{~一两}

\ingredient{熟火腿}{二两}
\ingredient{水发兰片}{二两}
\ingredient{料酒}{一两}
\ingredient{盐}{三分}
\ingredient{白糖(炒成汁)}{三钱}
\ingredient{酱油}{五钱}
\ingredient{姜}{三钱}
\ingredient{、}{葱三钱}
\ingredient{味精}{三分}
\ingredient{香油}{五钱}
\ingredient{清汤}{四斤}
\ingredient{菜油}{一斤耗一'两}

\ingredient{化猪油}{三两}
\ingredient{新炒布(见方一尺五)}{一幅}

\cooking

\step 将鸭子宰杀,去毛,挖腹,下锅煮去血水,宰去足爪、嘴壳,晾一下揩干水份,全身抹上料酒,下入八成热的油锅炸成浅黄色,捞起用开水透去油脂;火腿切成长一寸五、宽三分、厚一分五的片;兰片切成长一寸二、宽三分、厚一分五的片;口茉对破,姜拍松,葱挽成结待用。

\step 用大鱼碗一个,铺上清洗干净的纱布,将火腿、兰片、口茉摆成“三叠水”,把炸好的鸭子以鸭脯向下挨着摆好的火腿等放好,即将纱布对角抄拢包起打成结,提入罐子内

(:罐内先垫好鸡、鸭骨)。

\step 将白糖汁、盐、酱油、姜、葱、料酒(抹后剩下的)和清汤分别加入鏆内盖上盖子,在武火上烧二十分钟,改用文火,继续烧至十分火候,骨松肉粑时,将鸭子提起解开纱布结,翻入圆盘,揭去纱布,将鏆内滋汁倒入炒锅内收浓,加味精,提锅离火口,再加香油淋于鸭子上即成。

\notes

色鲜、味浓、粑香,特别适宜老年,原“神仙鸭子”即 尊称老年人为“老神仙”的意思。

\end{recipe}

% vim: filetype=tex noautoindent
% vim: fileencoding=utf-8
% vim: textwidth=78 tabstop=4 shiftwidth=4 softtabstop=4
