\begin{recipe}{菠饺玻璃肚}

\ingredients

\ingredient{猪肚头三个}{一斤}
\ingredient{清汤}{一斤}
\ingredient{香油}{二分}
\ingredient{猪瘦肉}{五两}
\ingredient{盐}{二钱}
\ingredient{味精}{五分}
\ingredient{干面粉}{一两}
\ingredient{酱油}{一钱}
\ingredient{胡椒面}{二分}
\ingredient{菠菜}{四两}
\ingredient{料酒}{二钱}
\ingredient{草碱(漂用)}{六钱}

\preparation

\step 选新鲜猪肚,只取肚头部分,清洗干净;用刀将两面的筋缠及边沿修去,成长方
形,随着形式用刀将肚头片成极薄的片(越薄越好)。片的要求:要薄、要匀、要不片
烂。

\step 用清水少许将草碱溶化于肚片上造匀,在盆内浸渍半小时后,用沸水冲入盖严;烫
焖十分钟揭盖,即将碱水泌去,另换清水;每隔五分钟换一次,如是换四次,直到将碱味
去掉为止。此时肚片则变为细嫩、柔软、半透明,如玻璃体状。

\step 痩肉用刀背砸茸,先用一两剔尽筋缠,加入酱油、料酒、香油及味精、胡椒面、盐
各少许拌成馅。菠菜淘洗干净,用手揉滥,挤水,拌干面粉,调匀,揉成绿色子面,擀成
圆形饺皮二十四张,将馅包入成半圆形的“成都水饺”式样,用沸水煮熟打起,与肚片在
大碗内对镶好。

\step 清汤在锅内烧沸,用肉茸分两次在锅内扫成白清汤,加味精、胡椒面、盐注入碗内
即成。

\features

汤清如水,绿白相间,脆嫩清淡,夏天尤宜。

\end{recipe}

% vim: filetype=tex noautoindent nojoinspaces
% vim: fileencoding=utf-8 formatoptions+=m
% vim: textwidth=78 tabstop=4 shiftwidth=4 softtabstop=4
