\begin{recipe}{萝卜连锅}

\ingredients

\ingredient{猪肉}{一斤}
\ingredient{白萝卜}{一斤半}
\ingredient{葱白}{三钱}
\ingredient{姜}{二钱}
\ingredient{菜油}{一两五}
\ingredient{千辣概}{五钱}
\ingredient{花椒}{约二十粒}
\ingredient{豆瓣}{一两五}
\ingredient{酱油}{一两五}
\ingredient{味糈}{一分}

\cooking

\step 选猪腿肥瘦二刀肉,去毛,刮洗干净。锅内的清水以把肉淹过为准(一次加足,不
要再添),在旺火上烧沸,再放入猪肉,至再沸时用汤瓢打去浮沫(在制作过程中要随时
打去浮沫)。随即加入花椒十粒,并把姜拍松放入,用木盖盖好。煮十分钟左右,用削尖
的竹筷从肉皮上往下插,略微用力,一穿即透便够火候。然后捞出晾一下,等热度稍降低
不太烫手时,按肉纹横切成两块。把肉块放在菜墩上,片成三寸长的薄片,愈薄愈好。
\step 萝卜去皮,洗净,切成约一寸五长、一寸宽、三分厚的片,连同葱白放入煮肉汤中
煮六、七分钟。至萝卜片刚粑,即将肉片倒入锅中搅匀,与萝卜片同煮二、三分钟。待萝
卜粑透时加入味精,即成“连锅”。(烹制中随时注意把锅盖好。)
\step 花椒十粒,与干辣椒一同放入锅中,在微火上用手铲不停翻炒,至辣椒刚炒酥时滴
入菜油少许炒匀,铲起砸成辣椒面,用大汤杯盛起。将菜油烧热,等温度降低一半时倒入
辣椒面中,随即加入豆瓣、酱油、味精等,调匀为料碟(料碟可备一、二碟,或每人一
碟)与连锅一同上席。

\notes

此菜是汤菜,色乳白,宜热吃。肉料取材要保证煮熟,切片时肥、瘦、皮连而不脱,入口
肥而不腻,瘦而不绵。吃时蘸调料,味麻辣鲜香,汤系生烧,肉味鲜甜。此菜四季咸宜,
在冬季佐餐常盛入火锅上席,以保持菜热汤鲜,故名“连锅”。

\end{recipe}

% vim: filetype=tex noautoindent
% vim: fileencoding=utf-8
% vim: textwidth=78 tabstop=4 shiftwidth=4 softtabstop=4
