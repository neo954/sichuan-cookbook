\begin{recipe}{松子肉}

\ingredients

\ingredient{豆油皮}{一张}
\ingredient{萝卜}{四两五}
\ingredient{猪肉}{二两}
\ingredient{鸡蛋}{二个}
\ingredient{胡概}{三分}
\ingredient{菜油}{一斤耗一'两五}
\ingredient{千豆粉}{三钱}
\ingredient{奶汤}{四两}
\ingredient{二汤}{二两}
\ingredient{盐}{五分}
\ingredient{味精}{二分}
\ingredient{松子}{二钱}
\ingredient{葱}{二钱}
\ingredient{姜}{六分}

\cooking

\step 选用肥瘦相连的猪肉,洗净,切成二寸长、一分半厚 的细丝。松子去皮捶茸放于肉丝中。萝卜去皮,切成二分 厚、二寸长的祖丝,挤干水。姜、葱均切成细丝。

\step 鸡蛋去壳与干豆粉调匀成为蛋清豆粉,加入味精、

图3 松子肉包叠法图

\step 豆油皮2,肉丝3,叠好后

盐、胡椒,于深碗 中混合好,再与渚 肉丝、萝卜丝、姜、 葱一并拌匀。

\step 豆油皮一张 修成六寸宽、八寸 长,平铺案板上, 而后把拌匀的猪肉丝、萝卜丝等铺在豆油皮上,铺成七分 厚,铺上豆油皮的一半,再把另一半叠过来,接头处涂蛋淸 豆粉封口,如图3。

\step 炒锅倒入菜油烧红,将叠好的豆油皮包放入炸透,炸 成金黄色,泌去炸油,放盘中晾冷。再用刀切成一寸半长、

三分宽、七分厚的长条,共二十四 条。以六条为一组,于碗中镶成田 字形,如图4。然后加二汤淹没内 条,上笼蒸三十分钟取出,扣于碗 内,再加奶汤四两、盐七分即成。

爵4 松子肉摆法

\notes

此菜香软可口,营养丰富,特别适于冬季食用。

\end{recipe}

% vim: filetype=tex noautoindent
% vim: fileencoding=utf-8
% vim: textwidth=78 tabstop=4 shiftwidth=4 softtabstop=4
