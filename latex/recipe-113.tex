\begin{recipe}{豆渣鸭脯}

\ingredients

\ingredient{鸭青一只}{约三斤}
\ingredient{生细豆法}{一斤}

\ingredient{化猪油}{五两}
\ingredient{盐}{五分}
\ingredient{料酒}{一两}
\ingredient{姜、葱}{各二钱}
\ingredient{胡椒}{二分}
\ingredient{味精}{二分}

\cooking

\step 鸭子宰杀后去毛,大开去脏腹、去足,于沸水中微煮,除净茸毛,去掉嘴壳。头和
翅盘在鸭背上,抹料酒、盐,加姜葱、二汤上笼蒸炤。

\step 鸭子在笼内取出晾冷,折去鸭身骨架,去掉四大骨及头、颈。再将鸭肉剔出,留完
整的鸭脯皮子,铺于二鱼碗内;鸭肉剁成细颗、剁匀,加味精、胡椒、盐、料酒拌匀,裝
于鸭皮面上,用纸盖着,上笼馏起待用。

\step 豆渣再用刀剁一次、剁细。热锅,温油,用文火炒。猪油可作两三次下,一直将豆
渣炒得与猪油混合时加盐少许,再继续将豆渣炒香,炒酥,吐油,呈深牙黄色。走菜时鸭
脯取出去纸,翻入盘中,将原汁泌入锅内,再加清汤与炒酥的豆渣混合炒匀,扞于鸭脯周
围即成。

\features

鸭味鲜美,豆渣香酥,圯嫩富油,吃时用调羹。

\end{recipe}

% vim: filetype=tex noautoindent
% vim: fileencoding=utf-8 formatoptions+=m
% vim: textwidth=78 tabstop=4 shiftwidth=4 softtabstop=4
