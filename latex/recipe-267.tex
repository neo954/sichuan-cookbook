\begin{recipe}{炸斑指}

\ingredients

\ingredient{猪肥肠头}{五根}
\ingredient{生姜(拍碎)}{二两}
\ingredient{青葱(切成寸节、}{二两}
\ingredient{白石凡(拍碎)}{少许}
\ingredient{醋}{二两}
\ingredient{干豆粉}{六钱}
\ingredient{绍酒}{六钱}
\ingredient{食盐、胡椒}{少许}

\cooking

肠头清洗干净。洗时加进生姜、青葱、白矾、醋等,以去其怪味。洗净后以沸水微煮(厨称“川一道”),捞起滤干,盛入碗里,加姜、葱、胡椒、食盐及绍酒等,置笼上蒸约两小时,俟蒸粑即取出,晾干水气,涂上干豆粉,以滚油炸之。炸时火不宜大,油不宜过滚,以防炸胡。至表皮现鸭黄色时,即捞起。斜切约一小指宽的小筒节,盛入盘内,配以葱、酱或椒盐,以荷叶饼佐食。

\notes

外脆内嫩,食化渣,味道香美。

\end{recipe}

% vim: filetype=tex noautoindent
% vim: fileencoding=utf-8 formatoptions+=m
% vim: textwidth=78 tabstop=4 shiftwidth=4 softtabstop=4
