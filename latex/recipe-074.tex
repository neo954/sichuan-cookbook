\begin{recipe}{三菌炖鸡}

\ingredients

\ingredient{仔鸡}{一斤半}
\ingredient{盐}{一钱五}
\ingredient{二汤}{二斤}
\ingredient{大葱}{三钱}
\ingredient{新鲜三菌}[\footnotemark]{一斤}
\ingredient{蒜}{二两五}
\ingredient{姜}{三钱}
\ingredient{化猪油}{二两}

\preparation

\step 三菌刮去菌身的粗皮和杂质,撕去老筋,除去菌足,淘洗干净;将菌顶大者切成四
牙,小者切成三牙;菌茎撕成两半,切成长一寸二分的段,用清水漂起。肥仔鸡连骨砍成
七分大的块。姜洗净,拍松,大葱去须,洗净。

\step 炒锅放炉上,放入猪油在锅内烧红,把姜、大葱、鸡块放入炒熟,加二汤、蒜等烧
开,舀入砂锅内用微火煨四十分钟左右。将三菌捞出滤干水气,再以猪油在炒锅内煸二分
钟后,仍倒入砂锅内,加盐,再用微火煨二十分钟,连汤带菜一起盛入碗中上席。

\features

此菜系汤菜,呈白色,鸡𤆵菌嫩,汤味鲜美。

\footnotetext{
三菌属于菌科,其形如伞,色白细嫩,味极鲜美,盛产于成都近郊。
}

\end{recipe}

% vim: filetype=tex noautoindent nojoinspaces
% vim: fileencoding=utf-8 formatoptions+=m
% vim: textwidth=78 tabstop=4 shiftwidth=4 softtabstop=4
