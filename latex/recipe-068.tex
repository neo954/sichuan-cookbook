\begin{recipe}{熘鸡米}

\ingredients

\ingredient{净鸡脯肉}{五两}
\ingredient{慈菇}{二两}
\ingredient{熟火腿}{一两}
\ingredient{白葱头}{一钱}
\ingredient{盐}{三分}
\ingredient{料酒}{二钱}
\ingredient{鸡蛋清}{一个}
\ingredient{干豆粉}{三钱}
\ingredient{水豆粉}{一钱}
\ingredient{味精}{一分}
\ingredient{胡椒面}{一分}
\ingredient{清汤}{一两}
\ingredient{香油或鸡油}{一钱}
\ingredient{化猪油}{四两耗一两五}

\preparation

\step 鸡脯肉切成如绿豆米大的丁,慈菇削去皮,火腿、整葱亦分别切成如绿豆米大的
丁,鸡蛋合干豆粉搅成蛋清豆粉,盐、料酒、水豆粉、味精、胡椒面、清汤合兑成滋汁。

\step 鸡肉丁用蛋清豆粉、盐、料酒拌和均匀,即投入炙过的五成热的油锅,用筷子滑
散,即投入火腿、茨菇,再滑散后将锅提离火口,泌去油剩约一两,再将锅放在火上,把
各丁拨在一边,放入葱末一煸,然后动作要快,烹入滋汁,用瓢浪一转,加香油或鸡油起
锅,盛入深色条盘。

\features

颜色调和,鲜嫩味美,酒饭均宜。

\end{recipe}

% vim: filetype=tex noautoindent nojoinspaces
% vim: fileencoding=utf-8 formatoptions+=m
% vim: textwidth=78 tabstop=4 shiftwidth=4 softtabstop=4
