\begin{recipe}{蝴蝶竹荪}

\ingredients

\ingredient{竹荪}{四钱}
\ingredient{鸡蛋}{五个}
\ingredient{鲜鱼肉}{四两}
\ingredient{肥膘肉}{二两}
\ingredient{鲜笋}{二两}
\ingredient{熟火腿}{一两}
\ingredient{黑芝麻}{一钱}
\ingredient{干豆粉}{五钱}
\ingredient{盐}{七分}
\ingredient{味精}{三分}
\ingredient{胡椒面}{二分}
\ingredient{料酒}{五钱}
\ingredient{酱油}{二钱}
\ingredient{猪瘦肉}{三两}
\ingredient{鸡脯肉}{三两}
\ingredient{清汤}{一斤半}
\ingredient{丝瓜}{三两}

\preparation

\step 将竹荪用温水泡胀,换清水洗数次,放在墩子上用刀修去两头,剖开成为六分宽、
二寸长的片子,共四十八片;将片的同一方向的两头修去呈半圆形,共修二十四片,余下
的二十四片用斜刀切断呈箭头形;将四十八片竹荪一并放入沸水内汆过,泌去清水,用好
汤喂一次,放入碗内装好待用。

\step 鸡蛋摊成蛋皮;鲜笋煮熟放在碗内;丝瓜刮后车下皮,洗净,切成二寸长的段,用
沸水沮熟,用清水漂起;再将丝瓜皮、蛋皮、鲜笋、火腿切成二寸长的二粗丝,分开装入
盘内。黑芝麻淘洗干净,捞起;半个蛋清拌成蛋清豆粉;鲜鱼肉去净皮翅,用刀背反复捶
茸(肉内不见籽为准);肥膘剁细〔如油脂)。再将鱼茸放入瓢内,三个半蛋清,八钱清
水,三分盐、剁好的肥膘茸及水发湿的干豆粉打成“鱼糁”装入碗内待用。

\step 将修去边角的竹荪摆成蝴蝶形,中间抹上蛋清豆粉,再将鱼糁刮于手心上呈尖刀形
,约一寸五长,放于抹豆粉处,二十四个照样摆完。再用铁夹将丝瓜火腿、蛋皮丝各夹一
丝入糁的中心,夹二根单丝插入糁的前面,放二粒芝麻在糁的上边,二十四个一起做完即
为蝴蝶竹荪。上笼蒸五分钟即熟(用时馏热)。鸡脯肉、猪瘦肉分别捶成茸子,并用清水
解散,装入碗内。

\step 将锅放在旺火上,倒下一斤半清汤加酱油、胡椒、盐,烧开后投入红茸子,推动数
转,微开时打尽肉沫;再投入白茸子,推动数转,微开时,同样打尽肉末,无杂质时,放
下味精,取出笼内的蝴蝶竹荪,用清汤过一次,翻入大汤盘内,再将锅内的清汤灌入即成
。

\features

形色美丽,汤鲜味美,适宜春夏二季。

\end{recipe}

% vim: filetype=tex noautoindent nojoinspaces
% vim: fileencoding=utf-8 formatoptions+=m
% vim: textwidth=78 tabstop=4 shiftwidth=4 softtabstop=4
