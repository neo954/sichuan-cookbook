\begin{recipe}{南煎丸子}

\ingredients

\ingredient{猪肉}{肥半斤瘦一斤(去骨皮)}
\ingredient{熟火腿}{一两}
\ingredient{水发兰片}{一两}
\ingredient{水发香菌}{一两}
\ingredient{鸡蛋}{二个}
\ingredient{慈菇}{六个}
\ingredient{菜心}{五两}
\ingredient{菜油}{一斤}
\ingredient{酱油}{五钱}
\ingredient{姜}{五钱}
\ingredient{葱白}{五钱}
\ingredient{水豆粉}{一两五}
\ingredient{香油}{二钱}
\ingredient{盐}{五分}
\ingredient{料酒}{五分}
\ingredient{胡椒、味精}{各少许}

\preparation

\step 菜心淘净,泹好;慈菇削皮;香菌洗净泥沙。

\step 猪肉切成小豌豆大的丁,火腿、兰片、香菌、慈菇,分别切成细丁;葱切细花;姜
去皮切细末。以上各材料加盐、酱油、水豆粉、鸡蛋(连黄)搅匀,共作成四个肉丸子,
略按扁。

\step 油烧至七成热,肉丸煎成金黄色,捞起入碗,加姜、葱,搭汤上笼,蒸𤆵。

\step 走菜时将丸子拈在盘内摆好,菜心用原汁加料酒、胡椒、盐、味精、酱油,吃味后
镶在菜的周围,余汁用水豆粉扯成滋汁,加香油淋上即成。

\features

颜色金黄,质地细软,鲜香可口上大圆盘。

\end{recipe}

% vim: filetype=tex noautoindent nojoinspaces
% vim: fileencoding=utf-8 formatoptions+=m
% vim: textwidth=78 tabstop=4 shiftwidth=4 softtabstop=4
