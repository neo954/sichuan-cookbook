\begin{recipe}{红烧鸽子}

\ingredients

\ingredient{鹤子三支}{约一斤半}
\ingredient{熟瘦火腿}{一两}
\ingredient{熟冬笋}{~一两五}
\ingredient{水发冬菇}{一两五}
\ingredient{花椒}{约十粒}
\ingredient{料酒}{三钱}
\ingredient{酱油}{五钱}
\ingredient{胡椒面}{二分}

\ingredient{盐}{一分}
\ingredient{葱白}{~一两}
\ingredient{姜}{二钱}
\ingredient{味精}{三分}
\ingredient{水豆粉}{三钱}
\ingredient{清汤}{二两五}
\ingredient{化猪油}{一斤耗二两}

\cooking

\step 	将三支鸽子宰杀后,去血退毛,清洗干净,从尾部的 背面横开一口,挖去内脏,去掉脚爪,用清水洗净,将鸽子 的两翅盘向鸽背,用手将盐、料酒、酱油、姜、花椒、葱等 放入鸽子的内外抹匀,放在盘内浸渍半小时。

\step 	将锅放在旺火上,放入化猪油,烧至七、八成热,将 渍好的鸽子捞起(剩下的作料待用)投入锅内,约炸五分钟捞 起,用蒸碗放好,将溃鸽子时用的作料再加上清汤,从鸽子上 淋下;用一种有拉力的皮筋纸蒙于碗口,放进蒸笼约蒸二点 半钟(根据鸽子的老嫩决定时间的长短〉,骨松翅裂为度。

\step 	将火腿、冬菇、冬笋分别切成三分见方的颗子;葱切 成三分长的段。

将蒸好的鸽子出笼,去掉碗口的皮筋纸及姜、葱、花 椒等不用,将碗内的鸽子摆于盘中,下面两个,上面一个堆 好。再将锅放在旺火上放下化猪油,将切好的火腿、冬菇、 冬笋的颗子一并投入锅内稍煽几下,将蒸鸽子时的原汁倒入 锅内,加入味精、胡椒,开锅后用豆粉勾芡,淋于鸽子上面 即成。

\notes

此菜颜色鲜美,味浓,鸽肉香嫩,富于营养,非常可 口,适宜春冬二季。

\end{recipe}

% vim: filetype=tex noautoindent
% vim: fileencoding=utf-8
% vim: textwidth=78 tabstop=4 shiftwidth=4 softtabstop=4
