\begin{recipe}{银杏鸭脯}

\ingredients

\ingredient{鸭子一只}{约二斤半}
\ingredient{胡椒}{三分}
\ingredient{葱}{三钱}
\ingredient{舞鲜白果}{八两}
\ingredient{盐}{四分}
\ingredient{花椒~}{约十粒}

\ingredient{料酒}{二两}
\ingredient{化猪油}{一斤耗'一五}
\ingredient{清汤}{六两}
\ingredient{味精}{三分}
\ingredient{姜}{三钱}
\ingredient{鸡油}{五钱}

\cooking

\step 白果捶破,去硬壳,在开水内煮熟,撕去皮膜,用刀切去两头,透去心,用开水汆去苦水,在猪油锅内炸一下捞起待用。

\step 鸭子宰杀后去毛去内脏,从杀口处宰去头及足,洗净,滴干水气,用盐、胡椒、料酒将鸭身内外抹匀,放入葱、姜、花椒,上笼蒸约一小时取出,去掉姜、葱、花椒,用刀从背脊处宰开,取净全身大小骨头后铺入二鱼碗,齐碗口修去边沿,成圆形定好。修下的鸭肉切成与白果大的丁颗,连同白果造匀,放于鸭脯上,将原汁泌入,加汤上笼蒸半小时,至粑滥翻入圆盘内。

\step 锅内渗清汤,加余下的枓酒、盐、味精、胡椒、水豆粉少许勾二流芡,放鸡油,挂白汁于鸭上即成。

\notes

质地鲜嫩,味美营养。

\end{recipe}

% vim: filetype=tex noautoindent
% vim: fileencoding=utf-8
% vim: textwidth=78 tabstop=4 shiftwidth=4 softtabstop=4
