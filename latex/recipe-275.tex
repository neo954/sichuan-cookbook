\begin{recipe}{叉烧全鸡}

\ingredients

\ingredient{嫩肥母鸡〔约三斤}{半左右)}
\ingredient{一只}{}

\ingredient{生姜丝}{少许}
\ingredient{葱丝}{少许}

\ingredient{肥瘦猪肉丝}{二两}
\ingredient{酱油}{少许}
\ingredient{榨菜丝}{一两}
\ingredient{食盐}{少许}
\ingredient{花椒}{十余颗}
\ingredient{小磨麻油}{六钱}
\ingredient{泡海椒}{二个}
\ingredient{饴糖}{三钱}

\cooking

鸡杀后去毛洗净,去脚,肚腹小开(从夹窝上开口),取出内脏。将肉丝,榨菜丝、花椒、泡海椒、姜丝、葱丝、酱油、食盐等,用麻油在锅中炒好,从鸡的夹窝口灌入腹内。

用开汤淋于鸡身上,到鸡皮绷伸时抹干鸡身水气,将绐糖揉散均匀地抹在鸡身上,将翅膀折断,莫让其跷起,晾干水气,即行上叉。叉子自胸脯伸向尾部,连腿叉上,以斗方杠炭火烧烤约一小时,鸡皮成黄红色时即成。

烧好后,将腹内香料取出,将鸡皮和肉分别片成片子。盘子的中间放肉,一端放皮,另一端放肚腹内取出的香料。

另备白油调料调味,并备火夹饼、葱佐食。

\notes

皮酥肉嫩、味香鲜美。

\end{recipe}

% vim: filetype=tex noautoindent
% vim: fileencoding=utf-8
% vim: textwidth=78 tabstop=4 shiftwidth=4 softtabstop=4
