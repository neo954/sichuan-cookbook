\begin{recipe}{蒸枣糕}

\ingredients

\ingredient{陕枣}{五两}
\ingredient{猪板油}{三两五}
\ingredient{鸡蛋}{二个}
\ingredient{红苔}{三两}
\ingredient{慈菇}{二两}
\ingredient{网油}{二两}
\ingredient{瓜片}{五钱}
\ingredient{核桃米}{一两}
\ingredient{玫瑰}{二钱}
\ingredient{白糖}{三两}

\preparation

\step 用铁丝漏瓢将枣子盛着,置旺火上把枣皮烧焦,边烧边簸动,烧至每个枣子都起黑
壳时,倒入冷水中泡约五分钟,捞起将黑壳去掉,并去掉枣核。

\step 核桃米入沸水中泡约二分钟,捞起去皮,入油锅中炸呈黄色捞起;慈菇削皮洗净;
红苕煮熟去净皮。

\step 猪板油去皮去筋,与剥出的枣肉分别用刀宰茸成泥;熟红苕研茸;核桃米、慈菇、
瓜片各切成细丁。

\step 宰出的枣泥、板油连同红苕泥用盆子盛着,加鸡蛋入内,用手搅匀后放入核桃米、
瓜片、慈菇、白糖、玫瑰等,再继续搅为混合泥。

\step 网油铺于蒸碗中垫底,油边吊在碗口外,用手理伸,把拌好的混和泥放入网油上
面,用手抹平整,再将吊在碗边的网油搭转来盖着,碗口用浸湿的厚纸封密,上笼用旺火
蒸约四十分钟,出笼翻入盘内,揭去网油不用,上面撒白糖入席。

\features

味甜香,软嫩适口,一般用于席桌冬季甜菜。

\end{recipe}

% vim: filetype=tex noautoindent nojoinspaces
% vim: fileencoding=utf-8 formatoptions+=m
% vim: textwidth=78 tabstop=4 shiftwidth=4 softtabstop=4
