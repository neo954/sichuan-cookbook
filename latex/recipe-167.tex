\begin{recipe}{红烧熊掌}

\ingredients

\ingredient{熊掌}{一对约三斤}
\ingredient{肥母鸡}{半只}
\ingredient{猪肉}{二斤}
\ingredient{大葱}{二斤}
\ingredient{化猪油}{二两五}
\ingredient{酱油}{六钱}
\ingredient{味精}{三分}
\ingredient{料酒}{一斤}
\ingredient{姜}{二两五}
\ingredient{盐}{七分}
\ingredient{胡椒面}{三分}
\ingredient{二汤}{二十二斤}

\cooking

\step 选肥嫩熊前掌一对,重约三斤左右(重量不包括熊腿),用清水十五斤在锅中用旺火煮一点半钟,捞出,去净茧巴,用铁夹子拈净茸毛,洗刷干净待用。

\step 将肥母鸡切成七分长、四分大的条方块;选连皮肥瘦猪肉二斤,切成一寸五长、二分厚的条方片子;大葱掐去须.

评口老皮,切下葱白,将葱叶分成三份待用;姜分切为四块(每 块五钱),拍松。

\step 将锅放炉上烧红,倒入猪油,再放入姜、葱叶一份稍炒,随即加入二汤、料酒和熊掌,煮十分钟后捞出熊掌,倒去锅中各料不用。按此作法和用料再继续煮三次后,捞出熊掌把骨剔尽。

\step 猪油五钱入锅,放入猪肉、鸡块煸炒,再加入酱油、姜、盐、二汤(二斤X、胡椒面、味精、料酒、葱白、熊掌等,用微火烧粑。然后取大圆盘一个,先将锅中葱白拣入,再将熊掌掌心向上盖在葱白上。锅中鸡肉等物捞去不用,用旺火将原汁熬酽,淋于熊掌上入席。

\notes

此菜肉烂味香,宜用于高级筵席。因熊掌膻味大,处理 过程要细致。

\end{recipe}

% vim: filetype=tex noautoindent
% vim: fileencoding=utf-8 formatoptions+=m
% vim: textwidth=78 tabstop=4 shiftwidth=4 softtabstop=4
