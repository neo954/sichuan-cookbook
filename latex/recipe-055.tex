\begin{recipe}{罈子肉}

\ingredients

\ingredient{猪肘子}{(五斤)一个}
\ingredient{酱油}{半斤}
\ingredient{𧎼蛀}{二两五}
\ingredient{肥鸭}{(四斤)一只}
\ingredient{冰糖汁}{二两五}
\ingredient{鸡蛋}{十个}
\ingredient{冬笋}{三斤}
\ingredient{姜}{二两}
\ingredient{猪骨}{二斤}
\ingredient{海参}{半市}
\ingredient{胡椒}{约三十颗}
\ingredient{开水}{十二斤}
\ingredient{墨鱼}{半斤}
\ingredient{肥母鸡}{(四斤)一只}
\ingredient{咸红酱油}{四两}
\ingredient{大金钩}{二两五}
\ingredient{肥瘦火腿}{一斤}
\ingredient{盐}{二钱}
\ingredient{干豆粉}{二两五}
\ingredient{口茉}{二两五}
\ingredient{大葱}{四两}
\ingredient{料酒}{四斤}
\ingredient{鱼翅}{半斤}
\ingredient{化猪油}{二斤(耗三两)}

\cooking

\step 海参用温水泡十六小时,洗净泥砂,挖去肠脏,切成长一寸、宽六分的条方块;用
二汤汆二次,用清水漂冷心,再用净布一方包好。鱼翅用温水泡十分钟,洗净上面的胶
水,去净子骨和杂质,放入大碗中,加清水一斤半,上笼蒸一小时,出笼泌去水,再用二
汤汆一次,用清水漂冷心,用稀眼净布一方包好。摇蛀用温水泡十分钟,洗净后用稀眼净
布包好。口茉用温水泡五分钟,洗净泥砂,去净根和杂质,用稀眼净布包好。大金钩洗
净。墨鱼用温水泡半小时,去净骨和杂质,将每条撕成两半。猪骨洗净。猪肘子肉去净茸
毛,刮洗干净,切成四大块。肥母鸡宰杀、去毛、去肠脏,切成两半,每半再切成两块。
肥瘦火腿刮洗干净,切成一寸二长、六分宽的条块。冬笋去壳,去老根,取用嫩尖部分,
切成一寸二长长、六分宽的条块。肥鸭宰杀,去。毛、去脏,洗净后切成两半,每半再切
成两块。
\step 锅中倒入清水十斤,放入鸡蛋,用旺火烧开,再将猪骨、猪肘子、鸡肉、鸭肉等放
入锅中煮五分钟。随后全部捞出,放入清水中漂冷心,洗净肉上泡沫,将猪骨、猪肘子、
鸡肉、鸭内盛入碗中待用。煮熟的鸡蛋去壳,漂入清水中;再把锅中水倒去,放入猪油烧
红,将去壳鸡蛋由水中捞起,滤去水分,放入干豆粉,裹满豆粉,放入油中炸成深黄色,
捞出待用。
\step 姜洗净拍松。大葱去鬚洗净。用稀眼净布一方把姜、大葱、胡椒等包好待用。
\step 准备干锯末二十斤。在平整干燥的地坪上,安上铁质三脚架〈高五寸〉。架内倒上
锯末五斤,架底处的锯末留一空窝,内放烧红的杠炭三段〈每段长约五寸〉,再把大口料
酒罈(内外有釉、无裂缝者)一个安于三脚架上。然后将猪骨垫于罈底和罈内四周,再将
开水、料酒、姜、葱布包、盐、红白酱油、火腿、猪肘子肉、鸡肉、鸭肉、冬笋、口茉布
包、墨鱼、螭蛀、大金钩、冰糖汁等依次放入罈中。用纸封固罈口,再把锯末十五斤堆于
罈外四周,煨四、五小时(在煨的过程中,要有专人负责看火,不能用旺火,也不能断
火)后,撕去罈口草纸,将鱼翅布包,海参布包,炸好的鸡蛋放入罈中,再煨半小时即
成。
\step 此菜吃法有两种:一种是将各种肉分开,分别装盘上席;另一种是将各种肉分小,
镶盘上席。此菜一般适于冬季制作,一次吃不完,可以分几次吃。上述配料可供二十人吃
一餐。

\notes

此菜在成都著名已久,由于用料复杂,水分较少,加之封固罈口微火煨焙而成,保持了各
种材料的原汁和原气。因此,菜呈金红色,鲜艳美观,肉极𤆵①烂,鲜美而浓香,可家常自
吃,又可用于筵席。

①𤆵:音“怕”阴平声,系四川土语。意思是肉烂了,但仍保持体态的完整。

\end{recipe}

% vim: filetype=tex noautoindent
% vim: fileencoding=utf-8
% vim: textwidth=78 tabstop=4 shiftwidth=4 softtabstop=4
