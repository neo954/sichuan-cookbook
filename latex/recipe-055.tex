\begin{recipe}{罈子肉}

\ingredients

\ingredient{猪肘子}{(五斤)一个}
\ingredient{酱油}{半斤}
\ingredient{𧎼蛀}{二两五}
\ingredient{肥鸭}{(四斤)一只}
\ingredient{冰糖汁}{二两五}
\ingredient{鸡蛋}{十个}
\ingredient{冬笋}{三斤}
\ingredient{姜}{二两}
\ingredient{猪骨}{二斤}
\ingredient{海参}{半市}
\ingredient{胡椒}{约三十颗}
\ingredient{开水}{十二斤}
\ingredient{墨鱼}{半斤}
\ingredient{肥母鸡}{(四斤)一只}
\ingredient{咸红酱油}{四两}
\ingredient{大金钩}{二两五}
\ingredient{肥瘦火腿}{一斤}
\ingredient{盐}{二钱}
\ingredient{干豆粉}{二两五}
\ingredient{口茉}{二两五}
\ingredient{大葱}{四两}
\ingredient{料酒}{四斤}
\ingredient{鱼翅}{半斤}
\ingredient{化猪油}{二斤(耗三两)}

\cooking

\step 海参用温水泡十六小时洗净泥砂挖去肠脏切成长一寸宽六分的条方块用二江伟二次用清水冷心,尝54晓
再用净布一方包好鱼翅用温水泡十分钟洗净上面的水去子骨和杂质放入大硫中加清水一斤半上笼一小时出笼泌去水再用二汤伟一次用清水漂冷心用稍眼净布一方包好蜒蛀用温水泡十分钟洗后用稀眼净布包好口苗用温水泡五分钟洗浑泥砂去根和杂质用稀眼净布包好大金钩洗洗墨鱼用温水泡半小时,去净骨和杂质将条成两半猪骨洗洗猪肘子肉去浑茸毛刮洗干净切成四大块肥母宰杀去毛去肠脏切成两半每半毋切成两块肥瘦火腹刮洗干净切成一寸二长六分宽条块冬笋去享去老根取嫩尖部分切成一寸二长六分宽条块。肥鸭宰杀去.毛去脏洗洗后切成两半每半再切成两。
\step 锅中倒入清水十斤放入鸡蛋用旺火烧升再将骨猪肘子鸡肉鸭肉等放入锅中焯分钟随后全部捞出放入清水中漂冷心洗净肉上泡证将猪骨猪肘、鸡肉鸭肉入硫中待用煮熟鸡蛋去声漆入清水中;再把锅中水倒去放入猪油烧红将壳鸡蛋由水中捞,滤去水分放入干豆粉褐满豆粉放入油中炸成深黄,搅出待。
\step 姜洗净拍松大去蚊洗洗用稀眼净布一方把、大葱胡椒包好彼。
\step 准备干锯未二十斤在整干燥的地坪上安上铁三脚(高五才。架内倒上锯未五斤架底处的锯末留一空窝内放烧红的杠炳三(每段长约五才,再把大口料酒(内外有釉无裂缝者一个安于三脚架上然后将骨于罐底和罄内四周再将开水料溯姜葱包、盐红白酱油火腹猪肘子肉鸡肉职肉冬笋口。55。茉市包墨鱼蟀蛀大金钩冰糠汁等依次放入露中。用纸封固缴口再把锯未十五斤堆于罄外四周熬四五小时(在烈的过程中要有专人负责看火不能用旺火也不能鸡蛋放入铁中再熬半小时即成。
\step 此菜吃法有两积一种是将各种肉分开分别装盘席古一种是将各积肉分小镶盘上席此菜一般适冬制作一次吃不完可以分几次吃上述配料可供十人吃一一餐。

\notes

此菜在成都著名已丿由于用料复杂水分较少加封固缬口微火熵焙而成保持了各种材料的原汁和原气因此菜昔金红色鲜艳美观肉传①〇烂鲜而浸香司常自吃又可于筵。

①炳音《怕“阴平声系四川土语意心是肉烂,但仍保持体的完。

\end{recipe}

% vim: filetype=tex noautoindent
% vim: fileencoding=utf-8
% vim: textwidth=78 tabstop=4 shiftwidth=4 softtabstop=4
