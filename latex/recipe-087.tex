\begin{recipe}{牡丹鸡片}

\ingredients

\ingredient{母鸡脯}{四两}
\ingredient{水豆粉}{少许}
\ingredient{盐}{一钱}
\ingredient{干豆粉}{三两}
\ingredient{鸡油}{二钱}
\ingredient{料酒}{少许}
\ingredient{鸡蛋清}{三个}
\ingredient{熟火腿}{一两}
\ingredient{味精}{二分}
\ingredient{面粉}{二钱}
\ingredient{小白菜嫩心}{数朵}
\ingredient{胡椒面}{一分}
\ingredient{化猪油}{一斤耗二两}
\ingredient{水发口茉}{三朵}

\preparation

\step 小白菜心淘洗干净,火腿片成长一寸二、宽八分极薄的片,口茉片薄。鸡蛋清先在
碗内用力一股劲快速搅成蛋泡如雪花,再将面粉倒下,调匀成略带粘性的蛋清汁

\step 鸡脯用刀片成极薄的片,长一寸二、薄八分;好干豆粉在墩上扞细用箩筛过,连同
鸡片,逐片用刀背轻轻细砸,使鸡片再向四周展伸而薄,但要砸至成片不滥。

\step 猪油烧至四成火候的温油,用筷将鸡片拈在调好的蛋清汁内粘一层入锅浸炸,炸泡
拈起,边炸边拈,保持白色,盛入盘。

\step 另用油少许,将小白菜心、火腿、口末編炒,即惨入鸡汤、水豆粉、盐、料酒、味
精、胡椒面烹成白汁,再将鸡倒下几簸,淋鸡油起锅。

\features

味鲜嫩,色美观,富营养,老幼均宜。

\end{recipe}

% vim: filetype=tex noautoindent nojoinspaces
% vim: fileencoding=utf-8 formatoptions+=m
% vim: textwidth=78 tabstop=4 shiftwidth=4 softtabstop=4
