\begin{recipe}{软炸口茉}

\ingredients

\ingredient{口茉}{二两五}
\ingredient{猪油}{一斤约耗三两}
\ingredient{鸡蛋清}{三个}
\ingredient{鸡汤}{五两}

\ingredient{盐}{六分}
\ingredient{干豆粉}{一'两}
\ingredient{味精}{二分}
\ingredient{料酒}{六钱}
\ingredient{姜}{二钱}
\ingredient{胡椒面}{}
\ingredient{葱}{二钱}
\ingredient{香油}{二钱}
\ingredient{蕃获酱}{二两}
\ingredient{白糖}{少许}

\cooking

\step 口茉选用大小均匀的,用清水泡半小时泌去水作别用,再换清水淘洗去泥沙。洗净后用沸水煮十分钟,而后再换水煮十分钟,发至胀透,即成水发口茉。鸡蛋清与豆粉调成干湿适度的蛋清豆粉。

\step 猪油放入锅内,在旺火上烧至五成热,放入葱、姜稍煸,即依次加入鸡汤、口茉、料酒、盐、味精、胡椒面等烧约八分钟,用汤瓢将口茉捞出,入筲箕内滤干水,放入蛋清豆粉内拌匀,使每块口茉都能裹上一层。

\step 猪油放入锅内,于旺火上烧至五成热,将裹好蛋清豆粉的口茉逐个放入稍炸。若火过旺即将锅端离火口,炸至口茉上的蛋清豆粉刚熟(看不见黄色仍是白的)即捞入盘中。要边放边捞,以免炸糊。若有粘连,可用手掰开。再将锅在旺火上烧至八成热,将盘中炸过的口茉全部倒入,约炸五分钟呈金黄色时泌去炸油,随即淋入香油,捞出盛入盘中。另以四个小碟盛入蕃茄酱〔蕃茄酱加香油、白糖、味精、盐各少许)蘸食。

\notes

此菜口茉皮酥内嫩,清香可口。

另一制作法:不裹蛋清豆粉而用糁裹口茉,味更鲜嫩。如 用糁,原料上须增加鸡肉二两五或鱼肉半斤、肥膘肉四两, 作打糁用。

\end{recipe}

% vim: filetype=tex noautoindent
% vim: fileencoding=utf-8 formatoptions+=m
% vim: textwidth=78 tabstop=4 shiftwidth=4 softtabstop=4
