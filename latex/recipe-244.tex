\begin{recipe}{火把玉兰片}

\ingredients

\ingredient{水发玉兰片}{四两}
\ingredient{网油}{一两}
\ingredient{味精}{二分}
\ingredient{熟鸡肉}{二两}
\ingredient{鸡油}{二钱}
\ingredient{水豆粉}{五钱}
\ingredient{熟火腿}{二两}
\ingredient{清汤}{四两}
\ingredient{带皮}{五钱}
\ingredient{料酒}{三钱}

\cooking

\step 玉兰片、鸡肉、火腿各切成长两寸的二粗丝;带皮发后淘洗干净,切成长五寸的细
丝。

\step 将兰片、火腿、鸡丝两头砌齐,用带丝一根从中间缠成一指大的小把三十六把(每
把包括三种丝子),两头留五分长不缠,象火把状。

\step 用蒸碗一个铺上网油,将缠好的把,摆成三叠水在碗内定好,掺少许汤上笼蒸约十
分钟取出。

\step 走菜时将网油揭去,翻入盘中。锅内掺清汤,加盐、料酒、味精、豆粉,勾二流芡
成白汁,起锅时加鸡油淋于菜上即成。

\notes

此菜风格朴素,作法简单,在四、五十年前走于一般海 参席桌中的“热吃”之一,菜美汤
鲜。

\end{recipe}

% vim: filetype=tex noautoindent
% vim: fileencoding=utf-8 formatoptions+=m
% vim: textwidth=78 tabstop=4 shiftwidth=4 softtabstop=4
