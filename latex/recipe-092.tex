\begin{recipe}{熘桃鸡卷}

\ingredients

\ingredient{鸡脯肉}{五两}
\ingredient{桃米}{二两}
\ingredient{熟火腿}{一两五}
\ingredient{水发口茉}{一两}
\ingredient{鸡蛋}{三个}
\ingredient{盐}{三分}
\ingredient{味精}{三分}
\ingredient{料酒}{二钱}
\ingredient{建兰菜}{五两}
\ingredient{化猪油}{一斤耗二两五}
\ingredient{干豆粉}{一两}
\ingredient{清汤}{三两}
\ingredient{香油}{二钱}

\preparation

\step 先将鸡脯洗净,对剖,脯子片成五分宽的长片,大约三十片;再将火腿切成三至四
分见方的片;口茉片成薄片。分开放入碗内。桃米发胀,选整瓣的,撕去皮渣;蛋清及豆
粉拌成蛋清豆粉。

\step 将片好的鸡片平铺于案上,抹上蛋清豆粉,每片内各放一个桃米,火腿一片,口茉
一片,再将鸡片裹好成鸡卷形。

\step 将锅置于旺火上,放入化猪油一斤,烧至六成火,再将卷好的鸡卷在拌匀的蛋清豆
粉内裹一转,放入锅内炸透捞起,放在圆盘内。泌去锅内余油,放入建兰菜炒熟,再加少
许料酒、盐,起锅入走菜盘内;再放入一两化油入锅内,烧至六成火,倒下兑好的滋汁
(盐、料酒、味精、豆粉、清汤)成二流芡,再将炸好的鸡卷滑入锅内,起锅淋入香油,
扦入建兰菜的面上即成。

\features

此菜脆嫩鲜香,颜色美观。

\end{recipe}

% vim: filetype=tex noautoindent nojoinspaces
% vim: fileencoding=utf-8 formatoptions+=m
% vim: textwidth=78 tabstop=4 shiftwidth=4 softtabstop=4
